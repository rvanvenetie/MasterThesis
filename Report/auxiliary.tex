\begin{document}
  A lower bound for $\uanorm{\vsiga}_{\w_a}$ can be given in terms or $r_a$.
  By the definition of the local equilibrium property~\eqref{eq:defsigma} we have $\ip{r_a, v} = - \ip{\v{\sigma_a}, \nabla v}_{\w_a}$ for $v \in H_\star^1(\w_a)$,
  and thus
  \[
    \norm{r_a}_{H_\star^1(\omega_a)'} = \sup_{\{v \in H_\star^1(\w_a) : \norm{\nabla v}_{\w_a} = 1\}} {\ip{r_a, v}} = \sup_{\{v \in H_\star^1(\w_a) : \norm{\nabla v}_{\w_a}=1\}} {\ip{\v{\sigma_a}, \nabla v}_{\w_a}} \leq \uanorm{\v{\sigma_a}}_{\w_a}.
  \]

  Fix an interior vertex  $a \in \V^{int}$. Applying the above and using orthogonality on constant functions reveals  
  \begin{align*}
    \uanorm{\v{\sigma_a}}_{\w_a} &\geq \norm{r_a}_{H^1_\star(\w_a)'} \\
    &= \sup_{\{v \in H^1_\star(\w_a) : v \ne 0\}} { \ip{r_a, v} \over \norm{\nabla v}_{\w_a}} \\
    &= \sup_{\{v \in H^1(\w_a) : v \not \equiv C_{st}\}} { \ip{r_a, v - v_{\w_a}} \over \norm{\nabla (v - v_{\w_a})}_{\w_a}} \\
    &= \sup_{\{v \in H^1(\w_a) : v \not \equiv C_{st}\}} { \ip{r_a, v} \over \norm{\nabla v}_{\w_a}}.
  \end{align*}
  Similar to the previous proof we decompose $r_a$ in triangle and edge terms
  \[
    \ip{r_a,v} = \sum_{K\subset \w_a} \langle{r^T_a, v}\rangle_K + \sum_{e \subset \gamma_a} \ip{r^e_a, v}_e,
    = \sum_{K\subset \w_a} \langle{\psi_a \left [f+\Delta U \right], v}\rangle_K + \sum_{e \subset \gamma_a} \ip{\psi_a \llbracket \nabla U \rrbracket, v}_e.
  \]
  The triangle and edge functions $(r^T_a, r^e_a)$ are actually broken polynomials from the spaces $\P_r^{-1}(\w_a)\times \P_r^{-1}(\gamma_a)$ for some degree $r$.

  Start with an element $K \subset \w_a$, and consider the space $H_0^1(K)$. Since a function $v \in H_0^1(K)$ vanishes on the boundary, 
  we can naturally extend it to a function $v \in H^1(\w_a)$ by setting $v \equiv 0$ on $\w_a \setminus K$. As $v$
  vanishes all edges $e \subset \gamma_a$ and all triangles except $K$, we have $\ip{r_a, v} = \langle{r_a^T, v}\rangle_K$ and thus
  \[
    \sup_{\{v \in H^1(\w_a) : v \not \equiv C_{st}\}} { \ip{r_a, v} \over \norm{\nabla v}_{\w_a}} 
    \geq \sup_{\{v \in H_0^1(K) : v \ne 0\}} { \langle{r^T_a, v}\rangle_K \over \norm{\nabla v}_{K}}
    = \sup_{\{v \in H_0^1(K) :  \norm{\nabla v}=1\}} { \langle\psi_a\left[ f + \Delta U\right],v\rangle_K }
  \]
  An estimate for this last term can be found using equivalence of norms (cf. \cite[Ex~9.x.5]{brenner}).
  
  That is, we claim that $\norm{q} := \sup_{\{v \in H_0^1(K) : \norm{\nabla v}=1\}} \langle \psi_a q, v\rangle_K$
  defines a norm for polynomials $q$ of degree $r$ on $K$. The only troublesome axiom to be satisfied
  is that $\norm{q} = 0$  implies $q=0$. Suppose that $q$ is a non-zero polynomial.
  Then there exists a point $x \in \inte(K)$ for which we have $q(x) \ne 0$, and without loss of generality we suppose that $q(x) > 0$.
  Since $q$ is continuous we have $q(y) > q(x) / 2 > 0$ for $y$ in $V(x)$, a small neighborhood of $x$.
  The hat function satisfies $\psi_a > 0$ in the interior of $K$. Taking $V(x)$ small enough therefore guarantees that $\psi_a q>0$ on $V(x)$.
  Finally, choose a continuous $v \in H_0^1(K)$  with $v(x) = 1$ and $v = 0$ outside of $V(x)$. For this $v$ we then find
  \[
    \norm{q} \geq { \langle  \psi_a q, v\rangle_K \over \norm{\nabla v}_K}  = {\langle \psi_a q, v\rangle_{V(x)} \over \norm{\nabla v}_K} > 0,
  \]
  because $\psi_a q \ne 0 \ne v$ in a small non-empty open neighborhood of $x$. Hence $\norm{\cdot}$ is a norm on the space of polynomials with degree $r$.
  
  The above also holds for a reference element $\hat K$. Using equivalence of norms, and the usual transformation lemma
  shows that for all elements $K \subset \w_a$ we have
  \[
    \sup_{\{v \in H_0^1(K) :  \norm{\nabla v}=1\}} { \langle\psi_a\left[ f + \Delta U\right],v\rangle_K } \gtrsim h_K \norm{ f + \Delta U }_K,
  \]
  for a constant depending on the shape regularity and (unfortunately) on the polynomial-degree $p$ used in $\VV$.  This
  provides a lower bound for the element related terms.
  
  
  Next, fix an edge $e \subset \gamma_a$ and denote $K_1, K_2$ for the triangles that share this edge. Consider the following space of functions:
  \[
  V_e := \set{v \in H_0^1(K_1 \cup K_2) : \ip{v,P}_{K_1}= \ip{v,P}_{K_2} = 0\quad \forall P \in \P_r}.\]
  Again we can naturally extend these functions to live in $H^1(\w_a)$. Since $v \in V_e$ vanish on all edges except $e$, and
  is orthogonal to polynomials of degree $r$ on $K_1$ and $K_2$, we have
  \[
    \sup_{\{v \in H^1(\w_a) : v \not \equiv C_{st}\}} { \ip{r_a, v} \over \norm{\nabla v}_{\w_a}} 
    \geq \sup_{\{v \in V_e: v \ne 0\}} { \langle r^e_a, v \rangle_e \over \norm{\nabla v}_{K_1 \cup K_2}}
    = \sup_{\{v \in V_e: v \ne 0\}} { \langle \psi_a \llbracket \nabla U \rrbracket, v \rangle_e \over \norm{\nabla v}_{K_1 \cup K_2}}
          \gtrsim h_e^{1/2} \norm{\llbracket \nabla U \rrbracket}_e.
  \]
  The last inequality is a variation of a well known result \cite[Ex~9.x.7]{brenner}. It can be proven by using the same
  technique we used for the element terms. The constant depends on both shape regularity and the polynomial-degree $p$.

  The last step consist of summing the above inequalities over $K \subset \w_a$ and $e \subset \gamma_a$, i.e.
  \begin{align*}
    \uanorm{\vsiga}_{\w_a} \geq \norm{r_a}_{H_\star^1(\w_a)'} &\gtrsim \sum_{K \subset \w_a} h_K \|f + \Delta U\|_K + \sum_{e \subset \gamma_a} h_e^{1/2} \norm{\llbracket \nabla U \rrbracket}_e\\
    &\geq \left(\min_{K \subset \w_a} h_K\right) \sum_{K \subset w_a} \norm{f + \Delta U}_K +  \left(\min_{e \subset \gamma_a}h_e^{1/2} \right) \sum_{e \subset \gamma_a} \norm{\llbracket \nabla U \rrbracket}_e\\
    &\gtrsim h_{\w_a} \norm{ f + \Delta U}_{\w_a} + h_{\w_a}^{1/2} \norm{ \llbracket \nabla U \rrbracket}_{\gamma_a}.
  \end{align*}
  The hidden constants depend the shape regularity and the polynomial-degree $p$. The proof is similar for $a \in \V^{bdr}$.
\end{proof}
\begin{cor}
  \label{lem:starequiv}
  The estimator $\uanorm{\v{\sigma_a}}_{\w_a}$ is equivalent to the classical estimator on the patch~$\w_a$, 
  \[
    \uanorm{\v{\sigma_a}}_{\w_a} \approx h_{\w_a}\norm{f + \Delta U}_{\w_a} + h^{1/2}_{\w_a} \norm{\llbracket \nabla U  \rrbracket}_{\gamma_a},
  \]
  for a constant depending on the shape regularity and the polynomial-degree $p$ used in~$\VV$.
\end{document}
