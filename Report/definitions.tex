\documentclass[thesis.tex]{subfiles}
\begin{document}
\section{Sobolev spaces}
A proof of the following theorems can be found in most standard works, e.g. \cite{gaticasimple}.
\begin{thm}
  \label{thm:green}
  Let $\O$ be a bounded domain with Lipschitz boundary, then for all $v,w \in H^1(\O)$  there holds
  \[
    \int_{\O} v \pd{w}{x_i} = - \int_{\O} w \pd{v}{x_i} +   \int_{\partial \O} vw n_i \quad i \in {1,\dots, n},
  \]
  so that for  $u \in H^2(\O)$ we have
  \[
    \int_{\O} v\Delta u = - \int_{\O} \nabla u \cdot \nabla v + \int_{\partial \O} v \nabla u \cdot \v{n}.
  \]
  Here the functions on the boundary are to be interpreted using the trace operator.
\end{thm}
\begin{defn}
  \label{def:hdiv}
The weak divergent Sobolev space is defined by
\[
  H(\div; \O) := \set{ \v{\sigma} \in [L^2(\O)]^2 : \div \v{\sigma} \in L^2(\O)}.
  \]
  Where $\div \v{\sigma} \in L^2(\O)$ is interpreted in distributional sense, i.e. there exists a $v\in L^2(\O)$ such that
  \[
    - \int_{\O} \nabla \f \cdot \v{\sigma} = \int_{\O} v\f\quad \forall \f \in D(\O).
  \]
  This turns $H(\div; \O)$ into a Hilbert space with inner product
  \[
    \ip{\v{\sigma}, \v{\tau}}_{\div,\O} :=  \ip{\v{\sigma}, \v{\tau}}_{\O} + \ip{\div \v{\sigma}, \div \v{\tau}}_{\O}\quad \text{ for } \v{\sigma}, \v{\tau} \in H(\div; \O).
  \]
\end{defn}
\begin{thm}
  \label{thm:divergence}
  Let $\O$ be a bounded domain with Lipschitz boundary, then for all $\sigma \in H(\div; \O)$ and $v \in H^1(\O)$ we  have
  \[
    \int_\O \v{\sigma} \cdot \nabla v + \int_\O  v \div \v{\sigma} = \int_{\partial \O} v \v{\sigma}\cdot \v{n}
  \]
\end{thm}
Actually the right hand side is an abuse of notation, formally one has that $\v{\sigma}\cdot\v{n}$ is an operator working on the
trace of $v$ on $\partial \O$. 
\section{Poincar\'e-Friedrichs inequality}
\label{sec:poincfried}
For some $H^1$ functions it is possible to (universally) bound the $L^2$-norm by its $H^1$-seminorm.
Both Poincar\'e and Friedrichs' inequality provide such bounds. Due to their similarity these type 
of inequalities are named \emph{Poincar\'e-Friedrichs inequality}.
\begin{thm}[Friedrichs' inequality]
  Suppose that $\O \subset \R^n$ is a  bounded Lipschitz domain, then for $h_\O$ the diameter of $\O$ we have
  \[
    \norm{v}_{\O} \leq h_\O  \norm{\nabla v}_{\O} \quad \forall v \in H^1_{0}(\O).
  \]
  For $\Gamma \subset \partial \O$ with positive $(n-1)$-dimensional measure there exists
  a constant $C_F(\O, \Gamma)$ such that
  \[
    \norm{v}_{\O} \leq C_{F} h_\O \norm{\nabla v}_{\O} \quad \forall v \in H^1(\O), v|_{\Gamma} = 0.
  \]
\end{thm}
\begin{thm}[Poincar\'e inequality]
  Again suppose that $\O \subset \R^n$ is a bounded Lipschitz domain, then there exists a constant $C_P(\O)$ such that
  \[
    \norm{v - v_{\O}}_{\O} \leq C_Ph_\O \norm{\nabla v}_{\O} \quad \forall v \in H^1(\O).
  \]
\end{thm}
Here $C_F$ and $C_P$ --- combined $C_{FP}$ in short --- are actually taken as the smallest constant such that the bound holds.
It is not directly clear how $C_{FP}$ depends on the shape of the domain. For some domain types explicit
upper bounds can be given. All \emph{convex} domains $\O$  satisfy $C_{P}(\O) \leq \frac{1}{\pi}$.
Constant bounds for Friedrichs' inequality in terms of geometry can be found in Literature,
e.g. \cite{zheng2005friedrichs, veeser2011poincare}. In particular, for stars $\w_a$  --- the elements touching
a vertex $a$ in some triangulation --- a constant bound can be given that only depends on the shape regularity of 
the associated triangulation: $C_{F}(\w_a) \leq C(\sup_{K \in \T} h_K / p_K)$.

\section{Raviart-Thomas elements}


Raviart-Thomas functions can be (partially) determined by their divergence and surface normal, as formalised by the folowing lemma.
\begin{lem}
  \label{lem:rtexists}
  Consider an element $K$ with Raviart-Thomas space $\RT_p(K)$.
  There exists a function $\vsig \in \RT_p(K)$ that  solves
  \begin{equation}
    \label{eq:rtexists}
  \begin{aligned}
    \div\vsig &= p_K\\
    \vsig \cdot n &= p_e,
  \end{aligned}
\end{equation}
  for polynomials $p_K \in \P_p(K)$ and $p_{e} \in \P_p(\partial K)$ satisfying
  the compatibility constraint
  \[
    \int_K p_K = \int_{\partial K} p_e.
  \]
\end{lem}
\begin{proof}
  Counting degrees of freedom shows that the system \eqref{eq:rtexists} is overdetermined by maximally one degree of freedom. Imposing
  the compatibility constraint takes away this extra freedom. Existence of a solution then follows from showing that the system is consistent.

  Another --- perhaps more elegant --- derivation is as follows. Consider the Neumann problem
  \[
    \Delta u = p_K \quad \text{in } K, \quad \nabla u\cdot n = p_e \quad \text{in } \partial K,
  \]
  or in weak formulation,
  \[
    \int_K \nabla u \cdot \nabla v = \int_{\partial K} v p_e - \int_K vp_K \quad \forall v \in H^1(K).
  \]
  This has a unique solution $u$ under the assumption that the right hand side has mean zero, which is exactly the compatibility constraint.
  Notice that $\nabla u$ satisfies the wanted properties \eqref{eq:rtexists}:
  \[
    \div \nabla u = \Delta = p_K, \quad \nabla u \cdot n = p_e.
  \]
  We then simply set $\vsig = \Pi^{RT} u$, for $\Pi^{RT}$ the interpolation projector on the space $\RT_p(K)$,
  formally defined in \cite[\S3.4]{brezzimixed}. From the properties of this projector \cite[Lem~3.7]{brezzimixed} we
  deduce that $\vsig$ is a function in $\RT_p(K)$ that satisfies \eqref{eq:rtexists}.
\end{proof}

\section{Auxiliary results}
\label{sec:aux}
Here the proof of Lemma \ref{lem:clasequivlow} is included for self-containedness.
  \begin{lemstar}
  The estimator $\uanorm{\v{\sigma_a}}_{\w_a}$ is also bounded from below by the classical estimator,
  \[
    \uanorm{\vsiga}_{\w_a} \gtrsim h_{\w_a}\norm{f + \Delta U}_{\w_a} + h^{1/2}_{\w_a} \norm{\llbracket \nabla U  \rrbracket}_{\gamma_a},
  \]
  for a constant depending on the shape regularity and the polynomial-degree $p$ used in~$\VV$.
  \end{lemstar}
\begin{proof}
  A lower bound for $\uanorm{\vsiga}_{\w_a}$ can be given in terms or $r_a$.
  By the definition of the local equilibrium property~\eqref{eq:defsigma} we have $\ip{r_a, v} = - \ip{\v{\sigma_a}, \nabla v}_{\w_a}$ for $v \in H_\star^1(\w_a)$,
  and thus
  \[
    \norm{r_a}_{H_\star^1(\omega_a)'} = \sup_{\{v \in H_\star^1(\w_a) : \norm{\nabla v}_{\w_a} = 1\}} {\ip{r_a, v}} = \sup_{\{v \in H_\star^1(\w_a) : \norm{\nabla v}_{\w_a}=1\}} {\ip{\v{\sigma_a}, \nabla v}_{\w_a}} \leq \uanorm{\v{\sigma_a}}_{\w_a}.
  \]

  Fix an interior vertex  $a \in \V^{int}$. Applying the above and using orthogonality on constant functions reveals  
  \begin{align*}
    \uanorm{\v{\sigma_a}}_{\w_a} &\geq \norm{r_a}_{H^1_\star(\w_a)'} \\
    &= \sup_{\{v \in H^1_\star(\w_a) : v \ne 0\}} { \ip{r_a, v} \over \norm{\nabla v}_{\w_a}} \\
    &= \sup_{\{v \in H^1(\w_a) : v \not \equiv C_{st}\}} { \ip{r_a, v - v_{\w_a}} \over \norm{\nabla (v - v_{\w_a})}_{\w_a}} \\
    &= \sup_{\{v \in H^1(\w_a) : v \not \equiv C_{st}\}} { \ip{r_a, v} \over \norm{\nabla v}_{\w_a}}.
  \end{align*}
  Similar to the previous proof we decompose $r_a$ in triangle and edge terms
  \[
    \ip{r_a,v} = \sum_{K\subset \w_a} \langle{r^T_a, v}\rangle_K + \sum_{e \subset \gamma_a} \ip{r^e_a, v}_e,
    = \sum_{K\subset \w_a} \langle{\psi_a \left [f+\Delta U \right], v}\rangle_K + \sum_{e \subset \gamma_a} \ip{\psi_a \llbracket \nabla U \rrbracket, v}_e.
  \]
  The triangle and edge functions $(r^T_a, r^e_a)$ are actually broken polynomials from the spaces $\P_r^{-1}(\w_a)\times \P_r^{-1}(\gamma_a)$ for some degree $r$.

  Start with an element $K \subset \w_a$, and consider the space $H_0^1(K)$. Since a function $v \in H_0^1(K)$ vanishes on the boundary, 
  we can naturally extend it to a function $v \in H^1(\w_a)$ by setting $v \equiv 0$ on $\w_a \setminus K$. As $v$
  vanishes all edges $e \subset \gamma_a$ and all triangles except $K$, we have $\ip{r_a, v} = \langle{r_a^T, v}\rangle_K$ and thus
  \[
    \sup_{\{v \in H^1(\w_a) : v \not \equiv C_{st}\}} { \ip{r_a, v} \over \norm{\nabla v}_{\w_a}} 
    \geq \sup_{\{v \in H_0^1(K) : v \ne 0\}} { \langle{r^T_a, v}\rangle_K \over \norm{\nabla v}_{K}}
    = \sup_{\{v \in H_0^1(K) :  \norm{\nabla v}=1\}} { \langle\psi_a\left[ f + \Delta U\right],v\rangle_K }
  \]
  An estimate for this last term can be found using equivalence of norms (cf. \cite[Ex~9.x.5]{brenner}):
  
  That is, we claim that $\norm{q} := \sup_{\{v \in H_0^1(K) : \norm{\nabla v}=1\}} \langle \psi_a q, v\rangle_K$
  defines a norm for polynomials $q$ of degree $r$ on $K$. The only troublesome axiom to be satisfied
  is that $\norm{q} = 0$  implies $q=0$. Suppose that $q$ is a non-zero polynomial.
  Then there exists a point $x \in \inte(K)$ for which we have $q(x) \ne 0$, and without loss of generality we suppose that $q(x) > 0$.
  Since $q$ is continuous we have $q(y) > q(x) / 2 > 0$ for $y$ in $V(x)$, a small neighborhood of $x$.
  The hat function satisfies $\psi_a > 0$ in the interior of $K$. Taking $V(x)$ small enough therefore guarantees that $\psi_a q>0$ on $V(x)$.
  Finally, choose a continuous $v \in H_0^1(K)$  with $v(x) = 1$ and $v = 0$ outside of $V(x)$. For this $v$ we then find
  \[
    \norm{q} \geq { \langle  \psi_a q, v\rangle_K \over \norm{\nabla v}_K}  = {\langle \psi_a q, v\rangle_{V(x)} \over \norm{\nabla v}_K} > 0,
  \]
  because $\psi_a q \ne 0 \ne v$ in a small non-empty open neighborhood of $x$. Hence $\norm{\cdot}$ is a norm on the space of polynomials with degree $r$.
  
  The above also holds for a reference element $\hat K$. Using equivalence of norms, and the usual transformation lemma
  shows that for all elements $K \subset \w_a$ we have
  \[
    \sup_{\{v \in H_0^1(K) :  \norm{\nabla v}=1\}} { \langle\psi_a\left[ f + \Delta U\right],v\rangle_K } \gtrsim h_K \norm{ f + \Delta U }_K,
  \]
  for a constant depending on the shape regularity and (unfortunately) on the polynomial-degree $p$ used in $\VV$.  This
  provides a lower bound for the element related terms.
  
  
  Next, fix an edge $e \subset \gamma_a$ and denote $K_1, K_2$ for the triangles that share this edge. Consider the following space of functions:
  \[
  V_e := \set{v \in H_0^1(K_1 \cup K_2) : \ip{v,P}_{K_1}= \ip{v,P}_{K_2} = 0\quad \forall P \in \P_r}.\]
  Again we can naturally extend these functions to live in $H^1(\w_a)$. Since $v \in V_e$ vanish on all edges except $e$, and
  is orthogonal to polynomials of degree $r$ on $K_1$ and $K_2$, we have
  \[
    \sup_{\{v \in H^1(\w_a) : v \not \equiv C_{st}\}} { \ip{r_a, v} \over \norm{\nabla v}_{\w_a}} 
    \geq \sup_{\{v \in V_e: v \ne 0\}} { \langle r^e_a, v \rangle_e \over \norm{\nabla v}_{K_1 \cup K_2}}
    = \sup_{\{v \in V_e: v \ne 0\}} { \langle \psi_a \llbracket \nabla U \rrbracket, v \rangle_e \over \norm{\nabla v}_{K_1 \cup K_2}}
          \gtrsim h_e^{1/2} \norm{\llbracket \nabla U \rrbracket}_e.
  \]
  The last inequality is a variation of a well known result \cite[Ex~9.x.7]{brenner}. It can be proven by using the same
  technique we used for the element terms. The constant depends on both shape regularity and the polynomial-degree $p$.

  The last step consist of summing the above inequalities over $K \subset \w_a$ and $e \subset \gamma_a$, i.e.
  \begin{align*}
    \uanorm{\vsiga}_{\w_a} \geq \norm{r_a}_{H_\star^1(\w_a)'} &\gtrsim \sum_{K \subset \w_a} h_K \|f + \Delta U\|_K + \sum_{e \subset \gamma_a} h_e^{1/2} \norm{\llbracket \nabla U \rrbracket}_e\\
    &\geq \left(\min_{K \subset \w_a} h_K\right) \sum_{K \subset w_a} \norm{f + \Delta U}_K +  \left(\min_{e \subset \gamma_a}h_e^{1/2} \right) \sum_{e \subset \gamma_a} \norm{\llbracket \nabla U \rrbracket}_e\\
    &\gtrsim h_{\w_a} \norm{ f + \Delta U}_{\w_a} + h_{\w_a}^{1/2} \norm{ \llbracket \nabla U \rrbracket}_{\gamma_a}.
  \end{align*}
  The hidden constants depend the shape regularity and the polynomial-degree $p$. The proof is similar for $a \in \V^{bdr}$.
\end{proof}
\end{document}
