\documentclass[thesis.tex]{subfiles}
\begin{document}
In the previous chapter the classical (or standard) residual error estimator is introduced as the driver for the
adaptive finite element method (AFEM). This results in an optimal algorithm, i.e. the adaptive meshes
generated by this method provide the highest possible convergence rate. U_hnfortunately, this asymptotic result
is a bit unfortunate for practical purposes due to the unknown constants. In practice
one would like to know \emph{when} to stop iterating, i.e. when the approximation error is small enough --- say
below some threshold. 

Another problem of the classical residual error estimator is that it is not polynomial-degree-robust; 
the constants depend on $p$, the degree of the finite element solutions. This becomes an issue
when considering \emph{hp}-AFEM, a version of AFEM where the polynomial degree can also vary
per simplex. For \emph{hp}-AFEM one would like an estimator with a constant independent of the polynomial degree
used in the finite element space.

Luckily, there are error estimators that suffer less from these constants problems. Recently quite
some research interesting has been shown for such constant-free estimators. One of these estimators is the
\emph{equilibrated residual error estimator} also called the \emph{Braess-Sch\"oberl estimator} \cite{braessequil, braessequilrobust,ernequil}.
In this chapter we will introduce this estimator, show that it is polynomial-robust, and prove that AFEM driven 
by this estimator is optimal.

For simplicity we (again) restrict ourself to the Poisson problem \eqref{eq:poisson} on a two-dimensional domain $\O \subset \R^2$,
and for some conforming triangulation $\T$ we consider $\VV(T)$: the (Lagrange) finite element space of degree $p$.\todo{p of p+1}
The Galerkin approximation (discrete solution) is denoted by $U_h \in \VV(T)$. If we do not need to stress dependence on the triangulation, we 
also write $V$ for this finite element space.
\section{Prager and Synge}
The so-called equilibrated residual estimators are based on the fundamental theorem of Prager and Synge \cite{prager}. 
For this we require the space $H(div; \Omega)$ as defined in \S \ref{sec:def}.  
\begin{thm}[Prager and Synge]
  Let $u \in H_0^1(\O)$ be the exact solution of the Poisson problem. 
  For a {flux} $\sigma \in H(\div; \O)$ satisfying the  equilibrium condition $\div \sigma + f = 0$ pointwise\todo{Is distributional also correct?},
it holds
\[
  \norm{\nabla u - \nabla v}^2_{L^2(\O)} + \norm{\nabla u - \sigma}^2_{L^2(\O)} = \norm{\nabla v - \sigma}^2_{L^2(\O)} \quad \forall v \in H_0^1(\O).
\]
\end{thm}
\begin{proof}
   Applying the divergence theorem yields,
  \begin{align*}
    &\int_\O  \left(\nabla u - \sigma\right) \cdot \nabla \left( u -  v\right)  \, \dif x \\ 
    =  - &\int_\O \left(u - v\right) \,  \nabla \cdot \left (\nabla u - \sigma\right) \dif x + \int_{\partial \O} (u - v) \left( \nabla u \cdot n - \sigma \cdot n\right) \dif s  = 0
  \end{align*}
  since from the assumptions  $\Delta u = -f = \div \sigma$ in $\O$, and $u - v = 0$ on $\partial \O$.
  From this orthogonality relation and Pythagoras we may conclude that
  \[
    \norm{\nabla(u-v)}^2_{L^2(\O)} + \norm{\nabla u - \sigma}^2_{L^2(\O)} = \norm{ -\nabla (u-v) + \nabla u - \sigma}^2_{L^2(\O)},
  \]
  which equals the asserted.
\end{proof}
For a flux $\sigma$ satisfying the equilibrium condition we obtain an 
\emph{reliable} constant-free estimator by replacing $v$ with the discrete solution $U_h$ in the previous theorem,
\begin{equation}
  \label{eq:synest}
  \enorm{u - U_h}_{\O}^2 =  \norm{\nabla u - \nabla U_h }^2_{L^2(\O)} \leq \norm{\nabla U_h - \sigma}^2_{L^2(\O)}.
\end{equation}

The question now arises how to construct $\sigma$. In general, one would like an estimator to be proportional the error.
For this, we need the estimator to also be \emph{efficient}: it should provide a lower bound for $\enorm{u -U_h}_{\O}$, up to
a constant and an oscillation term.

Suppose that $f$ is constant on each element in $\mathcal{T}$, then suitable fluxes $\sigma$ can be 
found in the lowest order Raviart-Thomas space $\RT_1$ --- properly defined in \S \todo{Introduce, alongside some
basic characterisations?}.
From \eqref{eq:synest} we then see that 
the best estimator in the Raviart-Thomas space is found by minimizing $\norm{\nabla U_h - \sigma}$ for all 
fluxes $\sigma \in \RT_1$ that are in equilibrium. However,
this global minimization procedure --- equivalent to the mixed finite element solution --- 
is too expensive for computation of an error estimate.

To overcome this problem, Braess and Sch\"oberl \cite{braessequil} propose minimizing local problems instead; a procedure
called \emph{equilibration}.

\section{Equilibration} 
Instead of constructing $\sigma$ the difference $\sigma^\triangle := \sigma - \nabla U_h$ is considered.
For the moment we let  $f$ be piece-wise constant, this avoids the effect of data oscillation.
The flux $\sigma$ will be constructed as an element in the 
Raviart-Thomas space $\RT_p$. The correction $\sigma^\triangle$ will therefore belong to the broken Raviart-Thomas space $\RT_p^{-1}$.

\todo{Move or delete paragraph}
\textcolor{red}{
The conditions on $\sigma$ being an equilibrated flux can now be rewritten (in distributional sense) as (TODO $H(\div)$ characterization) 
\begin{alignat*}{2}
  \sigma^\triangle + \nabla U_h \in H(\div; \O) &\iff \llbracket \sigma^\triangle  \rrbracket = -\llbracket \nabla U_h \rrbracket & \quad \text{ on F},\\
  \nabla \cdot ( \sigma^\triangle + \nabla U_h) = -f &\iff \nabla \cdot \sigma^\triangle = -f & \quad \text{ in T}.
\end{alignat*}
The latter follows as $U_h$ is the linear Lagrange solution, so  $\Delta U_h = 0$. Using a partition of unity, this can be turned into
a set of conditions for smaller patches.
}

Following \cite{ernequil} we denote $\V_h, \E_h$ for the set of vertices and edges in the triangulation~$\T_h$. 
For each vertex $a \in \V_h$ we write $\psi_a$ for the hat function at vertex $a$, 
i.e. the unique function in the linear finite element space
that takes value $1$ at $a$ and vanishes at the other vertices.
These hat functions provide us with a partition of unity: $\sum_{a \in \V_h} \psi_a \equiv 1$.
The local problems are solved on patches $\omega_a$, the star at a vertex $a \in \V_h$, alsp the support of
the hat function $\psi_a$:
\[
  \omega_a := \omega\left( \T_h, a\right) = \supp{\psi_a} = \bigcup \set{ K \in \T_h : a \in \partial T}.
\]
The total flux $\sigma^\triangle$ is the sum of local fluxes over these patches, 
\[
  \sigma^\triangle := \sum_{a \in \V_h} \sigma_{\omega_a}.
\]

As proposed  in \cite{braessequilrobust}, local fluxes $\sigma_a$ are found by decomposing the residual using the partition of unity. 
More precisely, let $v \in H_0^1(\O)$ and let $r$ be the usual residual, \todo{Define usual residual}
\[
  \ip{r,v} := a(u - U_h, v) = \ip{\nabla \left(u - U_h\right), \nabla v}_\O = \ip{f,v}_\O - \ip{\nabla U_h,\nabla v}_\O
\]
and for $a \in V_h$ the local residual is given by
\[
  \ip{r_a,v} := \ip{r,\psi_a v} = a(u - U_h, \psi_a v)_{\w_a}.
\]
The local flux $\sigma_a$ is taken from the broken Raviart-Thomas space $\RT_{p}^{-1}(\omega_a)$ such that
$\div \sigma_a = r_a$ holds in distributional sense; recall that we consider the \emph{weak} derivative, so that the latter
statement is equivalent to
\[
  \ip{r_a, v} = \ip{\div \sigma_a, v}_{\w_a} = - \ip{\sigma_a, \nabla v}_{\w_a} \quad \forall v \in H_\star^1(\w_a).
\]
\todo{Somehwere explicitely expand}
Both sides of this equation can be expressed in triangle and edge related terms by applying the divergence theorem.
After expanding, we see that the above holds if and only if
\begin{equation}
  \label{eq:sigmacons}
\begin{alignat*}{2}
  \div \sigma_a &= \psi_a \left (f + \Delta U_h\right) & \quad\text{in } T\in \T_h, T\subset \w_a\\
  \llbracket \sigma_a \rrbracket &= \psi_a \llbracket \nabla U_h \rrbracket & \quad \text{on } e \in \E^{int}_h, e \subset \text{int}(\w_a)\\
  \sigma_a \cdot n &= 0 &\quad \text{ on } \partial \w_a
\end{alignat*}
\end{equation}
 Since there is no unique solution, a solution $\sigma_a \in \RT_p^{-1}$
with \emph{minimal} $L_2$-norm is chosen. By construction and the partition of unity we have for $v \in H_0^1(\O)$,
\[
  \langle \div \sigma^\triangle,v\rangle_{\O} = \sum_{a \in \V_h} \ip{\div \sigma_a,v}_\O = \sum_{a \in \V_h} \ip{r_a,v} = \langle r, \sum_{a \in \V_h} \psi_a v \rangle  = \ip{r, v}.
\]
Set $\sigma = \nabla U_h + \sigma^\triangle$, then from Green's formulas we see that for $v \in V_h$
\begin{align*}
  \ip{\div \sigma, v}_\O &= \ip{\Delta U_h, v}_\O + \ip{r,v}\\
  &= \sum_{K \in \T_h} \ip{\Delta U_h,v}_K + \ip{f,v}_K - \ip{\nabla U_h, \nabla v}_K\\
  &= \sum_{K \in \T_h} \ip{\Delta U_h + f,v}_K + \ip{\Delta U_h, v}_K - \ip{\nabla U_h \cdot n, v}_{\partial K}\\
\end{align*}
\textcolor{blue}{
  This is complete non-sense, it does not equal $f$ in distribution, let alone pointwise.. How is this $\sigma$ eligible for Prager and Synge thm?  According to Braess et al. in \cite{braessequilrobust} it is. Annoying.
  Maybe the entire Prager and Synge theorem is not used? Does the flux satisfy the equilibrium condition states by Ern and Vohralik
  $\ip{\div \sigma, 1}_K = \ip{f, 1}_K$? Not sure.
}
\section{Residual bounds}
Using the relation between $\sigma_a$ and the local residual $r_a$ we can directly derive some useful bounds.
For an interior vertex $a \in \V_h^{int}$ the hat function $\psi_a$ belongs to the finite element space $V_h$. 
Using that $U_h$ is the Galerkin approximation therefore gives us
\[
  \ip{r_a, 1} = \ip{r, \psi_a} = \ip{f, \psi_a}_\O - \ip{\nabla U_h, \nabla \psi_a}_\O = 0.
\]
The functional $r_a$ vanishes on constants, and therefore we may interpret $r_a$
as a functional on the mean zero space. Again in the notation of \cite{ernequil}, we define for $a \in \V_h$,
\[
  H^1_\star(\omega_a) := \left[\begin{aligned}
      &\set{v \in H^1(\omega_a): \ip{v,1}_{\omega_a} = 0} &\quad a \in \V_h^{int},\\
    &\set{v \in H^1(\omega_a): v = 0 \text{ on }\partial \omega_a \cap \partial \O} &\quad a \in \V_h^{bdr}.
  \end{aligned}
\right.
\]
We employ this space with the norm $\norm{\nabla \cdot }_{\omega_a}$, which is a norm since \todo{ It is actually equivalent to the $H^1$-norm. Poincare-Friedrich}
\[
  \norm{\nabla v}_{\omega_a} = 0 \implies v = C_{st}, \quad v \in H_\star^1(\omega_a) \implies C_{st} = 0.
\]
With these definitions we are able to proof local efficiency and global reliability in terms of $r_a$.
\todo{Notation introduce}
\begin{lem}
  \label{lem:loceff}
  For every vertex $a \in \V_h$ we have the following local efficiency on $\omega_a$,
  \[
    \norm{r_a}_{H_\star^1(\omega_a)'} \lesssim \enorm{ u - U_h}_{\omega_a}
  \]
\end{lem}
\begin{proof}
  Per definitions and Cauchy-Schwarz we find
  \begin{align*}
    \norm{r_a}_{H_\star^1(\omega_a)^*} &= \sup_{v \in H_\star^1(\omega_a) : \norm{\nabla v}_{\w_a} = 1} \abs{\ip{r_a, v}}\\
    &= \sup_{v \in H_\star^1(\omega_a) : \norm{\nabla v}_{\w_a} = 1} \abs{\ip{\nabla \left(u - U_h\right), \nabla \left(\psi_a v\right)}_\O} \\
    &\leq \norm{\nabla u - \nabla U_h}_{\omega_a} \sup_{v \in H_\star^1(\omega_a): \|\nabla v\|_{\w_a}=1} \norm{ \nabla \left (\psi_a v \right)}_{\omega_a}.
  \end{align*}
  In order to get rid of the $\psi_a$-term we first apply the product rule,
  \begin{align*}
    \norm{\nabla\left(\psi_a v\right)}_{\omega_a} &= \norm{v \nabla \psi_a + \psi_a \nabla v}_{\omega_a}\\
    &\leq \norm{v \nabla \psi_a}_{\omega_a} + \norm{\psi_a \nabla v}_{\omega_a} \\
    &\leq \norm{v}_{\omega_a} \norm{\nabla \psi_a}_{\infty, \omega_a} + \norm{\psi_a}_{\infty, \omega_a} \norm{\nabla v}_{\omega_a},
  \end{align*}
  where the last inequality follows since $\omega_a$ is a bounded set. For the $\psi_a$ terms we note that 
  $\norm{\psi_a}_{\infty, \omega_a} = 1$, whereas the vector norm $\|\nabla \psi_a\|_{\w_a}$ is constant and bound on each triangle by
  $\rho_{T}$, and thus $\norm{\nabla \psi_a}_{\infty, \omega_a} \leq C h^{-1}_{\w_a}$ for some constant only depending on the 
  shape regularity of the triangulation and the maximum amount of triangles in a patch. 
  The $\norm{v}_{\omega_a}$-term can be estimated using Poincar\'e-Friedrich inequalities (cf \S\ref{sec:aux}):
  \[
    \norm{v}_{\omega_a} = \begin{cases}
      \norm{v - v_{\omega_a}} \leq C_{P,\omega_a} h_{\omega_a}\norm{\nabla v}_{\omega_a} & a \in \V_h^{int}, \\
      \norm{v} \leq C_{F,\omega_a} h_{\omega_a}\norm{\nabla v}_{\omega_a} & a \in V_h^{bdr},
    \end{cases}
  \]
  with the constants depending only on the shape regularity of the triangulation \todo{Reference??}.
  Combining these inequalities and using that $\norm{\nabla v}_{\w_a} = 1$ we arrive at the asserted.
\end{proof}
\begin{lem}
  \label{lem:globrel}
  A global error bound in terms of $r_a$ is given by,
  \[
    \enorm{u - U_h}_{\O}^2 \lesssim \sum_{a \in \V_h} \norm{r_a}^2_{H^1_\star(\omega_a)'}
  \]
\end{lem}
\begin{proof}
  Since $H_0^1(\O)$ is a Hilbert space with respect to $a(\cdot, \cdot) = \ip{\nabla \cdot, \nabla \cdot}_\O$, we find by
  duality
  \[
    \enorm{u - U_h}_{\O}^2 = \norm{\nabla u - \nabla U_h}_{\O}^2 = \sup_{ v \in H_0^1(\O): v \ne 0 } {\abs{a(u - U_h, v)} \over \norm{\nabla v}_{\O}} = \sup_{ v \in H_0^1(\O): v \ne 0 } {\abs{\ip{r, v}} \over \norm{\nabla v}_{\O} }.
  \]
  Rewriting the residual operator using the partition of unity yields for $v \in H_0^1(\O)$,
  \begin{align*}
    \ip{r , v} &= \sum_{a \in \V_h^{int}} \ip{r_a, v} + \sum_{a \in \V_h^{bdr}} \ip{r_a,v} \\
    &= \sum_{a \in \V_h^{int}} \ip{r_a, {v - v_{\omega_a}}} + \sum_{a \in \V_h^{bdr}} \ip{r_a, v} \\
    &\leq \sum_{a \in \V_h^{int}} \norm{r_a}_{H^1_\star(\omega_a)'} \norm{\nabla (v- v_{\omega_a})}_{\omega_a} 
    + \sum_{a \in \V_h^{bdr}} \norm{r_a}_{H^1_\star(\omega_a)'} \norm{\nabla v}_{\omega_a}\\
    &\lesssim \norm{\nabla v}_\O \sum_{a \in \V_h} \norm{r_a}_{H^1_\star(\omega_a)'}.
      \end{align*}
  The second equality follows from orthogonality on the constants; the first inequality follows since $v - v_{\omega_a}, v \in H_\star^1(\w_a)$ 
  for an interior resp. boundary vertex $a$; and $\sum_{a \in \V_h} \norm{\nabla v}_{\omega_a} \lesssim \norm{\nabla v}_{\O}$ since
  every triangle is contained in a uniformly bounded number of triangles. Combining these inequalities yields,
  \begin{align*}
    \norm{\nabla u - \nabla U_h}_{\O}^2 \leq \norm{\nabla v}_{\O}^{-1} \ip{r,v} \lesssim \norm{\nabla v}_{\O}^{-1} \norm{\nabla v}_{\O} \sum_{a \in \V_h} \norm{r_a}_{H^1_\star(\omega_a)'}.
  \end{align*}
  \textcolor{blue}{In particular, the constant can be (cheaply) calculated  given the initial triangulation. Are
  there tighter bounds, preferably without constants?}
\end{proof}

These bounds become relevant if they can be related to the (local) estimators~$\sigma_a$. Recall
that $\sigma_a$ was found as minimal norm function from $\RT_p^{-1}(\omega_a)$ satisfying $\div~\sigma_a~=~r_a$.
The fact that this vector function is of minimum norm enabled Braess et al. \cite{braessequilrobust} to prove the following powerful theorem.
\begin{thm}
  \label{thm:locresequiv}
  For $a \in \V_h$, let $\sigma_a$ be found as described above. 
  We have the bound $\norm{r_a}_{H_\star^1(\omega_a)'} \leq \norm{\sigma_a}_{\omega_a}$. But more importantly,
  a reverse bound also holds for a constant {not} depending on the polynomial-degree $p$ used in the finite element space $\VV(\T)$:
  \[
    \norm{\sigma_a}_{\omega_a} \lesssim \norm{r_a}_{H_\star^1(\omega)'}.
  \]
\end{thm}
\begin{proof}
  From the characteristic definition of $\sigma_a$ we see that for $v \in H_\star^1(\w_a)$ we have
  \[
    \ip{r_a, v} = \ip{\div \sigma_a, v}_{\w_a} = - \ip{\sigma_a, \nabla v}_{\w_a}.
  \]
  Therefore we directly find the first inequality, i.e.
  \[
    \norm{r_a}_{H_\star^1(\omega_a)'} = \sup_{v \in H_\star^1(\w_a) : \norm{\nabla v}_{\w_a} = 1} \abs{\ip{r_a, v}} = \sup_{v \in H_\star^1(\w_a) : \norm{\nabla v}_{\w_a}=1} \abs{\ip{\sigma_a, \nabla v}_{\w_a}} \leq \norm{\sigma_a}_{\w_a}.
  \]

  The proof of the second inequality is more involved. A constructive proof is given in \cite[Theorem~7]{braessequilrobust}.
\end{proof}
\begin{rem}
  Combing the last Theorem and Lemma we see that $\sum_{a \in \V_h} \norm{\sigma_a}_{\w_a}$ provides a reliable estimator.
  When using this estimator in an adaptive finite element method, an obvious refine strategy would consist
  of refining those triangles for which $\norm{\sigma_a}_{\w_a}$ is large. Later we will show that this is indeed
  an optimal choice. To prove this, we will relate this estimator the classical residual estimator.
  \\
  \textcolor{blue}{
    This sounds like --- and is provable --- an optimal strategy up to constant. However, there is still some 
    overlap in the patches causing constants to arrise. In view of lower constants, a more preferable method probably consists of refining the triangles $T$
    where $\|{\sigma^\triangle|_{T}}\| =\|{\sum_{a} \sigma_a|_{T}}\|$ is large. This should cancel the overlap in patches,
    and therefore provide better constants.
  }
\end{rem}
\begin{lem}
  \label{lem:starequiv}
  The estimator $\norm{\sigma_a}_{\w_a}$ is equivalent to the classical estimator for the patch~$\w_a$, 
  \todo{Notatie jump consistent maken}
  \[
    \norm{\sigma_a}^2_{\w_a} \approx h_{\w_a}^2\norm{f + \Delta U_h}^2_{\w_a} + h_{\w_a} \norm{\llbracket \nabla U_h \cdot n \rrbracket}_{\gamma_a}.
  \]
  Where $\gamma_a$ are the interior edges of $\w_a$, i.e. $\gamma_a = \E_h^{int} \cap \text{int}(\w_a)$.
\end{lem}
\begin{proof}
  We start with the $\gtrsim$ relation.  Applying the previous Theorem \ref{thm:locresequiv} and using orthogonality on constants yields  
  \begin{align*}
    \norm{\sigma_a}_{\w_a} &\geq \norm{r_a}_{H^1_\star(\w_a)'} \\
    &= \sup_{v \in H^1_\star(\w_a) : v \ne 0} { \ip{r_a, v} \over \norm{\nabla v}_{\w_a}} \\
    &= \sup_{v \in H^1(\w_a) : v \not \equiv C_{st}} { \ip{r_a, v - v_{\w_a}} \over \norm{\nabla (v - v_{\w_a})}_{\w_a}} \\
          &= \sup_{v \in H^1(\w_a) : v \not \equiv C_{st}} { \ip{r_a, v} \over \norm{\nabla v}_{\w_a}}.
  \end{align*}
  Recall \todo{refer} that we could decompose,
  \[
    \ip{r_a,v} = \sum_{T \subset \w_a} \langle{r^T_a, v}\rangle_T + \sum_{e \subset \gamma_a} \ip{r^e_a, v}_e,
  \]
  with 
  \[
   r^T_a = \psi_a \left[ f + \Delta U_h \right], \quad r^e_a = \psi_a \llbracket \nabla U_h \rrbracket.
  \]
  In particular $(r^T_a, r^e_a)$ are piecewise polynomials taken from the broken spaces $\P_r^{-1}(\w_a)\times \P_r^{-1}(\gamma_a)$.

  Fix a triangle $T \subset \w_a$, and consider the space $H_0^1(T)$. Since a function $v \in H_0^1(T)$ vanishes on the boundary, 
  we can naturally extend it to a function $v \in H^1(\w_a)$ by setting $v \equiv 0$ on $\w_a \setminus T$. Since $v$
  vanishes on every triangle except $T$, and all edges $e \subset \gamma_a$ we have have
  \[
         \sup_{v \in H^1(\w_a) : v \not \equiv C_{st}} { \ip{r_a, v} \over \norm{\nabla v}_{\w_a}} 
         \geq \sup_{v \in H_0^1(T) : v \ne 0} { \langle{r^T_a, v}\rangle_T \over \norm{\nabla v}_{T}}
         \gtrsim h_T \norm{r^T_a}_T.
  \]
  The last inequality is a known result, since $r^T_a$ is a polynomial resticted to $T$ (c.f. \cite[Ex~9.x.5]{brenner}).
  
  Next we fix an edge $e \subset \gamma_a$ and denote $T_1, T_2$ for the triangles that share this edge. Then we can consider the space
  \[
  V_e := \set{v \in H_0^1(T_1 \cup T_2) : \ip{v,P}_{T_1}= \ip{v,P}_{T_2} = 0\quad \forall P \in \P_r}.\]
  Again we can naturally extend these functions to live in $H^1(\w_a)$. Since $v \in V_e$ vanish on all edges except $e$, and
  it is orthogonal to polynomials of degree $r$ on $T_1, T_2$ we have
  \[
         \sup_{v \in H^1(\w_a) : v \not \equiv C_{st}} { \ip{r_a, v} \over \norm{\nabla v}_{\w_a}} 
         \geq \sup_{v \in V_e: v \ne 0} { \langle r^e_a, v \rangle_e \over \norm{\nabla v}_{T_1 \cup T_2}}
         \gtrsim h_e^{1/2} \norm{r^e_a}_e.
  \]
  The last inequality is again a known result, since $r^e_a$ is a polynomial resticted to $e$ (c.f. \cite[Ex~9.x.7]{brenner}).

  Finally, summing the inequalities over $T \subset \w_a$ and $e \subset \gamma_a$ yields
  \begin{align*}
    \norm{r_a}_{H_\star^1(\w_a)'} &\gtrsim \sum_{T \subset \w_a} h_T \|{r^T_a}\|_T + \sum_{e \subset \gamma_a} h_e^{1/2} \norm{r^e_a}_e\\
    &\gtrsim h_{\w_a} \norm{\psi_a \left[ f + \Delta U_h \right]}_{\w_a} + h_{\w_a}^{1/2} \norm{ \psi_a \llbracket \nabla U_h \rrbracket}_{\gamma_a}.
  \end{align*}
  With the last equality following since the number of triangles in a patch is uniformly bounded.
  \textcolor{blue}{How can we get rid of $\psi_a$? Away from the boundary $\partal \w_a$ we know that $\psi_a > \epsilon$ so we can bound this term. Not sure how to find a bound on boundary `strip'.}

  \textcolor{red}{
  Using an equivalence of norms we will show that for $P = (P^T, P^e) \in \P_r^{-1}(\w_a) \times \P_r^{-1}(\gamma)$ we have
    \[
      \|P\| := \sup_{v \in H^1(\w_a) : V\not \equiv C_{st}} \frac{\langle{v, P^T}\rangle_{\w_a} + \langle v, P^e \rangle}{\norm{\nabla v}_{\w_a}} \gtrsim
      \norm{p^T}_{\w_a} + \norm{p^e}_{\gamma_a}.
    \]
    First we need to show that $\norm{P}$ defines a norm: triangle inequality, scalar multiplication are easy.
    The next property is $\norm{P} = 0 \implies P^T = 0 \text{ and }  P^e = 0$. For this, consider a bubble function $B_a \in H^1(\w_a)$
    which vanishes on all edges  ($\E_h \cap \w_a$) and such that $B_a > 0$ on each triangle in $\w_a$. Then we have $B_a P^T \in H^1(\w_a)$, since $P^T$ is only problamatic on the edges (FORMALISE). And thus,
    \[
      \norm{P} = 0 \implies \langle{B_aP^T, P^T}\rangle_{\w_a} + \langle B_aP^T, P^e \rangle_{\gamma_a} = 0 \implies B_a(P^T)^2 = 0\quad\text{a.e.} \implies P^T = 0
    \]
    Similarly, we can pick (how?) a function $w \in H^1(\w_a)$ such that $w > 0$ on $\gamma_a$ and such that $w$ is orthogonal to $\P_r^{-1}(\w_a)$.
    We can (construction?) naturally extend $P^e$ (which is defined on the interior edges only) to a function $P^e$ which is defined for the whole of $\w_a$ such that $P^e \in H^1(\w_a)$. Then $wP^e \in H^1(\w_a)$ is a non-constant function for which we find
    \[
      \norm{P} = 0 \implies \langle{wP^e, P^T}\rangle_{\w_a} + \langle wP^e, P^e \rangle_{\gamma_a}  = 0 \implies w(P^e)^2 = 0 \quad\text{a.e on }\gamma_a \implies P^e = 0
    \]
    by orthogonality on $P^T$.
    The previous constructions also directly show that $\norm{P} \geq 0$, so that $\norm{P}$ is a norm on $\P_r^{-1}(\w_a)\times\P_r^{-1}(\gamma_a)$. By equivalence of norms on finite dimensional spaces we find the wanted. Unfortunately this constant
    still depends on $\w_a$. How can we overcome this problem? Replacing the inner products using substition?
    Given that the equivalence holds, we are directly done with the proof. We do not need the second part.
  }

  Now the other inequality $\lesssim$. Also from the previous theorem, we know that $\norm{\sigma_a}_{\w_a} \lesssim \norm{r_a}_{H_\star^1(\w_a)'}$. For $v \in H_\star^1(\w_a)$ we  rewrite the local residual $r_a$ in element and edge terms,
  \begin{align*}
    \ip{r_a, v} &= \sum_{T \subset \w_a} \ip{\psi_a\left(f + \Delta U_h\right), v}_{T} + \sum_{e \subset \gamma_a} \ip{\psi_a \llbracket \nabla U_h \rrbracket, v}_e\\
    &\leq \sum_{T \subset \w_a} \norm{\psi_a}_{\infty,T} \norm{f + \Delta U_h}_{T}\norm{v}_{T} + \sum_{e \subset \gamma_a} \norm{\psi_a}_{\infty, e} \norm{ \llbracket \nabla U_h \rrbracket}_e \norm{v}_e\\
    &\leq \sum_{T \subset \w_a} \norm{f + \Delta U_h}_{T}\norm{v}_{T} + \sum_{e \subset \gamma_a} \norm{ \llbracket \nabla U_h \rrbracket}_e \norm{v}_e
  \end{align*}
  For the element terms we may use Poincar\'e-Friedrichs inequality (c.f. \S\ref{sec:aux}) to find
  \[
    \sum_{T \subset \w_a} \norm{f + \Delta U_h}_{T} \norm{v}_T \leq \norm{v}_{\w_a} \sum_{T\subset \w_a} \norm{f + \Delta U_h}_{T} \leq 
    h_{\w_a} C_{FP} \norm{\nabla v}_{\w_a} \norm{f + \Delta U_h}_{\w_a}.
  \]
  \textcolor{blue}{Does not introduce the scaling term correctly? Does this constant depends on $\w_a$? Normally this would be avoided
  by considering a reference element, unfortunately I don't see how we can consider a reference \emph{patch}.
  Or does the constant only depend on the diameter and the shape regularity $k_{\T_h}$,
  i.e. $C_{FP}(\w_a) = \widetilde{C}_{FP}(k_{\T_h})\text{diam}(\w_a)$.}

  For the edge terms, let $e \subset \gamma_a$ be an edge with $T$ an adjoint triangle. Applying the trace theorem and transformation lemma
  gives us,
  \[ 
    \norm{v}_e \leq \norm{v}_{\partial T} \lesssim h_T^{-1/2} \norm{v}_T + h_T^{1/2} \norm{\nabla v}_T
  \]
  for a constant only depending on the shape regularity. 
  Summing over all edges then yields,
  \[
    \sum_{e \subset \gamma_a} \norm{v}_e \lesssim \sum_{e\subset \gamma_a}h_{T_e}^{-1/2} \norm{v}_{T_e} h_{T_e}^{1/2} \norm{\nabla v}_{T_e}
    \lesssim h_{\w_a}^{-1/2} \norm{v}_{\w_a} + h_{\w_a}^{1/2} \norm{\nabla v}_{\w_a} \lesssim h_{\w_a}^{1/2} \norm{\nabla v}_{\w_a}.
  \]
  Where we used that maximum number of triangles in a patch is universally bounded, and that $v \in H_\star^1(\w_a)$ to apply the Poincar\'e-
  Friedrichs inequality. For the edge terms we thus have
  \[
    \sum_{e \subset \gamma_a} \norm{ \llbracket \nabla U_h \rrbracket}_e \norm{v}_e \leq \norm{\llbracket \nabla U_h \rrbracket}_{\gamma_a} \sum_{e \subset \gamma_a} \norm{v}_e \lesssim h_{\w_a}^{1/2} \norm{\llbracket \nabla U_h \rrbracket}  \norm{\nabla v}_{T}. 
  \]

  Now combining the inequalities with the definition of the dual norm $\norm{r_a}_{H_\star^1(\w_a)'}$ yields the wanted upper bound.
\end{proof}

\section{Adaptive finite element method}
We will now consider an adaptive finite element method using the equilibrated residual estimator.
Recall that AFEM can be described by the following loop,
\[
  \texttt{SOLVE} \to \texttt{ESTIMATE} \to \texttt{MARK} \to \texttt{REFINE}.
\]
In words: calculate the Ritz-Galerkin solution for a triangulation, estimate
the error using an a posteriori error estimator, mark elements that need to be refined, refine
the triangulation, and start over. 

After specifying the details of these four modules, we will be able to show that the produced
sequence of Galerkin approximations converges with an optimal rate. This result
is similar to the one given before, except that we will now use a different estimator. This
result is a simplified form on the proof given by Casc\'on and Nochetto \cite{cascon2012}.
A different optimality proof is given by Kreuzer and Siebert \cite{ainsworthbernstein}; therein
slightly different versions of \texttt{MARK} and \texttt{REFINE} are used.

First we need some additional notation. Given a triangulation $\T$ we denote $\W$ for the set of patches, i.e.
\[
  \W := \set{\w_a : a \in \V}.
\]
Let $\sigma_a$ for $a \in \V$ be the equibilrated local fluxes. We notate the
  error estimator, oscilation and total error resp by:
\begin{align*}
  \eta^2_\T(U, \w_a) &:= \norm{\sigma_a}_{\w_a}, \\
  \osc^2_\T(f, \w_a) &:= \norm{h_\T\left(f - f_{\w_a}\right)}_{\w_a},\\
  \vartheta^2_\T(U, \w_a) &:= \eta^2_\T(U, \w_a) + \osc^2_\T(f, \w_a).
\end{align*}
The above quantities extend to sets of patches in the usual way 
--- just decompose it as a sum over patches in the set.
Often the $a$ is dropped from notation, but we stress that any patch $\w \in \W$ is implicitly
associated to a vertex $a \in \V$.


\paragraph{SOLVE}Given a triangulation $\T$, this method computes
\[
  U = \texttt{SOLVE}(\T),
\]
where $U \in \VV(\T)$ is the \emph{exact} Ritz-Galerkin solution. The effects of using an inexact solver have
been studied in the literature as well.
\paragraph{ESTIMATE} Given a triangulation $\T$ with vertices $\V$ and its discrete solution $U \in \VV(\T)$, this module 
calculates the total error indicators --- the equilibrated estimator plus the oscillation. We have
\[
  \set{\vartheta_\T(U, \w)}_{\w \in \W} = \texttt{ESTIMATE}(U, \T).
\]
\paragraph{MARK}
A \emph{D\"orfler Marking} strategy is used for marking. For a parameter $\theta \in (0,1]$ we calculate
\[
  \mathcal{M} = \texttt{MARK}( \set{\vartheta_\T(U, \w_a)}_{a \in \V}, \T),
\]
where $\mathcal{M} \subset \W$ is the minimal cardinality set that satisfies
\[
  \vartheta_\T(U,\mathcal{M}) \geq \theta \vartheta_\T(U, \W).
\]
\begin{rem}
  In contrast to the standard residual estimator, marking is done using the total error.
\end{rem}
\todo{Marking for total error really needed}
\paragraph{REFINE}
The \texttt{MARK} outputs a set $\mathcal{M}$ of \emph{patches} on which the error is proportionally large. 
Logically, all the triangles in this set should be refined. This module calculates a refinement
\[
  \T_\star = \texttt{REFINE}(\T, \mathcal{M}),
\]
where $T_\star$ is the smallest conforming refinement of $\T$ such that the triangles of
all patches in the marked set are bisected $b \geq 1$ times.
\todo{Rare refine method in cascon en nochetto}
\\\\
We can now formulate the AFEM algorithm and its produced sequences. For this we use the
iteration counter $k$ as a subscript to differentiate among the sets.
Let an initial triangulation $\T_0$ be given, set $k = 0$ and iterate the following steps
\begin{enumerate}
\item $U_k = \texttt{SOLVE}(\T_k)$;
\item $\set{\vartheta_k(U_k, \w)}_{\W_k} = \texttt{ESTIMATE}(U_k, \T_k)$;
  \item $\mathcal{M}_k = \texttt{MARK}(\set{\vartheta_k(U_k, \w)}_{\W_k}, \T_k)$;
  \item $\T_{k+1} = \texttt{REFINE}(\T_k, \mathcal{M}_k)$;
  \item $k  = k + 1$.
\end{enumerate}

In \cite[\S 4]{cascon2012} a list of assumptions are given under which the above AFEM algorithm produces
an optimal sequence of solutions $(U_k, \T_k)_{k \geq 0}$. First we will show that these
assumptions --- compare \cite[Assump~4.1]{cascon2012} --- hold using the equilibrium based estimator as described above.
In the standard case we had a refined set holding the triangles that were refined. In this case
we need to define something similar, but then in terms of refined patches:
\[
  \RR_{\T \to \T_\star} = \set{ \w \in \W : T \not \in \T_\star\quad \forall T \subset \w}.
\]
So $\RR_{\T \to \T_\star}$ consists of all patches in $\T$ that are \emph{entirely} refined.
\begin{lem}
  Let a triangulation $\T$ and a refinement $\T_\star \geq \T$ be given, with $U \in \VV(\T)$
  and $U_\star \in \VV(\T)$ the appropriate discrete solutions. There
  exists constants $C_1, C_2, C_3$ such that:
  \begin{enumerate}
    \item The estimator is reliable; the approximation error can be bound using the total error estimator:
      \[
        \enorm{ u - U}_{\O}^2 \leq C_1 \vartheta^2_\T(U, \W) = C_1 \left [ \eta^2_\T(U, \W) + \osc^2_\T(f, \W)\right]
      \]
    \item The estimator is efficient; the estimator provides a lower bound for the error (up to oscillation):
      \[
        C_2 \eta^2_\T(U, \W) \leq \enorm{u - U}_{\O} + \osc_\T^2(f, \W).
      \]
    \item The estimator provides a localized upper bound, i.e. the error between $U$ and $U_\star$ can be bound using the
      refined set $\RR_{\T \to \T_\star}$:
      \[
        \enorm{U_\star - U}^2_{\O}  \leq C_1\vartheta^2_\T(U, \RR_{\T \to \T_\star}).
      \]
    \item The estimator provides a local lower bound, i.e. the reverse of the above
      \[
        C_3 \eta^2_\T(U, \RR_{\T \to \T_\star}) \leq \enorm{U_\star - U}^2_{\O}  + \osc_\T^2(U, \RR).
      \]
      \textcolor{blue}{In \cite{cascon2012} this is actually defined in terms of $\RR^n$, this consists
        of all refined patches that satisfy the interior node property. In \cite{kreuzersiebert} this
        restriction is not made, in stead the estimator is locally compared to the residual estimator.
      The latter is what we do as well, so I don't think we need this additional interior node property.}
  \end{enumerate}
\end{lem}
\begin{proof}
  Reliability follows directly from Lemma \ref{lem:globrel} and Theorem \ref{thm:locresequiv},
  \todo{oscillation}
  \[
    \enorm{u - U}_{\O}^2 \lesssim \sum_{a \in \V_h} \norm{r_a}^2_{H_\star^1(\w_a)'} \leq \sum_{a \in \V_h} \norm{\sigma_a}_{\w_a} = \eta^2_\T(U, \W).
  \]

  \todo{Write down dependence constants}
  Efficiency follows similarly from Lemma \ref{lem:loceff} and Theorem \ref{thm:locresequiv},
  \[
    \enorm{u - U}_{\O}^2 \gtrsim \sum_{a \in \V_h} \enorm{u - U}_{\O}^2_{\w_a} \gtrsim \sum_{a \in \V_h} \norm{r_a}^2_{H_\star^1(\w_a)'} \gtrsim \sum_{a \in \V_h} \norm{\sigma_a}_{\w_a} = \eta^2_\T(U, \W),
  \]
  where the first inequality follows since each triangle is contained in a uniformly bounded number of patches.

  The localized upper bound follows from equivalence with the classical residual estimator. Denote $\hat \eta_\T$ for the
  classical residual estimator (cf Definition \ref{def:clasest}). 
  From Theorem \ref{thm:residual_erro} with $\hat \RR = \T \setminus \T_\star$ we have a localized upper bound for the standard 
  estimator $\hat \eta_\T$. We can then bound this in terms in patches, specifically
  \begin{align*}
    \norm{\nabla U_\star - \nabla U}^2 &\lesssim \hat\eta_\T(U, \hat\RR)^2 \\
    &=\sum_{T \in \hat \RR}h^2_T\norm{f +\Delta U}_T^2+h_T \norm{\llbracket \nabla U \cdot n \rrbracket}_{\partial T \setminus \partial \O} \\
    &\leq\sum_{T\in\hat\RR} \sum_{\w \in \W:T \subset \w}h_\w\norm{f + \Delta U}_\w + h_\w \norm{\llbracket \nabla U \cdot n\rrbracket}_{\E^{int} \cap \w}\\
    &= \sum_{w \in \widetilde\RR} h_\w\norm{f + \Delta U}_\w + h_\w \norm{\llbracket \nabla U \cdot n\rrbracket}_{\E^{int} \cap \w}\\
    &\lesssim \eta_\T^2(U, \widetilde \RR).
  \end{align*}
  Where the last inequality follows from the equivalence between the two estimators on patches, see Lemma \ref{lem:starequiv}.
  Note that we have established the required upper bound for a larger refined set
  \[
    \RR = \RR_{\T \to \T_\star} \subset \widetilde \RR = \set{\w \in \W : T \not \in \hat \RR \quad \exists T \subset \w}.
  \]
  In words, $\RR$ consists of those patches for which all the triangles have been refined, whereas $\widetilde \RR$ consists
  of all patches for which at least one triangle has been refined. However, we have
  \[
    \# \RR \approx \#\widetilde \RR,
  \]
  so this does not become a problem later on.
  \textcolor{blue}{Since we only consider one type of estimator, maybe its wise to just consider $\widetilde \RR$ as the
  definition of the refined set. Moreover, the diameters in the above equivalence have to be shown correctly.}

  Similarly we can find the discrete local lower bound from the residual estimator
  \begin{align*}
    \norm{\nabla U_\star - \nabla U}^2 + \osc_\T^2(f, \RR) &\gtrsim \norm{\nabla U_\star - \nabla U}^2 + \hat\osc_\T^2(f, \hat \RR) \\
    &\gtrsim \hat \eta^2_\T(U, \hat \RR) \\
    &= \sum_{T \in \hat \RR} h_T^2\norm{f + \Delta U}_T^2 + h_T\norm{\llbracket U \cdot n \rrbracket}_{\partial T \setminus \partial \O}\\
    &\gtrsim \sum_{\w \in \widetilde \RR} h_\w^2\norm{f + \Delta U}_\w^2 + h_\w \norm{\llbracket U \cdot n \rrbracket}_{\E^{int} \cap \w}= \eta_\T^2(U, \RR).
  \end{align*}
  Here the last inequality follows from the fact that every triangle is contained in a uniformly bounded number of patches.
  \textcolor{blue}{Why first inequality? Again, redefine using $\widetilde \RR$? Dropped the interior node property}
\end{proof}
Furthermore the oscilation satisfies certrain properties given in the following lemma
\begin{lem}
  \label{lem:oscasum}

\end{lem}

We can now prove the so-called contraction property; this shows convergence of the AFEM method described above.
\begin{thm}
There exists a constant $0 < \alpha < 1$ depending on the shape regularity of $\T_0, b, \theta$ such that
\[
  \norm{\nabla u - \nabla U_{k+1}}^2 + \osc^2_{k+1}(U_{k+1}, \W_{k+1}) \leq \alpha^2\left(\norm{\nabla u - \nabla U_k}^2 + \osc^2_k(U_k, \W_k\right)
\]
\end{thm}


\section{Optimality AFEM}
We will show that convergence of AFEM driven by the equilibrated estimator is actually optimal.
Denote $\TT$ for the set of all conforming refinements of $\T_0$. 
We need the definition of an optimal approximation class, as given by
\[
  \A_s := \set{ }
\]
\textcolor{blue}{Note that we incorporate the oscilation into this class. Somehow stress dependency or something}.
The proof is now based on the same steps as for the standard case in \S\cite{sec:afem}.
\begin{lem}
  Let $\T \in \TT$ with discrete solution $U \in \VV(\T)$, then for any refinement $\T \leq \T_\star \in \TT$ with
  discrete solution $U_\star \in \VV(\T_\star)$ that satisfies
  \[
    \norm{\nabla u - \nabla U_\star}^2 + \osc^2_{\T_\star}(U_\star, \W_\star) \leq \mu (\norm{\nabla u - \nabla U}^2 + \osc_\T^2(U, \W)).
  \]
  Then the refined set $\R_{\T \to \T_\star}$ satisfies the D\"orfler property
  \[
    \vartheta_\T(U, \R) \geq \theta \vartheta_\T(U, \W).
  \]
\end{lem}


\begin{lem}
  Let $\set{\T_k, \VV_k, U_k}_{k \geq 0}$ be the sequence of results produced by AFEM. If $u \in \A_s$ then the minimal D\"orfler set $\M_k$ satisfies
  \[
    \M_k \leq ..
  \]
\end{lem}

\begin{lem}
  There exists a constant solely depending on $\T_0$ and $b$ such that
  \[
    \# \T_k - \# \T_0 \lesssim \sum_{i = 0}^{k-1} \# \M_i \quad k \geq 1.
  \]
\end{lem}

\begin{lem}
  Suppose that $u \in \A_s$, then there exists a constant depending on $b$ and $\T_0$, such that
  \[
    \norm{\nabla u - \nabla U_k} + \osc_k(U_k, \W_k) \lesssim ...
  \]
\end{lem}

 As before we  only consider the homogeneous
Poisson problem (see \eqref{eq:poisson}). A proof for the more general second-order elliptic partial differential equations can be
found in the literature, e.g. \cite{cascon2008, cascon2012}.
\section{Oscillation}
\textcolor{blue}{Should we introduce oscillation afterwards, or integrate it directly?}
Thus far we have assumed the right hand side $f$ to be a piecewise constant function. For more general $f \in L^2(\O)$ the
estimations have to be altered. The general approach is replace the exact $f$ with a polynomial approximation. Denote
$\Pi_{p}$ for the projector onto polynomials of degree $p$, and as shorthand write $f_p = \Pi_p f$ for the projection of $f$. 
Let $U$  solve the Galerkin approximation with right hand side $f_p$ instead of $f$. The introduction of this approximation
introduces data oscillation. Since $\sigma_a \in \RT_p^{-1}(\w_a)$, we are still able to find $\div \sigma_a = r_a$. 
The oscillation terms given as




\section{Ern and Volharik}
An equivalent equilibration method is introduced by Ern and Volharik in \cite{ernequil}. They
propose constructing the actual flux $\sigma$, instead of the difference $\sigma - \nabla U$. 
%This implies that $\sigma
%This time $\sigma$
An equivalent construction 
\section{Construction}




\section{Auxiliary results}
\label{sec:aux}
Let $a \in V_h^{int}$, for $v \in H_\star^1(\w_a)$ we have $v_{\w_a} = 0$, so by Poincare we have a constant $C_P(\w_a)$ such that 
\[
  \norm{v}_{\w_a} = \norm{v - v_T}_{\w_a} \leq C_P(\w_a) \norm{\nabla v}_{\w_a}
\]
For $a \in V_h^{bdr}$, then for $v \in H_\star^1(\w_a)$ we know that $v = 0$ on $\partial \O \cap \partial \w_a$, so by Friedrichs inequality
(cf Finite and Boundary Element Tearing and Interconnecting Solvers) we
have a constant $C_F(\w_a, \partial \O \cap \partial \w_a)$ such that
\[
  \norm{v}_{\w_a} \leq C_F(\w_a, \partial \O \cap \partial \w_a) \norm{\nabla v}_{\w_a} \quad \forall v \in H_\star^1(\w_a).
\]

\begin{proof}
\end{proof}


\end{document}
