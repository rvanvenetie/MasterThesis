\documentclass[thesis.tex]{subfiles}
\begin{document}
In the previous chapter the classical (or standard) residual error estimator is introduced as the driver for the
adaptive finite element method (AFEM). This results in an optimal algorithm, i.e. the adaptive meshes
generated by this method provide the highest possible convergence rate. Unfortunately, this asymptotic result
is a bit unfortunate for practical purposes due to the unknown constants. In practice
one would like to know \emph{when} to stop iterating, i.e. when the approximation error is small enough --- say
below some threshold. 

Another problem of the classical residual error estimator is that it is not polynomial-degree-robust; 
the constants depend on $p$, the degree of the finite element solutions. This becomes an issue
when considering \emph{hp}-AFEM, a version of AFEM where the polynomial degree can also vary
per simplex. For \emph{hp}-AFEM one would like an estimator with a constant independent of the polynomial degree
used in the finite element space.

Luckily, there are error estimators that suffer less from these constants problems. Recently quite
some research interest has been shown for such constant-free estimators. One of these estimators is the
\emph{equilibrated residual error estimator} also called the \emph{Braess-Sch\"oberl estimator} \cite{braessequil, braessequilrobust,ernequil}.
In this chapter we will introduce this estimator alongside some results, and prove that AFEM driven 
by this estimator is again optimal.

For simplicity we (again) restrict ourself to the Poisson problem \eqref{eq:poisson} on a two-dimensional domain $\O \subset \R^2$,
with a right hand side $f \in L^2(\O)$. For some conforming triangulation $\T_h$ of the domain we consider $\VV(\T_h)$: the (Lagrange) finite element space of degree $p\geq 1$. We assume each triangulation to be uniformly shape regular, i.e. $\sup_{K \in \T_h} h_K/p_K \leq \kappa_{h}$ for some constant $\kappa_{h}$.
The Galerkin approximation (discrete solution) is denoted by $U_h \in \VV(\T_h)$.
If we do not need to stress dependence on the triangulation, we also write $\VV$ for this finite element space with $U$ for the discrete
solution.
\section{Prager and Synge}
The so-called equilibrated residual estimators are based on the fundamental theorem of Prager and Synge \cite{prager}. 
For this we need the space $H(\div; \Omega)$ as defined in \ref{def:hdiv}. Vector-valued functions will be underlined
to emphasise this characteristic.
\begin{thm}[Prager and Synge]
  \label{thm:prager}
  Let $u \in H_0^1(\O)$ be the exact solution of the Poisson problem with a right hand side $f \in L^2(\O)$. 
  For a flux $\v{\sigma} \in H(\div; \O)$ satisfying the equilibrium condition $\div \v{\sigma} + f = 0$ in $L^2$-sense, it holds
\[
  \norm{\nabla u - \nabla v}^2_{\O} + \norm{\nabla u - \v{\sigma}}^2_{\O} = \norm{\nabla v - \v{\sigma}}^2_{\O} \quad \forall v \in H_0^1(\O).
\]
\end{thm}
\begin{proof}
  Applying the divergence theorem \eqref{thm:divergence} yields,
  \begin{align*}
    &\int_\O  \left(\nabla u - \v{\sigma}\right) \cdot \nabla \left( u -  v\right)  \, \dif x \\ 
    =  - &\int_\O \left(u - v\right) \,  \div \left (\nabla u - \v{\sigma}\right) \dif x + \int_{\partial \O} (u - v) \left( \nabla u \cdot n - \v{\sigma} \cdot n\right) \dif s  = 0
  \end{align*}
  since from the assumptions  $\Delta u = -f = \div \v{\sigma}$ in $\O$, and $u - v = 0$ on $\partial \O$.
  From this orthogonality relation and Pythagoras' identity we may conclude that
  \[
    \norm{\nabla(u-v)}^2_{\O} + \norm{\nabla u - \v{\sigma}}^2_{\O} = \norm{ -\nabla (u-v) + \nabla u - \v{\sigma}}^2_{\O},
  \]
  which equals the asserted.
\end{proof}
For a flux $\vsig$ satisfying the equilibrium condition we obtain an 
\emph{reliable} constant-free estimator by replacing $v$ with the discrete solution $U_h$ in the previous theorem,
\begin{equation}
  \label{eq:synest}
  \enorm{u - U_h}_{\O}^2 =  \norm{\nabla u - \nabla U_h }^2_{\O} \leq \norm{\nabla U_h - \vsig}^2_{\O}.
\end{equation}

The question now arises how to construct $\vsig$. In general one would like an estimator to be proportional to the error.
For this, we need the estimator to also be \emph{efficient}: it should provide a lower bound for $\enorm{u -U_h}_{\O}$, up to
a constant and possibly an oscillation term.

Suppose that $f$ is a piecewise polynomial of degree at most $p-1$ on $\T_h$, i.e.
\[
  f\in \P_{p-1}^{-1}(\T_h) := \set{ f \in L^2(\O): f|_K \in \P_{p-1}(K) \quad \forall K \in \T_h},
\]
and consider the  $p$-th order Raviart-Thomas space $\RT_p(\T_h)$ defined by:
\begin{align*} 
  \RT_p(K)    &:= \left[ \P_p(K)\right]^2 + \P_p(K)\v{x}, \\
  \RT_p^{-1}(\T_h) &= \set{ \v{\sigma} \in \left[L^2(\O)\right]^2 : \v{\sigma}|_K \in \RT_p(K) \quad \forall K \in \T_h},\\
  \RT_p(\T_h) &:= H(\div; \O) \cap \RT_p^{-1}(\T_h).
\end{align*}
Evidently fluxes $\v{\sigma} \in \RT_p(K)$ satisfy $\div \v{\sigma} \in \P_p^{-1}(\T_h)$; the divergence mapping
is even surjective \cite[Prop~2.3.3]{brezzimixed}, and thus $\RT_p(K)$ provides equilibrated fluxes.
 From~\eqref{eq:synest} we then see that 
the sharpest estimator in the Raviart-Thomas space is found by minimizing $\norm{\nabla U_h - \sigma}$ over all 
fluxes $\v{\sigma} \in \RT_p(\T_h)$ that are in equilibrium. However,
this global minimization procedure --- equivalent to the mixed finite element solution \cite{braess2007finite} --- 
is too expensive for computation of an error estimate.

To overcome this problem, Braess and Sch\"oberl \cite{braessequil} propose minimizing local problems instead; a procedure
called \emph{equilibration}.

\section{Equilibration} 
Instead of constructing flux $\v{\sigma}$ the difference $\v{\sigma}^\triangle := \nabla U_h -  \v{\sigma}$ is considered.
Again we assume that $f$ belongs to the broken polynomial space $\P_{p-1}^{-1}(\T_h)$, as this avoids the effect of
data oscillation.  The flux $\v{\sigma}$ will be constructed in $\RT_p(\T_h)$; the correction $\vsig^\triangle$ must 
therefore belong to the broken Raviart-Thomas space $\RT_p^{-1}(\T_h)$.

Following \cite{ernequil}, we use $\V_h, \E_h$ to denote  the set of vertices and edges in the triangulation~$\T_h$. 
Superscripts  $^{int}$ or $^{bdr}$ are added to indicate restrictions of $\V_h, \E_h$ to the interior or the boundary.
For each vertex $a \in \V_h$, we write $\psi_a$ for the hat function at vertex $a$, 
i.e. the unique function in the linear finite element space
that takes value $1$ at $a$ and vanishes at the other vertices.
These hat functions form a partition of unity: $\sum_{a \in \V_h} \psi_a \equiv \1$.
The local problems are solved on patches $\omega_a$, given by the star at a vertex $a \in \V_h$, also being the support of
the hat function $\psi_a$. We denote $\gamma_a$ for the union of interior edges touching $a$. 
So for a vertex $a \in \V_h$ we have
\[
  \omega_a       := \omega\left(\T_h, a\right) = \bigcup \{ K \in \T_h : a \in \partial K\},
  \quad \gamma_a := \gamma\left(\T_h, a\right) = \bigcup \{ e \in \E_h^{int} : a \in e\}.
\]
Often we are interested in the set of elements $K \in \T_h$ that make up $\w_a$. We abuse the notation and intuitively write $K \subset \w_a$
under sums as shorthand for $\set{K \in \T_h : a \in \partial K}$; similarly we write $e \subset \gamma_a$ to mean $\{e \in \E_h^{int} : a\in e\}$.
  
For each patch $\w_a$ we solve a local problem resulting in a  flux $\vsiga$. 
The total difference flux $\v{\sigma}^\triangle$ is then taken as the sum of local fluxes over these patches, 
\[
  \v{\sigma}^\triangle := \sum_{a \in \V_h} \vsiga.
\] 
As proposed  in \cite{braessequilrobust}, local fluxes are found by decomposing the residual using the partition of unity. 
Write $r \in H_0^1(\O)'$ for the usual residual, i.e. for $v\in H_0^1(\O)$ 
\[
  \ip{r,v} = r(v) := a(u - U_h, v) = \ip{\nabla \left(u - U_h\right), \nabla v}_\O = \ip{f,v}_\O - \ip{\nabla U_h,\nabla v}_\O.
\]
For every vertex $a \in V_h$ the local residual at $a$ is defined by
\[
  \ip{r_a,v} := \ip{r,\psi_a v} = a(u - U_h, \psi_a v)_{\w_a}.
\]

The local flux $\vsiga$ is taken from the broken Raviart-Thomas space $\RT_{p}^{-1}(\omega_a)$ such that
\begin{equation}
  \label{eq:defsigma}
  \ip{r_a, v} = - \ip{\v{\sigma_a}, \nabla v}_{\w_a} \quad \forall v \in H^1(\w_a) \cap H_0^1(\O).
\end{equation}
The residual can be expressed in triangle and edge related terms by applying the divergence theorem.
This leads to the following decomposition:
\begin{equation}
  \label{eq:locdecomp}
  \ip{r_a,v} = \ip{f, \psi_a v}_{\w_a} - \ip{\nabla U_h, \nabla (\psi_av)}_{\w_a} = \sum_{ K\subset \w_a} \langle{r^T_a, v}\rangle_K + \sum_{ e \subset \gamma_a} \ip{r^e_a, v}_e,
\end{equation}
with $L^2$ inner products over elements $K$ and (lower dimensional) edges $e$, and
\[
  r^T_a := \psi_a \left[ f + \Delta U_h \right], \quad r^e_a := \psi_a \llbracket \nabla U_h \rrbracket.
\]
Similarly rewriting $\ip{\v{\sigma_a}, \nabla v}_{\w_a}$ in triangle and edge related terms yields,
\[
  -\ip{\v{\sigma_a}, \nabla v}_{\w_a} = \sum_{K \subset \w_a} \ip{\div \v{\sigma_a}, v}_K + \sum_{ e\subset \gamma_a} \ip{\llbracket \v{\sigma_a}\rrbracket, v} + \ip{\vsiga\cdot n, v}_{\partial \w_a \setminus \partial \O}.
\]
After expanding both sides, we see that \eqref{eq:defsigma} holds if and only if
\begin{equation}
  \label{eq:sigmacons}
  \begin{alignedat}{3}
    \div \v{\sigma_a} &= \psi_a \left [f + \Delta U_h\right] && \quad\text{in }  K\subset \w_a ,\\
    \llbracket \v{\sigma_a} \rrbracket &= \psi_a \llbracket \nabla U_h \rrbracket && \quad \text{on } e \subset \gamma_a,\\
    \v{\sigma_a} \cdot n &= 0 &&\quad \text{on } \partial \w_a \setminus \partial \O.
  \end{alignedat}
\end{equation}
By this last prerequisite we can equivalently restrict ourself to finding $\v{\sigma_a}$ in
\[
  \RT_{p,0}^{-1}(\w_a) := \set{ \v{\sigma} \in \RT_p^{-1}(\w_a) : \v{\sigma} \cdot n_e = 0 \quad \text{ for } e \in \partial \w_a \setminus \partial \O}.
\]

To prove that this system has a solution we make the following observation.
For an interior vertex $a \in \V_h^{int}$ the hat function $\psi_a$ belongs to the finite element space~$\VV_h$. 
Using that $U_h$ is the Galerkin approximation therefore gives us $\ip{r_a,\1} = \ip{f, \psi_a}_\O - \ip{\nabla U_h, \nabla \psi_a}_{\O} = 0$.
That is, the local residual vanishes on constant functions.
\begin{thm}
  \label{thm:sigmasolvable}
  There exists a flux $\vsiga \in \RT_{p,0}^{-1}(\w_a)$ that solves the above system \eqref{eq:sigmacons}.
\end{thm}
\begin{proof}
  We use a (well known) result for Raviart-Thomas elements proved in Lemma~\ref{lem:rtexists}. Namely, 
  for an element $K$ there exists an function $\vtau \in \RT_p(K)$ such that
  \[
    \div \vtau = P_K \, \in \P_p(K), \quad \vtau \cdot n = P_e \in \,\P_p(\partial K),
  \]
  if the polynomials satisfy the compatibility constraint $\int_K P_K = \int_\partial K P_e$.
  Notice that the polynomials in \eqref{eq:sigmacons} are indeed of the correct degree.

  \todo{Image patches}
  Fix an interior vertex $a \in \V_h^{int}$, so that $\w_a$ is a convex convex. Let $\set{K_1,\dots,K_n}$ be a numbering
  of the $n$ distinct elements $K$ such that $K_i \cap K_{i+1} =e_i \in \gamma_a$ with $1 \leq i \leq n-1$. Since $a$ is an
  interior vertex we also have $K_1 \cap K_n = e_n$. Every element $K_m$ thus has edges $e_m, e_{m-1}$ and an boundary edge $\tilde e_m$. For $K_1$ we pick a $\vsig_1\in \RT_p(K_1)$ that solves
  \begin{alignat*}{3}
    \div \vsig_1 &= \psi_a \left [f + \Delta U_h\right] && \quad\text{in }  K_1\subset \w_a ,\\
    \vsig_1 \cdot n  &= p_1 && \quad \text{on } e_{1} \subset \gamma_a,\\
    \vsig_1 \cdot n  &= 0 && \quad \text{on } \tilde e_1 \subset \gamma_a,\\
    \vsig_1 \cdot n  &= \psi_a \llbracket \nabla U_h \rrbracket && \quad \text{on } e_n \subset \gamma_a,
  \end{alignat*}
  with $p_1$ a polynomial chosen such that the compatibility condition holds. Similarly we pick $\vsig_2 \in \RT_p(K_2)$
  such that the jump from $\vsig_1$ to $\vsig_2$ over $e_1$ equals $\psi_a \llbracket \nabla U_h \rrbracket$ --- and of course
  with the other required properties. 
  
  We can repeat this process until we arrive at $K_n$, at this point we are no longer
  free to pick the `next'  boundary polynomial. We would like to set 
  $\vsig_n = 0$ on $e_n$, and $\vsig_n = \psi_a \llbracket \nabla U_h \rrbracket - P_{n-1}$ on $e_{n-1}$ so that the jumps to correctly behave.
  From orthogonality on constants we have the identity,
  \[
    \sum_{i = 1}^n \int_{K_i} \psi_a \left[f+\nabla U_h\right] = \sum_{i = 1}^n \int_{e_i} \psi_a \llbracket \nabla U_h \rrbracket,
  \]
  so that for the last element $K_n$ we have
  \[
    \int_{K_n} \psi_a \left[ f+ \nabla U_h \right] = \int_{e_{n-1}} \psi_a \llbracket \nabla U_h \rrbracket - P_{n-1}.
  \]
  This equation follows from the jump behavior in combination with the compatibility condition. From this latter equation it is then clear
  we can also find a correct $\vsig_n \in \RT_p(K_{n})$. Taking $\vsig_a|_{K_i} := \vsig_i$ then provides us with a $\vsig_a \in \RT_{p,0}^{-1}(\w_a)$ that satisfies \eqref{eq:sigmacons}.

  For a boundary vertex $a \in \V^{bdr}$ the process is similar.  We can again find a numbering $\set{K_1, \dots, K_n}$
  of the elements inside the patch $\w_a$ such that $K_i \cap K_{i+1} = e_i$. This time the
  elements $K_1$ and $K_n$ do not share a boundary, since $a$ is a boundary vertex. We can simply the method described above
  without having a conflict at the last element $K_n$, since \eqref{eq:sigmacons} does not prescribe conditions for edges
  on the domain boundary. This process results in a $\vsig_a$ that solves the system \eqref{eq:sigmacons}

  {
    \color{blue}
  We generalize a similar proof found in  \cite[Lemma~3]{braessequil}. Fix an interior vertex $a \in \V_h^{int}$,
  and pick a $\vsig^1 \in \RT_{p,0}^{-1}(\w_a)$ that satisfies the
  following interface conditions:
  \[
    \llbracket \vsig^1 \rrbracket = \psi_a \llbracket \nabla U_h \rrbracket \,\, \text{on } e \subset \gamma_a \quad \text{and}\quad \vsig^1 \cdot n = 0 
    \,\,\text {on } \partial \w_a.
  \]
  This is possible since both the right hand sides are polynomials of degree (at most)~$p$, and
  the normal restriction of $\RT_{p}(K)$ to  $\partial K$ equals $\P_p^{-1}(\partial K)$.  Using the divergence theorem, we see that the mean divergence of $\vsig^1$ satisfies,
  \begin{align*}
    \ip{\div \vsig^1,\1}_{\w_a} &=  
    \sum_{ K \subset \w_a} -\ip{\vsig^1, \nabla \1}_{K} + \sum_{ e \subset \gamma_a} +\ip{\llbracket \vsig^1 \rrbracket, \1}_{e} + \ip{\vsig^1\cdot n, \1}_{\partial \w_a}\\
    &=\sum_{ e \subset \gamma_a} \ip{\psi_a \llbracket \nabla U_h  \rrbracket, \1}_{e}.
  \end{align*}
  On the other hand, by orthogonality on constants of the local residual we also have
  \begin{align*}
    \ip{\psi_a \left[ f + \Delta U_h \right],\1}_{\w_a} &= \ip{r_a, \1} + 
    \sum_{ e \subset \gamma_a}\psi_a \ip{\llbracket \nabla U_h \rrbracket}_e = \sum_{ e\subset \gamma_a} \ip{\psi_a\llbracket \nabla U_h \rrbracket, \1}_e.
  \end{align*}
  We conclude that  $ \psi_a \left[ f+ \Delta U_h\right]  - \div \vsig^1 $ has zero mean.
  
  Both terms in the above difference are broken polynomials,
  so that $\div \vsig^1 - \psi_a \left[ f + \Delta U_h\right] \in \P_p^{-1}(\w_a)$. 
  In \cite{arnold2006differential} it is shown that the following sequence is exact:\footnote{The range of the first operator is the kernel of the second operator.}
  \[
    \set{ \vsig \in \RT_p(\w_a) : \vsig \cdot n = 0  \text{  on } \partial \w_a} \xrightarrow{\div} \P_p^{-1}(\w_a) \xrightarrow{mean} \R.
  \]
  Therefore we can find a $\vsig^2 \in \RT_p(\w_a)$ such that,
  \[
    \div \vsig^2 = \psi_a \left[ f + \Delta U_h\right] - \div \vsig^1\quad  \text{in } \w_a , \quad  \vsig^2 \cdot n = 0 \quad\text{on } \partial \w_a.
  \]
  Since $\vsig^2 \in \RT_p(\w_a)$, we know that $\llbracket \vsig^2 \rrbracket$ vanishes on internal edges (cf. Theorem~\ref{thm:rtjump}).  Setting  $\vsiga = \vsig^1 + \vsig^2$ then yields a  $\vsiga \in \RT_{p,0}^{-1}(\w_a)$ that satisfies \eqref{eq:sigmacons}.

  The proof for $a \in \V_h^{bdr}$ is similar, we use that the following map is surjective: \todo{Ref of bewijs}
  \[
    \set{\vsig \in \RT_p(\w_a) : \vsig \cdot n = 0 \text{ on } \partial \w_a \setminus \partial \O} \to \P_p^{-1}(\w_a).
  \]
}
\end{proof}
Since there is no unique solution that solves the system \eqref{eq:sigmacons}, a \emph{minimal} $L_2$-norm is chosen. In summary,
the locally equilibrated flux $\v{\sigma_a}$ is given by
\[
  \v{\sigma_a} \in \RT_{p,0}^{-1}(\w_a) \quad \text{ s.t. $\v{\sigma_a}$ satisfies \eqref{eq:defsigma} and } \uanorm{\v{\sigma_a}}_{\w_a} = \min_{\v{\sigma} \in \RT_{p,0}^{-1}(\w_a) : \v{\sigma} \text{ satisfies \eqref{eq:defsigma}}} \norm{\v{\sigma}}_{\w_a}.
\]
These locally equilibrated fluxes are lifted to the entire space by taking $\vsig^\triangle = \sum_{a \in \V_h} \vsiga$. 
Finally set $\v{\sigma} = \nabla U_h - \vsig^\triangle$, then from Green's identities we have for $v \in H_0^1(\O)$:
\begin{align*}
  \ip{\vsig, \nabla v}_{\O} &= \ip{\nabla U_h, \nabla v}_{\O} - \sum_{a \in \V_h} \ip{\vsiga, \nabla v}_{\w_a} \\
     &= \ip{\nabla U_h, \nabla v}_{\O} + \sum_{a \in \V_h} \ip{r_a, v} \\
  &= \ip{\nabla U_h, \nabla v}_{\O} + \ip{r,v} = \ip{f,v}_{\O}.
\end{align*}
Since this holds for all $v \in H_0^1(\O)$, we know that $\vsig$ has a weak divergence that is given by $\div \vsig =  -f$. 
Invoking Prager and Synge then gives the wanted upper bound:
\[
  \enorm{u - U_h}_{\O} \leq \norm{\nabla U_h - \v{\sigma}}_{\O} = \uanorm{\vsig^\triangle} = \norm{\sum_{a \in \V_h} \vsiga}.
\]
\section{Residual bounds}
Using the relation between $\v{\sigma_a}$ and the local residual $r_a$ we can directly derive some useful bounds.
Recall that for $a \in \V_h^{int}$  the functional $r_a$ vanishes on constant functions, and therefore we may interpret $r_a$
as a functional on the mean zero space. Again in the notation of~\cite{ernequil}, we define for $a \in \V_h$,
\[
  H^1_\star(\omega_a) := \left[\begin{aligned}
      &\set{v \in H^1(\omega_a):  \ip{v,\1}_{\omega_a} =:v_{\omega_a}  = 0} &\quad a \in \V_h^{int},\\
    &\set{v \in H^1(\omega_a): v = 0 \text{ on }\partial \omega_a \cap \partial \O} &\quad a \in \V_h^{bdr}.
  \end{aligned}
\right.
\]
We equip this space with the norm $\norm{\nabla \cdot }_{\omega_a}$, which is a norm since 
\[
  \norm{\nabla v}_{\omega_a} = 0 \implies v = C_{st}, \quad v \in H_\star^1(\omega_a) \implies C_{st} = 0.
\]
Even stronger,
this norm is equivalent to the $H^1$-norm due to Poincar\'e-Friedrichs inequality (cf. \S\ref{sec:poincfried}).
With these definitions we are able to proof local efficiency and global reliability in terms of $r_a$.
\begin{lem}
  \label{lem:loceff}
  For every vertex $a \in \V_h$ we have the following local efficiency on $\omega_a$,
  \[
    \norm{r_a}_{H_\star^1(\omega_a)'} \lesssim \enorm{ u - U_h}_{\omega_a}.
  \]
\end{lem}
\begin{proof}
  By definition and the Cauchy-Schwarz inequality we find
  \begin{align*}
    \norm{r_a}_{H_\star^1(\omega_a)'} &= \sup_{v \in H_\star^1(\omega_a) : \norm{\nabla v}_{\w_a} = 1} \abs{\ip{r_a, v}}\\
    &= \sup_{v \in H_\star^1(\omega_a) : \norm{\nabla v}_{\w_a} = 1} \abs{\ip{\nabla \left(u - U_h\right), \nabla \left(\psi_a v\right)}_\O} \\
    &\leq \norm{\nabla u - \nabla U_h}_{\omega_a} \sup_{v \in H_\star^1(\omega_a): \|\nabla v\|_{\w_a}=1} \norm{ \nabla \left (\psi_a v \right)}_{\omega_a}.
  \end{align*}
  In order to get rid of the $\psi_a$-term we first apply the product rule,
  \begin{align*}
    \norm{\nabla\left(\psi_a v\right)}_{\omega_a} &= \norm{v \nabla \psi_a + \psi_a \nabla v}_{\omega_a}\\
    &\leq \norm{v \nabla \psi_a}_{\omega_a} + \norm{\psi_a \nabla v}_{\omega_a} \\
    &\leq \norm{v}_{\omega_a} \norm{\nabla \psi_a}_{\infty, \omega_a} + \norm{\psi_a}_{\infty, \omega_a} \norm{\nabla v}_{\omega_a},
  \end{align*}
  where the last inequality follows since $\omega_a$ is a bounded set. For the $\psi_a$ terms we note that 
  $\norm{\psi_a}_{\infty, \omega_a} = 1$, whereas the vector norm $\|\nabla \psi_a\|_{\w_a}$ is constant and bounded on each triangle by
  $\rho_{K}$, and thus $\norm{\nabla \psi_a}_{\infty, \omega_a} \leq C h^{-1}_{\w_a}$ for some constant only depending on the 
  shape regularity $\kappa_h$ of the triangulation and the maximum amount of triangles in a patch. 
  The $\norm{v}_{\omega_a}$-term can be estimated using Poincar\'e-Friedrich inequalities (cf \S\ref{sec:poincfried}):
  \[
    \norm{v}_{\omega_a} = \begin{cases}
      \norm{v - v_{\omega_a}} \leq C_{P,\omega_a} h_{\omega_a}\norm{\nabla v}_{\omega_a} & a \in \V_h^{int}, \\
      \norm{v} \leq C_{F,\omega_a} h_{\omega_a}\norm{\nabla v}_{\omega_a} & a \in V_h^{bdr},
    \end{cases}
  \]
  with the constants depending only on the shape regularity $\kappa_h$ of the triangulation.
  Combining these inequalities and using that $\norm{\nabla v}_{\w_a} = 1$ we arrive at the asserted.
\end{proof}
\begin{lem}
  \label{lem:globrel}
  A global error bound in terms of $r_a$ is given by,
  \[
    \enorm{u - U_h}_{\O}^2 \leq 3 \sum_{a \in \V_h} \norm{r_a}^2_{H^1_\star(\omega_a)'}
  \]
\end{lem}
\begin{proof}
  Since $H_0^1(\O)$ is a Hilbert with respect to $ \ip{\nabla \cdot, \nabla \cdot}_\O$, we find by duality
  \[
    \enorm{u - U_h}_{\O} = \norm{\nabla u - \nabla U_h}_{\O} = \sup_{ v \in H_0^1(\O): v \ne 0 } {\abs{a(u - U_h, v)} \over \norm{\nabla v}_{\O}} = \sup_{ v \in H_0^1(\O): v \ne 0 } {\abs{\ip{r, v}} \over \norm{\nabla v}_{\O} }.
  \]
  Rewriting the residual operator using the partition of unity for $v \in H_0^1(\O)$ yields
  \begin{align*}
    \ip{r , v} &= \sum_{a \in \V_h^{int}} \ip{r_a, {v - v_{\omega_a}}} + \sum_{a \in \V_h^{bdr}} \ip{r_a, v} \\
    &\leq \sum_{a \in \V_h^{int}} \norm{r_a}_{H^1_\star(\omega_a)'} \norm{\nabla (v- v_{\omega_a})}_{\omega_a} 
    + \sum_{a \in \V_h^{bdr}} \norm{r_a}_{H^1_\star(\omega_a)'} \norm{\nabla v}_{\omega_a}\\
    &= \sum_{a \in V_h} \norm{r_a}_{H^1_\star(\w_a)'} \norm{\nabla v}_{\w_a}\\
    &\leq \sqrt{\sum_{a \in \V_h} \norm{\nabla v}_{\w_a}^2} \sqrt{\sum_{a \in \V_h} \norm{r_a}^2_{H^1_\star(\w_a)'}} \leq \sqrt{3}\norm{\nabla v}_{\O}\sqrt{\sum_{a\in \V_h} \norm{r_a}^2_{H_\star^1(\w_a)'}}.
  \end{align*}
  The first equality follows from orthogonality on the constants; the first inequality follows since $v - v_{\omega_a}, v \in H_\star^1(\w_a)$ 
  for an interior resp. boundary vertex $a$; and finally $\sum_{a \in \V_h} \norm{\nabla v}^2_{\omega_a} \leq 3\norm{\nabla v}_{\O}^2$ holds
  as every triangle is contained in a three patches. Combining these inequalities for a non-zero $v\in H_0^1(\O)$ yields
  \begin{align*}
    \norm{\nabla u - \nabla U_h}_{\O} \leq \norm{\nabla v}_{\O}^{-1} \ip{r,v} \leq \sqrt{3} \norm{\nabla v}_{\O}^{-1} \norm{\nabla v}_{\O} \sqrt{\sum_{a \in \V_h} \norm{r_a}_{H^1_\star(\omega_a)'}}.
  \end{align*}
\end{proof}

These bounds become relevant if they can be related to the (local) estimators~$\v{\sigma_a}$. Recall that $\vsiga$
was found as the minimal norm function satisfying \eqref{eq:defsigma}. The fact that this vector-valued function is of minimum norm enabled Braess et al. \cite{braessequilrobust} to prove the following powerful theorem.
\begin{thm}
  \label{thm:locresequiv}
  For $a \in \V_h$ let $\v{\sigma_a}$ be found as described above. 
  We have the bound $\norm{r_a}_{H_\star^1(\omega_a)'} \leq \uanorm{\v{\sigma_a}}_{\omega_a}$. But more importantly,
  a reverse bound also holds for a constant {not} depending on the polynomial-degree $p$ used in the finite element space $\VV_h$:
  \[
    \uanorm{\v{\sigma_a}}_{\omega_a} \lesssim \norm{r_a}_{H_\star^1(\w_a)'}.
  \]
\end{thm}
\begin{proof}
  From \eqref{eq:defsigma} we have for $v \in H_\star^1(\w_a)$ that $\ip{r_a, v} = - \ip{\v{\sigma_a}, \nabla v}_{\w_a}$.
  This directly provides us with the first inequality, i.e.
  \[
    \norm{r_a}_{H_\star^1(\omega_a)'} = \sup_{v \in H_\star^1(\w_a) : \norm{\nabla v}_{\w_a} = 1} \abs{\ip{r_a, v}} = \sup_{v \in H_\star^1(\w_a) : \norm{\nabla v}_{\w_a}=1} \abs{\ip{\v{\sigma_a}, \nabla v}_{\w_a}} \leq \uanorm{\v{\sigma_a}}_{\w_a}.
  \]

  The proof of the second inequality is much more involved. A constructive proof is given in \cite[Theorem~7]{braessequilrobust}.
\end{proof}
\begin{rem}
  Combing the previous Theorem and Lemma's we see that $\sqrt{\sum_{a \in \V_h} \uanorm{\v{\sigma_a}}^2_{\w_a}}$ provides a reliable and efficient estimator.
  When using this estimator in an adaptive finite element method, an obvious refine strategy would consist
  of refining those patches for which $\uanorm{\v{\sigma_a}}_{\w_a}$ is large. Later we will show that this is indeed
  an optimal choice. To prove this, we will relate this estimator the classical residual estimator.
\end{rem}
\begin{lem}
  \label{lem:starequiv}
  The estimator $\uanorm{\v{\sigma_a}}_{\w_a}$ is equivalent to the classical estimator for the patch~$\w_a$, 
  \[
    \uanorm{\v{\sigma_a}}_{\w_a} \approx h_{\w_a}\norm{f + \Delta U_h}_{\w_a} + h^{1/2}_{\w_a} \norm{\llbracket \nabla U_h  \rrbracket}_{\gamma_a}.
  \]
\end{lem}
\begin{proof}
  We start with the $\gtrsim$ relation.  Consider a vertex  $a \in \V_h^{int}$, applying the previous Theorem~\ref{thm:locresequiv} and using orthogonality on constants yields  
  \begin{align*}
    \uanorm{\v{\sigma_a}}_{\w_a} &\geq \norm{r_a}_{H^1_\star(\w_a)'} \\
    &\geq \sup_{v \in H^1_\star(\w_a) : v \ne 0} { \ip{r_a, v} \over \norm{\nabla v}_{\w_a}} \\
    &= \sup_{v \in H^1(\w_a) : v \not \equiv C_{st}} { \ip{r_a, v - v_{\w_a}} \over \norm{\nabla (v - v_{\w_a})}_{\w_a}} \\
          &= \sup_{v \in H^1(\w_a) : v \not \equiv C_{st}} { \ip{r_a, v} \over \norm{\nabla v}_{\w_a}}.
  \end{align*}
  Recall the decomposition of $r_a$ in triangle and edge terms \eqref{eq:locdecomp},
  \[
    \ip{r_a,v} = \sum_{K\subset \w_a} \langle{r^T_a, v}\rangle_K + \sum_{e \subset \gamma_a} \ip{r^e_a, v}_e,
    = \sum_{K\subset \w_a} \langle{\psi_a \left [f+\Delta U_h \right], v}\rangle_K + \sum_{e \subset \gamma_a} \ip{\psi_a \llbracket \nabla U_h \rrbracket, v}_e.
  \]
  In particular $(r^T_a, r^e_a)$ are broken polynomials taken from the spaces $\P_r^{-1}(\w_a)\times \P_r^{-1}(\gamma_a)$ for some degree $r$.

  Fix a triangle $K \subset \w_a$, and consider the space $H_0^1(K)$. Since a function $v \in H_0^1(K)$ vanishes on the boundary, 
  we can naturally extend it to a function $v \in H^1(\w_a)$ by setting $v \equiv 0$ on $\w_a \setminus K$. As $v$
  vanishes on every triangle except $K$ and on  all edges $e \subset \gamma_a$, we have $\ip{r_a, v} = \langle{r_a^T, v}\rangle_K$ and thus
  \[
         \sup_{v \in H^1(\w_a) : v \not \equiv C_{st}} { \ip{r_a, v} \over \norm{\nabla v}_{\w_a}} 
         \geq \sup_{v \in H_0^1(K) : v \ne 0} { \langle{r^T_a, v}\rangle_K \over \norm{\nabla v}_{K}}
         \gtrsim h_K \|{r^T_a}\|_K.
  \]
  \todo{Pak ref triangle, om van de $\psi_a$ term af te komen}
  The last inequality is a known result, since $r^T_a|_K \in \P_r(K)$ (cf. \cite[Ex~9.x.5]{brenner}).
  
  Next, fix an edge $e \subset \gamma_a$ and denote $K_1, K_2$ for the triangles that share this edge. Consider the following space of functions:
  \[
  V_e := \set{v \in H_0^1(K_1 \cup K_2) : \ip{v,P}_{K_1}= \ip{v,P}_{K_2} = 0\quad \forall P \in \P_r}.\]
  Again we can naturally extend these functions to live in $H^1(\w_a)$. Since $v \in V_e$ vanish on all edges except $e$, and
  is orthogonal to polynomials of degree $r$ on $K_1$ and $K_2$, we have
  \[
         \sup_{v \in H^1(\w_a) : v \not \equiv C_{st}} { \ip{r_a, v} \over \norm{\nabla v}_{\w_a}} 
         \geq \sup_{v \in V_e: v \ne 0} { \langle r^e_a, v \rangle_e \over \norm{\nabla v}_{K_1 \cup K_2}}
         \gtrsim h_e^{1/2} \norm{r^e_a}_e.
  \]
  The last inequality is also a standard result, since $r^e_a \in \P_r(e)$ (cf. \cite[Ex~9.x.7]{brenner}).

  Finally, summing the inequalities over $K \subset \w_a$ and $e \subset \gamma_a$ yields
  \begin{align*}
    \norm{r_a}_{H_\star^1(\w_a)'} &\gtrsim \sum_{K \subset \w_a} h_K \|{r^T_a}\|_K + \sum_{e \subset \gamma_a} h_e^{1/2} \norm{r^e_a}_e\\
    &\gtrsim h_{\w_a} \norm{\psi_a \left[ f + \Delta U_h \right]}_{\w_a} + h_{\w_a}^{1/2} \norm{ \psi_a \llbracket \nabla U_h \rrbracket}_{\gamma_a}.
  \end{align*}
  With the last equality following since the number of triangles in a patch is uniformly bounded. The proof is similar for $a \in \V_h^{bdr}$.

  \textcolor{blue}{How can we get rid of $\psi_a$? Away from the boundary $\partial \w_a$ we know that $\psi_a > \epsilon$ so we can bound this term. Not sure how to find a bound on boundary `strip'. Use equivalence of norms on a reference triangle}


  Now the other inequality $\lesssim$. From the previous theorem we know that $\uanorm{\v{\sigma_a}}_{\w_a} \lesssim \norm{r_a}_{H_\star^1(\w_a)'}$. For $v \in H_\star^1(\w_a)$ decompose the local residual $r_a$ in element and edge terms,
  \begin{align*}
    \ip{r_a, v} &= \sum_{K \subset \w_a} \ip{\psi_a\left[f + \Delta U_h\right], v}_{K} + \sum_{e \subset \gamma_a} \ip{\psi_a \llbracket \nabla U_h \rrbracket, v}_e\\
    &\leq \sum_{K \subset \w_a} \norm{\psi_a}_{\infty,K} \norm{f + \Delta U_h}_{K}\norm{v}_{K} + \sum_{e \subset \gamma_a} \norm{\psi_a}_{\infty, e} \norm{ \llbracket \nabla U_h \rrbracket}_e \norm{v}_e\\
    &\leq \sum_{K \subset \w_a} \norm{f + \Delta U_h}_{K}\norm{v}_{K} + \sum_{e \subset \gamma_a} \norm{ \llbracket \nabla U_h \rrbracket}_e \norm{v}_e.
  \end{align*}
  For the element terms we may use Poincar\'e-Friedrichs inequality (cf. \S\ref{sec:poincfried}) to find
  \begin{align*}
    \sum_{K \subset \w_a} \norm{f + \Delta U_h}_{K} \norm{v}_K &\leq \sqrt{\sum_{K \subset \w_a} \norm{f + \Delta U_h}^2_K}\sqrt{\sum_{K \subset \w_a} \norm{v}^2_K} \\
    &= \norm{v}_{\w_a} \norm{f + \Delta U_h}_{\w_a} \\
    &\leq h_{\w_a} C_{PF,\w_a} \norm{\nabla v}_{\w_a} \norm{f + \Delta U_h}_{\w_a}.
  \end{align*}

  For the edge terms, let $e \subset \gamma_a$ be an edge with $K_e$ an adjoint triangle. Applying the trace theorem and transformation lemma
  gives us,
  \[ 
    \norm{v}_e \leq \norm{v}_{\partial K_e} \lesssim h_{K_e}^{-1/2} \norm{v}_{K_e} + h_{K_e}^{1/2} \norm{\nabla v}_{K_e},
  \]
  for a constant only depending on the shape regularity. 
  Summing over all edges yields,
  \[
    \sum_{e \subset \gamma_a} \norm{v}_e \lesssim \sum_{e\subset \gamma_a}h_{K_e}^{-1/2} \norm{v}_{K_e} +  h_{K_e}^{1/2} \norm{\nabla v}_{K_e}
    \lesssim h_{\w_a}^{-1/2} \norm{v}_{\w_a} + h_{\w_a}^{1/2} \norm{\nabla v}_{\w_a} \lesssim h_{\w_a}^{1/2} \norm{\nabla v}_{\w_a}.
  \]
  Where we used that maximum number of triangles in a patch is universally bounded, and that $v \in H_\star^1(\w_a)$ to apply the
  Poincar\'e-Friedrichs inequality. For the edge terms we thus have\todo{Can we use CS for sequences as well?}
  \[
    \sum_{e \subset \gamma_a} \norm{ \llbracket \nabla U_h \rrbracket}_e \norm{v}_e \leq \norm{\llbracket \nabla U_h \rrbracket}_{\gamma_a} \sum_{e \subset \gamma_a} \norm{v}_e \lesssim h_{\w_a}^{1/2} \norm{\llbracket \nabla U_h \rrbracket}  \norm{\nabla v}_{\gamma_a}. 
  \]
  Combining the inequalities in the last three paragraphs yields the wanted upper bound.
\end{proof}
\section{Oscillation}
Thus far we have assumed the right hand side $f$ to be a broken polynomial of degree at most $p-1$. In practice one 
often has a more general $f \in L^2(\O)$, and thus we should use alter the above method to replace $f$ with a computable function.
The general approach is replace the exact $f$ with a polynomial approximation. 
Denote $Q_K$ for the $L^2(K)$-orthogonal projector onto polynomials of degree $p$ on $K$.
Similarly, notate $Q_a$ for the $L^2(\w_a)$-orthogonal projection on the broken polynomial space $\P_p^{-1}(\w_a)$.
We will replace the (exact) local residuals~$r_a$ by discretized local residuals  $\tilde r_a  \in H_0^1(\O)'$ where 
\begin{align*}
  \ip{\tilde r_a, v} &:= \ip{Q_a(\psi_af),v}_{\w_a} - \ip{\nabla U_h, \nabla \left(\psi_a v\right)}_{\w_a} \\
   &= \sum_{K \subset \w_a} \ip{Q_K \left(\psi_a \left[f + \Delta U_h\right]\right), v}_K 
  +\sum_{e \subset \gamma_a} \ip{\psi_a \llbracket \nabla U_h\rrbracket, v}_e.
\end{align*}
Since $\1 \in \P_p^{-1}(\w_a)$ and $Q_a$ is an orthogonal projector we have 
\[
  \ip{Q_a(\psi_a f), \1}_{\w_a} = \ip{\psi_a f, Q_a(\1)}_{\w_a} = \ip{\psi_a f, \1},
\]
and thus $\tilde r_a$ also vanishes on constants by the Galerkin orthogonality. We can therefore interpret $\tilde r_a$ as a functional on the space  $H_\star^1(\w_a)$.

From now on $\v{\sigma_a}$ will be constructed using the discretized local residual. That is, 
for each vertex $a \in \V_h$ we let  $\v{\sigma_a}\in\RT_{p,0}^{-1}(\w_a)$ be the minimal $L^2$-norm
flux satisfying
\begin{equation}
  \label{eq:sigmadefdisc}
  \ip{\tilde r_a, v} = - \ip{\v{\sigma_a}, \nabla v}_{\w_a} \quad \forall v \in H^1(\w_a) \cap H_0^1(\O).
\end{equation}
The proof that such a flux $\vsiga$ exists is identical to the proof
of Theorem~\ref{thm:sigmasolvable} because $\tilde r_a$ also
vanishes on constant functions.


Of course, this discretization comes at a price. The
constant-free estimate provided by Prager and Synge
has the following discretized counterpart (cf. \cite[Thm~3.1]{ernequil})
\begin{thm}
  \label{thm:discsynge}
  Let $\vsiga$ be as described above and $\vsig^\triangle = \sum_{a \in \V_h} \vsiga$  then
  \[
    \enorm{u - U_h}_{\O}^2 \leq \sum_{K \in \T_h} \left[ \uanorm{\vsig^\triangle}_{K} + \frac{h_K}{\pi} \uanorm{ (I-Q_a)(\psi_a f)}_{K}\right]^2.
  \]
\end{thm}
\begin{proof}
  Compare the decomposition of $r_a$ and $\tilde r_a$ in terms
  of triangle and edge terms. Since $Q_K(\psi_a \Delta U_h) = \psi_a \Delta U_h$ one easily sees that most terms in the  difference  $r_a - \tilde r_a$ cancel; we are left with the $\psi_a f$ term. Fix $v \in H_0^1(\O)$, the residual is given by
  \begin{align*}
    \ip{r, v} &= \sum_{a \in \V_h} \ip{r_a, v} = \sum_{a \in \V_h} \ip{\tilde r_a , v} + \sum_{a \in \V_h} \sum_{ K \subset \w_a}\ip{(I - Q_K)(\psi_a f), v}_{K}\\
    &= \sum_{a \in V_h} -\ip{\vsiga, \nabla v}_{\w_a} + \sum_{K \in \T_h} \sum_{\set{a \in \V_h : a \in \partial K}} \ip{(I - Q_K)(\psi_a f),v}_K\\
    &=  \sum_{K \in \T_h} \left[-\ip{\vsig^\triangle, \nabla v}_{K} +  \ip{(I - Q_K)f,v}_K \right].
  \end{align*}
  Here we used that $\psi_a$ forms a partition of unity on each
  element $K$ when summed over its vertices $K$. The projector $(I - Q_K)$ is orthogonal to constant functions, so that we may replace $v$ by its mean zero variant $v - v_{K}$. Doing so and invoking
  Cauchy-Schwarz' shows
  \begin{align*}
    \ip{r, v} &\leq \sum_{K \in \T_h} \left[ \uanorm{\vsig^\triangle}_K\uanorm{\nabla v}_K + \uanorm{(I - Q_K)f}_K\uanorm{v - v_{K}}_K\right]\\
    &\leq \sum_{K \in \T_h} \left[ \uanorm{\vsig^\triangle}_K + \frac{h_K}{\pi}\uanorm{(I - Q_K)f}_K\right]\norm{\nabla v}_K \\
    &\leq \sqrt{\sum_{K \in \T_h} \norm{\nabla v}^2_K} \sqrt{\sum_{K \in \T_h}\left[ \uanorm{\vsig^\triangle}_K + \frac{h_K}{\pi}\uanorm{(I - Q_K)f}_K\right]^2}.
  \end{align*}
  The second inequality follows from the Poincar\'e inequality for the convex
  domain $K$, i.e. $C_{P,K} = \pi^{-1}$ --- see also \S\ref{sec:poincfried}. The result follows from the duality given by $ \enorm{u - U_h}_\O = \sup_{ v\in H_0^1(\O) : \norm{\nabla v}_{\O} = 1} \ip{r,v}$.
\end{proof}
Similarly Lemma's \ref{lem:loceff}, \ref{lem:globrel} and Theorem \ref{thm:locresequiv} concerning local bounds on $\w_a$  have
the following discretized counterpart. 
\begin{thm}
  \label{thm:discbounds}
  Fix $a \in \V_h$. The discretized local residual provides reliability by
  \begin{align*}
    \enorm{u - U_h}_{\O}^2 &\leq 3\sum_{a\in \V_h} \left[\norm{\tilde r_a}_{H^1_\star(\w_a)'} + C_{PF,\w_a} h_{\w_a} \norm{(I - Q_a)(\psi_a f)}_{\w_a}\right]^2\\
    &\lesssim \sum_{a \in \V_h} \norm{\tilde r_a}_{H^1_\star(\w_a)'}^2 + h_{\w_a}^2\norm{(I - Q_a)(\psi_a f)}_{\w_a}^2.
  \end{align*}
  and local efficiency with
  \[
    \norm{\tilde r_a}_{H_\star^1(\w_a)'} \lesssim \enorm{u - U_h}_{\w_a} + h_{\w_a} \norm{(I - Q_a) (\psi_a f)}_{\w_a}.
  \]

  For $\v{\sigma_a}$ constructed as described above we have
  \[
    \uanorm{\v{\sigma_a}}_{\w_a} \lesssim \norm{\tilde r_a}_{H_\star^1(\w_a)'} \leq \uanorm{\v{\sigma_a}}_{\w_a}.
  \]
  for a constant only depending on the shape regularity of the triangulation.
\end{thm}
\begin{proof}
  Calculate the norm of $r_a - \tilde r_a$:
  \begin{align*}
    \norm{r_a - \tilde r_a}_{H^1_\star(\w_a)'} &= \sup_{v \in H_\star^1(\w_a) : v \ne 0} \frac{\ip{r_a,v} - \ip{\tilde r_a, v}}{\norm{\nabla v}_{\w_a}} \\
    &= \sup_{v \in H_\star^1(\w_a) : v \ne 0} \frac{ \ip{(I - Q_a)(\psi_a f), v}_{\w_a}}{\norm{\nabla v}_{\w_a}}\\
    &\leq \sup_{v \in H_\star^1(\w_a) : v \ne 0} \frac{\norm{v}_{\w_a} \norm{(I - Q_a)(\psi_a f)}_{\w_a}}{\norm{\nabla v}_{\w_a}}\\
    &\leq h_{\w_a} C_{PF,\w_a} \norm{(I - Q_a)(\psi_a f)}_{\w_a},
  \end{align*}
  where the last inequality follows (again) from Poincar\'e-Friedrichs inequality.

  Reliability follows from the triangle inequality in combination with Lemma~\ref{lem:globrel},
  \begin{align*}
    \enorm{u - U_h}^2_{\O} &\leq 3 \sum_{a \in \V_h} \norm{r_a}^2_{H_\star^1(\w_a)'}\\
    &\leq 3 \sum_{a \in \V_h} \left[ \norm{\tilde r_a}^2_{H_\star^1(\w_a)'} + \norm{r_a - \tilde r_a}^2_{H_\star^1(\w_a)'}\right]\\
    &\leq 3 \sum_{a\in \V_h} \left[\norm{\tilde r_a}_{H^1_\star(\w_a)'} + C_{PF,\w_a} h_{\w_a} \norm{(I - Q_a)(\psi_a f)}_{\w_a}\right]^2\\
    &\lesssim \sum_{a \in \V_h} \norm{\tilde r_a}_{H^1_\star(\w_a)'}^2 + h_{\w_a}^2\norm{(I - Q_a)(\psi_a f)}_{\w_a}^2,
  \end{align*}
  where we used Young's inequality $(x+y)^2 \leq 2x^2 + 2y^2$ for positive $x,y$.

  The efficiency result follows directly from the triangle inequality and Lemma \ref{lem:loceff}.

  The proof of Theorem~\ref{thm:locresequiv} holds for any functional $r \in H_\star^1(\w_a)'$ that can be decomposed in triangle and edge terms.
  In particular it holds for $\tilde r_a$, providing us with the last part of the theorem.
\end{proof}
The oscillation terms will also appear in the local equivalence with the standard residual estimator. 
The effects are summarized in the following lemma. The proof follows easily by mimicking the proof of Lemma \ref{lem:starequiv}
using the results from the previous Theorem.
\begin{lem}
  \label{lem:locequivosc}
  The estimator $\uanorm{\v{\sigma_a}}_{\w_a}$ is locally equivalent to the standard residual estimator up to oscillation terms.
  To be precise, 
  \begin{align*}
    \norm{\v{\sigma_a}} + h_{\w_a}\norm{(I - Q_a)(\psi_a f)}_{\w_a} &\gtrsim h_{\w_a} \norm{f + \Delta U_h}_{\w_a} + h^{1/2}_{\w_a} \norm{\llbracket \nabla U_h \rrbracket}_{\gamma_a},\\
    \norm{\v{\sigma_a}} &\lesssim h_{\w_a}\norm{(I - Q_a)(\psi_a f)}_{\w_a} + h_{\w_a} \norm{f + \Delta U_h}_{\w_a} + h^{1/2}_{\w_a} \norm{\llbracket \nabla U_h \rrbracket}_{\gamma_a}.
  \end{align*}
\end{lem}
{\color{blue}
  By the above method we still need to use the $p$-th order Raviart-Thomas space to generate estimations $\sigma_a$.
  We could use the $p-1$-th order Raviart-Thomas space by defining
  \[
    \ip{\hat r_a,v} := \sum_{K \subset \w_a} \ip{Q_K \left(\psi_a[f + \Delta U_h]\right), v}_K +
    \sum_{e \subset \gamma_a} \ip{Q_e \left(\psi_a \llbracket \nabla U_h \rrbracket \right),v}_e,
  \]
  for $Q_k,Q_e$ the $L^2$-orthogonal projections on the polynomial space $\P_{p-1}(K)$ resp $\P_{p-1}(e)$.
  This holds since $\llbracket \nabla U_h \rrbracket \in \P_{p-1}(e)$, so that
  \[
    \sum_{a \in \V_h} Q_e \left(\psi_a \llbracket \nabla U_h \rrbracket\right) = Q_e (\llbracket \nabla U_h \rrbracket) = \llbracket \nabla U_h \rrbracket.
  \]
  We thus still have the upper bound of Theorem \ref{thm:discsynge}.
  We do get the extra oscillation term $h_{\w_a}^{1/2} \norm{(I- Q_{a,e})(\psi_a \llbracket \nabla U_h \rrbracket)}_{\gamma_a}$.
  For $Q_{a,e}$ the orthogonal projector on the broken space $\P_{p-1}^1(\gamma_a)$.

  We still have efficiency --- albeit with an extra oscillation term --- but we are able to calculate
  the local flux in the lower dimensional $\RT_{p-1, 0}(\w_a)$.
}

\section{Adaptive finite element method}
\label{sec:afemequil}
We will now consider an adaptive finite element method using the equilibrated residual estimator.
Recall that AFEM can be described by the following loop,
\[
  \texttt{SOLVE} \to \texttt{ESTIMATE} \to \texttt{MARK} \to \texttt{REFINE}.
\]
In words: calculate the Ritz-Galerkin solution for a triangulation, estimate
the error using an a posteriori error estimator, mark elements that need to be refined, refine
the triangulation, and start over. 

After specifying the details of these four modules, we will be able to show that the produced
sequence of Galerkin approximations converges with an optimal rate. This result
is similar to the classical one in \S\ref{sec:afem}, except that we will now use a different estimator. This
result is a simplified form on the proof given by Casc\'on and Nochetto \cite{cascon2012}.
A different optimality proof is given by Kreuzer and Siebert \cite{ainsworthbernstein}; therein
slightly different versions of \texttt{MARK} and \texttt{REFINE} are used.

First we need some additional notation. Given a triangulation $\T$ we denote $\W$ for the set of star patches, i.e. 
\todo{Better notation than $\W$?}
\[
  \W := \set{\w_a : a \in \V}.
\]
For $a \in \V$ let $\v{\sigma_a}$ be the equilibrated local flux corresponding to the discrete solution $U \in \V(\T)$. We notate the
  error estimator, oscilation and total error resp. by:
\begin{align*}
  \eta^2_\T(U, \w_a) &:= \uanorm{\v{\sigma_a}}^2_{\w_a}, \\
  \osc^2_\T(f, \w_a) &:=h^2_{\w_a} \norm{(I - Q_a) (\psi_a f)}^2_{\w_a}, \\
  \vartheta^2_\T(U, \w_a) &:= \eta^2_\T(U, \w_a) + \osc^2_\T(f, \w_a).
\end{align*}
The above quantities extend to sets of patches in the usual way 
--- just decompose it as a sum over the patches in the set.
Often we drop $a$ from the notation, but we stress that any patch $\w \in \W$ is implicitly
associated to a vertex $a \in \V$.


\paragraph{SOLVE}Given a triangulation $\T$, this method computes
\[
  U = \texttt{SOLVE}(\T),
\]
with $U \in \VV(\T)$ the  Ritz-Galerkin solution in \emph{exact} arithmetic. The effects of using an inexact solver have
been studied in the literature as well, see for example \cite[\S7]{carstensen2014axioms}.
\paragraph{ESTIMATE} Given a triangulation $\T$ with vertices $\V$ and its discrete solution $U \in \VV(\T)$, this module 
calculates the total error indicators --- the equilibrated estimator plus the oscillation. We have
\[
  \set{\vartheta_\T(U, \w_a)}_{a \in \V} = \texttt{ESTIMATE}(U, \T).
\]
\paragraph{MARK}
A \emph{D\"orfler Marking} strategy is used for marking. For a parameter $\theta \in (0,1]$ we calculate
\[
  \mathcal{M} = \texttt{MARK}( \set{\vartheta_\T(U, \w_a)}_{a \in \V}, \T),
\]
where $\mathcal{M} \subset \W$ is the \emph{minimal} cardinality set that satisfies
\[
  \vartheta_\T(U,\mathcal{M}) \geq \theta \vartheta_\T(U, \W).
\]
\begin{rem}
  In contrast to the standard residual estimator, marking is done using the total error.
\end{rem}
\todo{Marking for total error really needed?}
\paragraph{REFINE}
The \texttt{MARK} outputs a set $\mathcal{M}$ of \emph{patches} on which the error is proportionally large. 
Logically, all the triangles in this set should be refined. This module calculates a refinement
\[
  \T_\star = \texttt{REFINE}(\T, \mathcal{M}),
\]
with $\T_\star$ the smallest conforming refinement of $\T$ such that the triangles of
all patches in the marked set $\M$ are bisected.
\todo{Weird refine method in cascon en nochetto}
\\\\
We can now formulate the AFEM algorithm and its produced sequences. For this we use the
iteration counter $k$ as a subscript to differentiate among the sets.
Let an initial triangulation $\T_0$ be given, set $k = 0$ and iterate the following steps:
\begin{enumerate}
\item $U_k = \texttt{SOLVE}(\T_k)$;
\item $\set{\vartheta_k(U_k, \w_a)}_{a \in \V_k} = \texttt{ESTIMATE}(U_k, \T_k)$;
  \item $\mathcal{M}_k = \texttt{MARK}(\set{\vartheta_k(U_k, \w_a)}_{a \in \V_k}, \T_k)$;
  \item $\T_{k+1} = \texttt{REFINE}(\T_k, \mathcal{M}_k)$;
  \item $k  = k + 1$.
\end{enumerate}

In \cite[\S 4]{cascon2012} a list of assumptions are given under which the above AFEM algorithm produces
an optimal sequence of solutions $(U_k, \T_k)_{k \geq 0}$. First we will show that these
assumptions --- compare \cite[Assump~4.1]{cascon2012} --- hold using the equilibrium based estimator as described above.
In the standard case we had a refined set containing the triangles that were refined. In this case
we need to define something similar, but then in terms of refined patches:
\[
  \RR_{\T \to \T_\star} := \W \setminus \W_\star =  \set{ \w_a \in \W :  K \not \in \T_\star\quad \text{ for some }  K \subset \w_a}.
\]
So $\RR_{\T \to \T_\star}$ consists of all patches in $\T$ that contain at least \emph{one} triangle that is refined. 
\begin{lem}
  \label{lem:assumptions}
  Let a triangulation $\T$ and a refinement $\T_\star \geq \T$ be given, with $U \in \VV(\T)$
  and $U_\star \in \VV(\T)$ the appropriate discrete solutions. There
  exists constants $C_1, C_2, C_3$ such that:
  \begin{enumerate}
    \item The estimator is reliable; the approximation error can be bound using the total error estimator:
      \begin{equation}
        \label{eq:globupper}
        \enorm{ u - U}_{\O}^2 \leq C_1 \vartheta^2_\T(U, \W) = C_1 \left [ \eta^2_\T(U, \W) + \osc^2_\T(f, \W)\right]
      \end{equation}
    \item The estimator is efficient; the estimator provides a lower bound for the error (up to oscillation):
      \begin{equation}
        \label{eq:globlower}
        C_2 \eta^2_\T(U, \W) \leq \enorm{u - U}^2_{\O} + \osc_\T^2(f, \W).
      \end{equation}
    \item The estimator provides a localized upper bound; the error between $U$ and $U_\star$ can be bound using the
      refined set $\RR_{\T \to \T_\star}$:
      \begin{equation}
        \label{eq:locupper}
        \enorm{U_\star - U}^2_{\O}  \leq C_1\vartheta^2_\T(U, \RR_{\T \to \T_\star}).
      \end{equation}
    \item The estimator provides a local lower bound, i.e. the reverse of the above
      \begin{equation}
        \label{eq:loclower}
        C_3 \eta^2_\T(U, \wt \RR_{\T \to \T_\star}) \leq \enorm{U_\star - U}^2_{\O}  + \osc_\T^2(U, \RR_{\T \to \T_\star}).
      \end{equation}
      With $\wt \RR_{\T \to \T_\star}$ the set of patches that are entirely refined when going from $\T$ to $\T_\star$.
      \textcolor{blue}{In \cite{cascon2012} this is actually defined in terms of $\wt \RR = \RR^n$, this consists
        of all refined patches that satisfy the interior node property. In \cite{kreuzersiebert} this
        restriction is not made, instead the estimator is locally compared to the residual estimator.
      The latter is what we do as well, so I don't think we need this additional interior node property.}
  \end{enumerate}
\end{lem}
\begin{proof}
  Reliability follows directly from Theorem \ref{thm:discbounds},
  \[
    \enorm{u - U}_{\O}^2 \lesssim \sum_{a\in \V_h}\norm{\tilde r_a}^2_{H^1_\star(\w_a)'} + h^2_{\w_a} \norm{(I - Q_a)(\psi_a f)}^2_{\w_a} \leq \sum_{a \in V_h} \uanorm{\v{\sigma_a}}^2_{\w_a} + \osc_\T^2(f, \w_a) = \vartheta^2_\T(U, \W)
  \]
  By the same theorem we can also deduce global efficiency,
  \begin{align*}
    \eta_\T^2(U, \W)  &= \sum_{a \in \V} \uanorm{\vsiga}_{\w_a}^2 \lesssim \sum_{a \in \V} \norm{\tilde r_a}^2_{H^1_\star(\w_a)'} \\
    &\lesssim \sum_{a \in \V} \enorm{u - U}_{\w_a}^2 + h_{\w_a}^2 \norm{(I-Q_a)(\psi_a f)}_{\w_a}^2 \\
    &= 3 \enorm{u - U}^2_{\O} + \osc_\T^2(f, \W).
  \end{align*}
  The last equality follows since every triangle is contained in exactly three patches.

  The localized upper bound follows from equivalence with the classical residual estimator. Denote $\hat \eta_\T$ for the
  classical residual estimator (cf. Definition \ref{def:clasest}). 
  By Theorem~\ref{thm:residual_erro} with $\hat \RR = \T \setminus \T_\star$ we have a localized upper bound for the standard 
  estimator $\hat \eta_\T$: 
  \[
    \enorm{ U_\star -  U}^2 \lesssim \hat\eta^2_\T(U, \hat\RR) =\sum_{K \in \hat \RR}h^2_K\norm{f +\Delta U}_K^2+h_K \norm{\llbracket \nabla U  \rrbracket}^2_{\partial K \setminus \partial \O}.
  \]
  For any patch $\w_a$ containing a triangle $K$ we obviously have $h_K^2\norm{f + \Delta U}_K^2 \leq h_{\w_a}^2\norm{f + \Delta U}_{\w_a}^2$.
  Consider an edge $e \subset \partial K \setminus \partial \O$.
  Since this is a non-boundary edge, we know that it must be part of the skeleton $\gamma_a$ for two vertices $a$ of $K$. Hence for $\V_K$, the
  vertices of $K$, we have $\norm{\llbracket \nabla U \rrbracket}_{\partial K \setminus \partial \O}^2 \leq \sum_{a \in \V_K} \norm{\llbracket \nabla U\rrbracket}^2_{\gamma_a}$.
  From these remarks we deduce 
  \begin{align*}
    \enorm{ U_\star -  U}^2 &\lesssim 
  \sum_{K \in \hat \RR}\sum_{a \in \V_K} h_{\w_a}^2 \norm{f + \Delta U}_{\w_a}^2 + h_{\w_a} \norm{\llbracket \nabla U \rrbracket}_{\gamma_a}^2\\
  &\lesssim \eta^2_\T(U, \RR) + \osc_\T^2(f,  \RR) = \vartheta_\T^2(U, \RR),
\end{align*}
  where the last inequality follows from the equivalence between the two estimators on patches; see Lemma \ref{lem:locequivosc}.
  Here we used $\RR := \RR_{\T \to \T_\star} = \{\w_a \in \W : a \in \V_K, K \in \hat \RR\}$.

  Proceed similarly for the discrete lower bound, i.e. invoke Lemma \ref{lem:locequivosc} for equivalence
  with the standard residual estimator. Writing $\wt \RR := \wt \RR_{\T \to \T_\star}$ this gives
  \begin{align*}
    \eta_\T^2(U, \wt \RR) &= \sum_{\w_a \in \wt \RR} \norm{\vsiga}_{\w_a}^2\\
    &\lesssim \sum_{\w_a \in \wt \RR} \left[ h^2_{\w_a}\norm{(I - Q_a)(\psi_a f)}^2_{\w_a} + h^2_{\w_a} \norm{f + \Delta U}^2_{\w_a} + h_{\w_a} \norm{\llbracket \nabla U \rrbracket}^2_{\gamma_a} \right].
  \end{align*}
  Focus on the last two terms of this expression. Every patch $\w_a \in \wt \RR$ consists entirely  refined triangles.
  Using that $h_{\w_a} \approx h_K$ for every $K \subset \w_a$ shows
  \[
    \sum_{\w_a \in \wt \RR} h^2_{\w_a} \norm{ f + \Delta U}^2_{\w_a} \lesssim \sum_{\w_a \in \wt \RR} \sum_{K \subset \w_a} h^2_K\norm{f + \Delta U}^2_K \lesssim \sum_{K \in \hat \RR} h^2_K\norm{f + \Delta U}^2_K.
  \]
  Where we used that every triangle occurs in a maximum of three patches, and that $\hat \RR$ consists of \emph{all} refined triangles.
  Similar arguments show that the edge terms satisfy,
  \[
    \sum_{\w_a \in \wt \RR} h_{\w_a} \norm{\llbracket \nabla U \rrbracket}_{\gamma_a}^2 \lesssim \sum_{K \in \hat \RR} h_K \norm{ \llbracket \nabla U \rrbracket}^2_{\partial K \setminus \partial \O}.
  \]
  Combining these reads,
  \begin{align}
    \eta_\T^2(U, \wt \RR) &\lesssim \sum_{\w_a \in \wt \RR} h_{\w_a}^2 \norm{(I-Q_a)(\psi_a f)}^2_{\w_a} + \sum_{K \in \hat \RR} h_K^2\norm{f + \nabla U}^2_K + h_K\norm{\llbracket \nabla U \rrbracket}^2_{\partial K \setminus \partial \O}\nonumber\\
     &= \osc^2_\T(f, \wt \RR) + \hat \eta^2_\T(U, \hat \RR)\nonumber \\
    &\lesssim \osc^2_\T(f, \wt \RR) + \enorm{u - U}^2_{\O} + \hat\osc_\T^2(f, \hat \RR) \label{eq:disclowprf}.
  \end{align}
  The last equality follows from the discretized local lower bound for the standard estimator --- with $\hat \osc$ the 
  standard oscillation as in Definition \ref{def:clasest}.
  The proof is completed by noting that for an element $K$ we have $\sum_{a \in \V_K} \psi_a = \1|_K$ and thus
  \begin{align*}
    \hat \osc_\T^2(f, K) &= h_K^2 \norm{(I - Q_K)(f)}^2_K \\
    &\leq \sum_{a \in \V_K} h_K^2 \norm{(I - Q_K)(\psi_a f)}^2_K \lesssim h^2_{\w_a} \sum_{a\in \V_K} \norm{(I - Q_K)(\psi_a f)}^2_K.
  \end{align*}
  Again using that $\RR = \{\w_a \in \W : a\in \V_K, K \in \hat R\}$ shows that
  $\osc_\T^2(f, \wt \RR) + \hat \osc^2_\T(f, \hat \RR) \lesssim \osc_\T^2(f, \RR)$. Together with \eqref{eq:disclowprf} this gives
  the lower bound.
\end{proof}
Furthermore the oscillation satisfies the reduction property.
\begin{lem}
  \label{lem:oscasum}
  The oscillation satisfies the oscillation reduction property. That is, for any $V \in \VV(\T)$ and $V_\star \in \VV(\T_\star)$ with $\T \leq \T_\star$  there exists a constant $0 < \lambda < 1$ such that
  \[
    \osc_{\T_\star}^2(f, \W_{\T_\star}) \leq \osc^2_{\T}(f, \W_{\T}) - \lambda \osc^2_{\T}(f, \RR_{\T \to \T_\star}).
  \]
\end{lem}
\begin{proof}
  Denote $\W = \W_{\T}$ and $\W_\star = \W_{\T_\star}$. Rewriting the left hand side yields,
  \begin{align*}
    \osc_{\T_\star}^2(f, \W_\star) &= \osc^2_{\T_\star}(f, \W_\star \cap \W) + \osc^2_{\T_\star}(f, \W_\star \setminus \W)\\
    &= \osc^2_{\T}(f, \W_\star \cap \W) + \osc^2_{\T_\star}(f, \W_\star \setminus \W)\\
    &= \osc^2_{\T}(f, \W) - \osc^2_{\T}(f, \W \setminus \W_\star) + \osc_{\T_\star}^2(f, \W_\star \setminus \W).
  \end{align*}
  Since $\RR_{\T \to \T_\star} = \W \setminus \W_\star$ the result holds if there exists $0 < \lambda < 1$ such that
  \begin{equation}
    \label{eq:blabla}
    \osc_{\T_\star}^2(f, \W_\star \setminus \W) \leq (1 - \lambda) \osc_{\T}^2(f, \W\setminus \W_\star).
  \end{equation}
  
  Expand both oscillation terms as sum over their patches:
  \begin{align*}
    \osc_{\T_\star}^2(f,\W_\star \setminus \W) &= \sum_{\w_a \in \W_\star \setminus \W} h_{\w_a}^2 \norm{(I-Q_a)(\psi_a f)}^2_{\w_a} =
    \sum_{\w_a \in \W_\star \setminus \W} \sum_{K \subset \w_a} h_{\w_a}^2 \norm{(I-Q_K)(\psi_a f)}_{K}^2,\\
    \osc_{\T}^2(f,\W \setminus \W_\star) &= \sum_{\w_a \in \W \setminus \W_\star} h_{\w_a}^2 \norm{(I-Q_a)(\psi_a f)}^2_{\w_a} =
    \sum_{\w_a \in \W \setminus \W_\star} \sum_{K \subset \w_a} h_{\w_a}^2 \norm{(I-Q_K)(\psi_a f)}_{K}^2.
  \end{align*}
  If all the terms on the right hand side of \eqref{eq:blabla} vanish, then the left hand side vanishes trivially, and the result 
  is true. Therefore we may assume that we have non-vanishing terms. For an unrefined element $K$ in the above sum the
  norm $\norm{(I - Q_K)(\psi_a f)}_K^2$ appears on both sides; but the diameter coefficient $h_{\w_a}$ is non-increasing
  when refining a patch. For these terms we thus have the inequality,
  \[
    \sum_{\set{K \in \T \cap \T_\star : K \in \w_a \in \W_\star \setminus \W}} h_{\w_a}^2 \norm{(I-Q_K)(\psi_af)}^2_K \leq
    \sum_{\set{K \in \T \cap \T_\star : K \in \w_a \in \W \setminus \W_\star}} h_{\w_a}^2 \norm{(I-Q_K)(\psi_af)}^2_K.
  \]

  More interesting are the refined elements $K$. Let $K \in \T$ be bisected with children $K_1, K_2 \in \T_{\star}$, i.e. $K = K_1 \cup K_2$.
  Denote $a_1,a_2,a_3$ for the vertices of $K$, and suppose that $a_{12}$ is the vertex introduced by bisection on edge $a_1a_2$.
  Then under the assumption that $(I-Q_K)(\psi_{a_3}f)$ does not vanish on $K$ we find\todo{Strict inequal true?}
  \[
    \norm{(I-Q_{K_1})(\psi_{a_3}f)}^2_{K_1} + \norm{(I-Q_{K_2})(\psi_{a_3}f)}^2_{K_2} < \norm{(I-Q_K)(\psi_{a_3}f)}^2_K.
  \]
  {\color{blue}
  How do we proceed? Is this even true? We must have something like
  \begin{align*}
    &\norm{(I-Q_{K_1})(\psi_{a_1}f)}^2_{K_1} + \norm{(I-Q_{K_2})(\psi_{a_2}f)}^2_{K_2} \\
    +&\norm{(I-Q_{K_1})(\psi_{a_{12}}f)}^2_{K_1} + \norm{(I-Q_{K_2})(\psi_{a_{12}}f)}^2_{K_2}\\
    \leq &\norm{(I-Q_K)(\psi_{a_1}f)}^2_K + \norm{(I-Q_K)(\psi_{a_2}f)}^2_K
  \end{align*}
  How can we proof such a thing? Annoying that the newly introduced node did not appear before. 
}
\end{proof}


We can now prove the so-called contraction property; this shows a weighted convergence of the AFEM method described above.
\begin{thm}
There exists a constant $0 < \alpha < 1$ and $\gamma > 0$, depending on the shape regularity of $\T_0$ and $\lambda, \theta$ such that
\[
  \enorm{ u - U_{k+1}}^2 +\gamma \osc^2_{k+1}(f, \W_{k+1}) \leq \alpha^2\left(\enorm{ u - U_k}^2 + \gamma\osc^2_k(f, \W_k)\right)
\]
\end{thm}
\begin{proof}
  Adapt the notation used in \cite{cascon2012}, i.e. for $j =0,1,2,\dots$ we write
  \begin{alignat*}{2}
    e^2_j &:= \enorm{u - U_j}^2_{\O}, &\quad E^2_j &:= \enorm{U_{j+1} - U_j}_{\O}^2,\\
    \osc^2_j &:= \osc_{\T_{j}}^2(f, \W_{\T_{j}}), &\quad \osc^2_j(\mathcal{M}_j) &:= \osc^2_{\T_j}(f, \mathcal{M}_j), \\
    \eta^2_j &:= \eta_{\T_j}^2(U_j, \W_{\T_j}), &\quad \eta^2_j(\mathcal{M}_j) &:= \eta^2_{\T_j}(U, \mathcal{M}_j),\\
    \RR_j & := \RR_{\T_j \to \T_{j+1}}.
  \end{alignat*}

  A piecewise polynomial on $\T_{k}$ is also a piecewise polynomial on $\T_{k+1}$, since $\T_{k+1} \geq \T_k$.
  Therefore we have $U_{k+1} - U_{k} \in \VV(\T_{k+1})$; so by Galerkin orthogonality one has,
  \[
    \enorm{u - U_k}_{\O}^2 = \enorm{u - U_{k+1}}^2_{\O} + \enorm{U_{k+1} - U_{k+1}}^2_{\O} \implies e_{k+1}^2 = e_k^2 - E_k^2.
  \]
  Introduce some constants $\gamma > 0$ and $0 < \beta < 1$ to be selected later:
  \[
    e_{k+1}^2 + \gamma \osc_{k+1}^2 = e_k^2 - E_k^2 + \gamma \osc_{k+1}^2 \leq e_k^2 - \beta E_k^2 + \gamma \osc_{k+1}^2.
  \]
  Invoking the oscillation reduction from Lemma \ref{lem:oscasum} property results in
  \begin{equation}
    \label{eq:ineq1}
  \begin{aligned}
    e_{k+1}^2 + \gamma \osc^2_{k+1} \leq e_k^2 - \beta E_k^2 + \gamma \osc_k^2 - \gamma\lambda \osc_k^2(\RR_k).
      %e_{k+1}^2 + \gamma \osc^2_{k+1} \leq e_k^2 
    %e_{k+1}^2 + \gamma \osc^2_{k+1} &= e_k^2 - E_k^2 + \gamma \osc^2_{k+1} \\
    %&\leq e_k^2 - E_k^2 + \gamma \left[\osc_k^2 - \lambda  \osc^2_k(\RR_k) \right].
  \end{aligned}
\end{equation}
The discrete lower bound \ref{eq:loclower} from Lemma \ref{lem:assumptions} reads as
  \[
    E_k^2 = \enorm{U_{k+1} -  U_k}_\O \geq C_3 \eta_k^2(\wt \RR_k) - \osc_k^2(\RR_k) \geq C_3 \eta_k^2(\mathcal{M}_k) - \osc_k^2(\RR_k),
  \]
  where the last inequality follows since all marked patches are entirely refined when going from $\T_k$ to $\T_{k+1}$,
  and thus $\M_k \subset \wt \RR_k$.
  By inserting this into inequality \eqref{eq:ineq1} we obtain
  \begin{align*}
    e_{k+1}^2 + \gamma \osc^2_{k+1} &\leq e_k^2 - \beta C_3 \eta_k^2(\M_k) - \beta \osc_k^2(\RR_k) + \gamma \osc_k^2 - \gamma\lambda  \osc^2_k(\RR_k) \\
  &= e_k^2 + \gamma \osc_k^2 - \beta C_3\eta_k^2(\M_k) - \left[\lambda \gamma - \beta\right]\osc^2_k(\RR_k).
  \end{align*}
  Under the assumption that $\lambda \gamma \geq \beta$ we may replace $\osc_k^2(\RR_k)$ by the smaller $\osc_k^2(\mathcal{M}_k)$:
  \[
    e_{k+1}^2 + \gamma \osc^2_{k+1} \leq e_k^2 + \gamma \osc_k^2 - C_3 \beta \eta_k^2(\M_k) - \left[\lambda \gamma - \beta\right]\osc^2_k(\M_k).
  \]
  Now pick $\beta$ such that  the coefficients of $\osc_k^2(\M_k)$ and $\eta_k^2(\M_k)$ match, i.e.
  \[
    \beta C_3 = \lambda \gamma - \beta \implies \beta := \frac{1 }{1 + C_3}\lambda \gamma,
  \]
  which ensures that $\lambda \gamma \geq \beta$. Replace $\beta$ and use the definition $\vartheta_k^2 = \eta_k^2 + \osc_k^2$ to attain
  \begin{align*}
  e_{k+1}^2 + \gamma \osc^2_{k+1} &\leq e_k^2+\gamma \osc_k^2 - \frac{C_3}{1+C_3} \lambda \gamma \vartheta_k^2(\M_k) \\
    &\leq e_k^2 + \gamma \osc_k^2 - \frac{C_3}{1 + C_3} \lambda \gamma \theta^2 \vartheta_k^2,
    %- \left[\lambda \gamma - 1 - C_3\right]\osc^2_k(\M_k) \\
    % &\leq e_k^2 + \gamma \osc_k^2 - C_3 \vartheta_k^2(\M_k)\\
    % &\leq e_k^2 + \gamma \osc_k^2 - C_3 \theta^2 \vartheta_k^2,
  \end{align*}
  with the last step following from the D\"orfler marking property, i.e. $\vartheta_k^2(\mathcal{M}_k) \geq \theta^2 \vartheta_k^2$. 
  Since the total error dominates oscillation, $\vartheta_k^2 \geq \osc_k^2$, we infer that
  \[
    e_{k+1}^2 + \gamma \osc_{k+1}^2 \leq e_k^2 + \gamma \osc_k^2 - \frac{C_3}{2(1 + C_3)} \lambda \gamma \theta^2\left(\vartheta_k^2 + \osc_k^2\right).
  \]
  Rewriting the (global) reliability bound \ref{eq:globupper} shows $\vartheta_k^2 \geq 1/C_1 e_k^2$, and thus
  \begin{align*}
    e_{k+1}^2 + \gamma \osc_{k+1}^2 &\leq \left(1 - \frac{C_3 \lambda \theta^2}{2C_1(1+C_3)}\gamma\right) e_k^2 + \gamma \left( 1 - \frac{C_3}{2(1 + C_3)}\lambda\theta^2\right)\osc_k^2\\
    &:= \alpha^2_1(\gamma) e_k^2 + \gamma \alpha_2^2 \osc_k^2.
  \end{align*}
  Since $\lambda$ and $\theta^2$ are both contained in the interval $(0,1)$ we have $0 < \alpha^2_2 < 1$. We are left to
  pick $\gamma$ such that $0 < \alpha_1^2(\gamma) < 1$, which translates into
  \[
    0 < \gamma < \frac{2C_1(1 + C_3)}{C_3\lambda \theta^2}.
  \]
  Any $\gamma$ satisfying the above completes the proof, since $\alpha^2 = \max\set{\alpha_1^2(\gamma), \alpha_2^2} < 1$ with
  \[
    e_{k+1}^2 + \gamma \osc_{k+1}^2 \leq \alpha_1^2(\gamma)e_k^2 + \gamma \alpha_2^2\osc_k^2 \leq \alpha^2 \left( e_k^2 + \gamma \osc_k^2\right).
  \]

  \iffalse
  A piecewise polynomial on $\T_{k}$ is also a piecewise polynomial on $\T_{k+1}$, since $\T_{k+1} \geq \T_k$.
  Therefore we have $U_{k+1} - U_{k} \in \VV(\T_{k+1})$; combined with Galerkin orthogonality yields,
  \[
    \enorm{u - U_k}_{\O}^2 = \enorm{u - U_{k+1}}^2_{\O} + \enorm{U_{k+1} - U_{k}}^2_{\O} \implies e_{k+1}^2 = e_k^2 - E_k^2.
  \]
  Introducing constants $\gamma > 0, \delta > 0$ and $0 < \beta < 1$, one easily sees that
  \[
    e_{k+1}^2 + \gamma \osc_{k+1}^2 = e_k^2 - E_k^2 + \gamma \osc_{k+1}^2 \leq e_k^2 - \beta E_k^2 + (1 + \delta)\gamma \osc_{k+1}^2.
  \]
  Invoking the oscillation reduction from Lemma \ref{lem:oscasum} property results in,
  \begin{equation}
    \label{eq:ineq2}
    e_{k+1}^2 + \gamma \osc^2_{k+1} \leq e_k^2 - \beta E_k^2 + (1 + \delta)\gamma\left[ \osc^2_k - \lambda \osc^2_k(\RR_k)\right].
  \end{equation}
  The discrete lower bound from Lemma \ref{lem:assumptions} read as
  \[
    \enorm{U_{k+1} -  U_k}_\O \geq C_3 \eta_k^2(\RR_k) - \osc_k^2(\RR_k) \geq C_3 \eta_k^2(\mathcal{M}_k) - \osc_k^2(\RR_k),
  \]
  where the last inequality follows since $\mathcal{M}_k \subset \RR_k$.
  Inserting this into inequality \eqref{eq:ineq2} results in,
  \begin{align*}
    e_{k+1} + \gamma \osc^2_{k+1} &\leq e_k^2 - \beta \left[C_3 \eta_k^2(\mathcal{M}_k) - \osc_k^2(\RR_k)\right] +  (1 + \delta)\gamma\left[ \osc^2_k - \lambda \osc^2_k(\RR_k)\right] \\
    &\leq e_k^2 + (1 + \delta)\gamma \osc_k^2 - \beta C_3 \eta_k^2(\M_k) - \left[(1+\delta)\lambda \gamma - \beta\right]\osc^2_k(\RR_k).
  \end{align*}
  Let the  constants satisfy $(1 + \delta) \lambda \gamma\geq \beta$, 
  then we may replace $\osc_k^2(\RR_k)$ by the smaller $\osc_k^2(\mathcal{M}_k)$.
  Pick $\beta$ such that the coefficients of $\osc_k^2(\mathcal{M}_k)$ and $\eta_k^2(\M_k)$ match, i.e.
  \[
    (1 + \delta)\lambda \gamma - \beta = \beta C_3 \implies \beta = \frac{1}{1 + C_3} (1+\delta)\lambda \gamma
  \]
  then the inequality becomes
  \begin{align*}
    e_{k+1} + \gamma \osc^2_{k+1} \leq e_k^2 + (1 + \delta)\gamma \osc_k^2 - \frac{C_3}{1 + C_3} \gamma \lambda (1 + \delta) \left[\eta_k^2(\M_k) + \osc_k^2(\M_k)\right].
  \end{align*}
  This last term is per definition equal to $\vartheta_k^2(\M_k)$.
  By the D\"orfler marking property we have $\vartheta_k^2(\mathcal{M}_k) \geq \theta^2 \vartheta_k^2$, so that the above simplifies to
  \[
    %e_{k+1} + \gamma \osc^2_{k+1} \leq e_k^2 + (1 + \delta)\gamma \osc_k^2 - \frac{C_3}{1 + C_3} \gamma \lambda (1 + \delta)\theta^2\vartheta_k^2.
  \]
  Invoke the reliability bound from \ref{lem:assumptions} to reduce this to,
  \begin{align*}
 e_{k+1} + \gamma \osc^2_{k+1} &\leq (1 - C_1^{-1} C_3 \theta^2)e_k^2 + \gamma \osc^2_k - (\lambda \gamma - 1 - C_3) \osc_k^2(\mathcal{M}_k).
  \end{align*}
  \textcolor{blue}{
    Can we proceed? Does not seem like it.
  }
  \fi
\end{proof}


\section{Optimality AFEM}
We will show that convergence of AFEM driven by the equilibrated estimator is actually optimal.
From the reliability and efficiency of the estimator we find that
\[
  \enorm{u - U}^2_\O + \osc^2_{\T}(f, \W) \approx \eta_\T^2(U, \W) + \osc_\T^2(f, \W) = \vartheta_\T^2(U, \W).
\]
The total error estimator is equivalent to the approximation error (up to an oscillation term).
Since we cannot get rid of the oscillation term (cf. \cite[Rem~5.1]{cascon2008}), it makes sense to incorporate this into the
definition of an approximation class.

Denote $\TT$ for the set of all conforming refinements of $\T_0$ found by applying bisection. Moreover, let $\TT_N \subset \TT$ be the restriction of
triangulations with at most $N$ triangles more than $\T_0$, i.e. $\T \in \TT_N$ if $\T \leq \#T_0 + N$.
We define the optimal approximation class in terms of $s > 0$. For $v \in H_0^1(\O)$ with $\Delta v \in L^2(\O)$ let
$V_\T \in \VV(\T)$ denote the Ritz-Galerkin approximation of the Poisson problem associated with $v$.
A measure for the optimal convergence rate of $v$ is given by
\[
  \abs{v}_{\A_s} := \sup_{N > 0}N^s \inf_{\T \in \TT_N} \left(\enorm{v - V_\T}_{\O}^2 + \osc_\T^2(f, \W)\right)^{1/2}.
\]
As approximation class we take all the functions for which this measure is finite, i.e.
\[
  \A_s := \set{ v \in H_0^1(\O) : \Delta v \in L^2(\O), \quad  \abs{v}_{\A_s} < \infty}.
\]
So if the Poisson solution $u$  satisfies $u \in \A_s$, then the total 
error $e(N)$ of the discrete solution $U \in \VV(\T)$ on the \emph{best} partition $\T \in \TT_N$ satisfies
\begin{equation}
  \label{eq:errorafem}
  e(N) := \left(\enorm{u - U}^2_\O + \osc^2_{\T}(f,\W)\right)^{1/2} \leq N^{-s} \abs{u}_{\A_s}.
\end{equation}
Our goal is to proof that the sequence $\left(U_k\right)_{k \geq 0}$ satisfies a similar convergence rate.

\textcolor{blue}{
  Notice that this optimality class explicitly depends on our definition of oscillation. This makes
  it tougher to compare optimally results among different estimators. Possibly show equivalence of
our oscillation term with a standard one?}

The proof of this optimally is very similar to the standard case in \S\ref{sec:afem}. 
We opt to follow the steps given by Casc\'on and Nochetto in \cite{cascon2012}.
Define the maximum marking parameter by
\begin{equation}
  \label{eq:theta}
  \theta_*^2 := \frac{C_2}{(1+ C_1)(1 + C_2)}.
\end{equation}
\begin{lem}
  Let $\T \in \TT$ a triangulation with discrete solution $U \in \VV(\T)$, a marking parameter $\theta \in (0, \theta_*)$ and 
  a refinement $\T_\star \geq \T \in \TT$ with discrete solution $U_\star \in \VV(\T_\star)$ that satisfies
  \[
    \enorm{ u - U_\star}_\O^2+ \osc^2_{\T_\star}(f, \W_\star) \leq \mu (\enorm{u - U}_\O^2 + \osc_\T^2(f, \W)),
  \]
  for $\mu := 1 - {\theta^2 \over \theta_*^2}$.

  Then the refined set $\RR = \RR_{\T \to \T_\star}$ satisfies the D\"orfler property
  \[
    \vartheta_\T(U, \RR) \geq \theta \vartheta_\T(U, \W).
  \]
\end{lem}
\begin{proof}
  From the efficiency bound \eqref{eq:globlower} in Lemma \ref{lem:assumptions} we deduce
  \[
    C_2 \vartheta_\T^2(U, \W) \leq \enorm{u - U}_\O^2 + (1+C_2)\osc_\T^2(f, \W) \leq (1 + C_2)\left(\enorm{u - U}_\O + \osc_\T^2(f, \W)\right),
  \]
  and thus
  \[
    {C_2 \over 1 + C_2} \vartheta_\T^2(U, \W) \leq \enorm{u - U}_\O^2 + \osc_\T^2(f,\W).
  \]
  Together with the assumption this becomes
  \begin{equation}
    \label{eq:maineq}
  \begin{aligned}
    (1 - \mu){C_2 \over 1 + C_2} \vartheta_\T^2(U, \W) &\leq (1 - \mu) \left[ \enorm{u - U}_\O^2 + \osc_\T^2(f, \W)\right]\\
    &\leq \enorm{u - U}_\O^2 + \osc_\T^2(f,\W) - \enorm{u - U_\star}_\O^2 - \osc_{\T_\star}^2(f, \W_\star).
  \end{aligned}
\end{equation}
  The error and oscillation term satisfy
  \begin{align*}
    \enorm{U_\star - U}^2_\O &= \enorm{u - U}^2_\O - \enorm{u - U_\star}^2_\O,\\
    \osc_\T^2(f, \W) &= \osc_{\T_\star}^2(f, \W \cap \W_\star) + \osc_{\T_\star}^2(f, \RR) \leq \osc_{\T_\star}^2(f, \W_\star) + \osc^2_{\T}(f, \RR).
  \end{align*}
  From inserting these into  \eqref{eq:maineq} we infer that
  \begin{align*}
    (1 - \mu){C_2 \over 1 + C_2} \vartheta_\T^2(U, \W) &\leq \enorm{U_\star - U}_\O^2 + \osc_{\T}(f, \RR) \\
    &\leq \enorm{U_\star - U}_\O^2 + \vartheta_\T(U, \RR) \\
    &\leq (1 + C_1) \vartheta_\T^2(U, \RR),
  \end{align*}
  where the last step follows from the localized upper bound \eqref{eq:locupper}.
  In conclusion, by definition of $\theta_*^2$ and $\mu$ we see that
  \[
    \vartheta_\T^2(U, \RR) \geq  (1 - \mu){C_2 \over (1 + C_1)(1 + C_2)}\vartheta_\T^2(U, \W) \geq \theta^2 \vartheta_\T^2(U, \W). 
  \]
\end{proof}

\begin{cor}
  Let $\set{\T_k, \VV_k, U_k}_{k \geq 0}$ be the sequence of results produced by AFEM for a marking parameter $\theta \in (0, \theta_*)$.
  If $u \in \A_s$ then the minimal D\"orfler set $\M_k$ satisfies
  \[
    \M_k \lesssim \left(1 - \frac{\theta^2}{\theta^2_*}\right)^{-1 / 2s} \abs{u}_{\A_s}^{1/s} \left(\enorm{u - U_k}_\O^2 + \osc^2_{\T_k}(f, \W_k)\right)^{-1/2s}.
  \]
\end{cor}
\begin{proof}
  Let $\mu$ be as in the previous Lemma and set $\epsilon^2 := \mu \left( \enorm{u - U_k}_\O^2 + \osc_{\T_k}^2(f, \W_k)\right)$.
  Since $u \in \A_s$, by definition of the approximation class there exists a triangulation $\T_\epsilon \in \TT$ with
  discrete solution $U_\epsilon$ such that\footnote{To see this, let $e(N)$ denote the smallest error on a triangulation from $\TT_N$. This function is decreasing, since $\TT_N \subset \TT_{N+1}$, and we have $e(N) \to 0$ as $N \to \infty$. Therefore, there exists a $N$ such that $\epsilon \leq e(N)$ and thus $\epsilon N^s \leq e(N) N^s \leq \abs{u}_{\A_s}$.}
  \[
    \# \T_\epsilon - \# \T_0 \leq \abs{u}_{\A_s}^{1/s} \epsilon^{-1/s}, \quad \text{and } \enorm{u - U_\epsilon}_\O^2 + \osc^2_{\T_\epsilon}(f, \W_{\epsilon}) \leq \epsilon^2.
  \]
  For $\T_\star = \T_k \oplus \T_\epsilon$ --- the smallest common refinement of $\T_k$ and $\T_\epsilon$ ---  
  we have\footnote{This follows from Galerkin orthogonality and the oscillation reduction property.}
  \begin{align*}
    \enorm{u - U_\star}_\O^2 + \osc^2_{\T_\star}(f, \W_\star) \leq \enorm{u - U_\epsilon}_\O^2 + \osc^2_{\T_\epsilon}(f,\W_\epsilon) \leq \epsilon^2.
  \end{align*}
  By choice of $\epsilon^2$ it follows that $u - U_\star$ satisfies the conditions of the previous lemma, and thus that $\RR_{\T_k \to \T_\star}$ satisfies the D\"orfler property. Since $\mathcal{M}_k$ is the \emph{minimal} cardinality set that satisfies
  the D\"orfler property we may conclude that $\# \mathcal{M}_k \leq \# \RR_{\T_K \to \T_\star}$.
  
  Now $\RR_{\T_k \to \T_\star}$ consists of all those patches containing a refined triangle going from $\T_k$ to $\T_\star$. Since
  every patch has a uniformly bounded number of triangles, with a bound only depending on $\T_0$, we conclude
  \[
    \# \mathcal{M}_k \leq \RR_{\T_k \to \T_\star} \lesssim \# \T_\star - \# T_k \leq \# \T_\epsilon - \# \T_0 \leq \abs{u}_{\A_s}^{1/s}\epsilon^{-1/s}.
  \]
  The third inequality is a property of the smallest common refinement \cite[Lem~3.7]{cascon2008}.
\end{proof}
We are almost ready to prove the main optimality result. For this we need to bound the number of refined triangles. This number can inflate in
one iteration, but the cumulative sum behaves correctly. We have
\begin{equation}
  \label{eq:nvbound}
  \# \T_k - \# \T_0 \lesssim \sum_{i = 0}^{k-1} \# \mathcal{M}_i.
\end{equation}
For a proof in the standard case, see \cite{ste08}. In the classical case $\mathcal{\hat M}_i$ consists of the marked \emph{triangles}.
We consider marked \emph{patches} $\mathcal{M}_i$, but these are equivalent in terms of cardinality --- $\#\mathcal{\hat M}_i \approx \# \mathcal{M}_i$ --- so the above result also holds in our case.
We are finally ready to state the main theorem.
\begin{thm}
  Suppose that the marking parameter $\theta$ satisfies $\theta \in (0, \theta_*)$ (see \eqref{eq:theta}),
  and suppose that $u \in \A_s$ for some $s >0$.

  Then the produced sequence ${U_k, \VV_k \T_k}_{k \geq 0}$ of AFEM-solutions satisfies 
  \[
    \left(\enorm{u - U_k}^2_\O + \osc^2_k(f,\W_k)\right)^{1/2} \lesssim \abs{u}_{\A_s} \left(\#\T_k - \#\T_0\right)^{-s},
  \]
  for a constant only depending on $\T_0$ and thus independent of $s$.
\end{thm}
\begin{proof}
Combine the contraction property, the previous corollary and inequality \eqref{eq:nvbound}:
\begin{align*}
  \# \T_k - \# \T_0 &\lesssim \sum_{i = 0}^{k-1} \# \mathcal{M}_i \\
  &\lesssim \left(1 - \frac{\theta^2}{\theta^2_*}\right)^{-1 / 2s} \abs{u}_{\A_s}^{1/s} \sum_{i=0}^{k-1} \left(\enorm{u - U_i}_\O^2 + \osc^2_{\T_i}(f, \W_i)\right)^{-1/2s}\\
  &\approx \left(1 - \frac{\theta^2}{\theta^2_*}\right)^{-1 / 2s} \abs{u}_{\A_s}^{1/s} \sum_{i=0}^{k-1} \left(\enorm{u - U_i}_\O^2 + \gamma\osc^2_{\T_i}(f, \W_i)\right)^{-1/2s}\\
  &\leq \left(1 - \frac{\theta^2}{\theta^2_*}\right)^{-1 / 2s} \abs{u}_{\A_s}^{1/s} \left(\sum_{i=1}^{k} \alpha^{i \over s}\right) \left(\enorm{u - U_k}_\O^2 + \gamma\osc^2_{\T_k}(f, \W_k)\right)^{-1/2s}\\
  &\approx \abs{u}_{\A_s}^{1/s}\left(\enorm{u - U_k}_\O^2 + \osc^2_{\T_k}(f, \W_k)\right)^{-1/2s}.
\end{align*}
Here we used that $\alpha < 1$, so that it forms a geometric series.
 Standard algebra calculations on this inequality provide us with the wanted result.
\end{proof}
\begin{rem}
  Compare this result to the definition of the approximation class $\A_s$.
  The best possible error $e(N)$ on $\TT_N$ satisfies $e(N) \leq N^{-s}\abs{u}_{\A_s}$  \eqref{eq:errorafem}.
  By the previous theorem exists a constant $C_{opt}$ such that the error made by AFEM is bound by
  \[
    \left(\enorm{u - U_k}^2_\O + \osc^2_k(f, \W_k)\right)^{1/2} \leq C_{opt} \left(\#\T_k - \#\T_0\right)^{-s} \abs{u}_{\A_s}.
  \]
  Hence, the AFEM provided solutions have the same optimal convergence rate.
\end{rem}
{  \color{blue}
  Can we circumvent the equivalence with the standard residual estimator?

  The above AFEM works by marking elements according to the total error $\vartheta$. Can we achieve
  optimality as well if we mark only for the estimator $\eta$? I think this is possible, given that we have something like
  \[
    \osc_\T^2(f,\w_a) \leq \eta_\T^2(U, \w_a).
  \]

  This is more or less the optimally proof. My biggest objection with everything is that this AFEM method assumes
  we mark the elements according tot $\norm{\v{\sigma_a}}_{\w_a}$. Whereas the constant-free
  estimator is given by $\norm{\sum \v{\sigma_a}}_{K}$ for triangles $K \in \T$.
  Is the above easily adaptable to use with the second estimator instead?
}

\end{document}
