\documentclass[thesis.tex]{subfiles}
\begin{document}
In the previous chapter the classical (or standard) residual error estimator is introduced as the driver for the
adaptive finite element method (AFEM). This results in an optimal algorithm, i.e. the adaptive meshes
generated by this method provide the highest possible convergence rate. Unfortunately, this asymptotic result
is a bit unfortunate for practical purposes due to the unknown constants. In practice
one would like to know \emph{when} to stop iterating, i.e. when the approximation error is small enough --- say
below some threshold. 

Another problem of the classical residual error estimator is that it is not polynomial-degree-robust; 
the constants depend on $p$, the degree of the finite element solutions. This becomes an issue
when considering \emph{hp}-AFEM, a version of AFEM where the polynomial degree can also vary
per simplex. For \emph{hp}-AFEM one would like an estimator with a constant independent of the polynomial degree
used in the finite element space.

Luckily, there are error estimators that suffer less from these constants problems. Recently quite
some research interesting has been shown for such constant-free estimators. One of these estimators is the
\emph{equilibrated residual error estimator} also called the \emph{Braess-Sch\"oberl estimator} \cite{braessequil, braessequilrobust,ernequil}.
In this chapter we will introduce this estimator alongside some results, and prove that AFEM driven 
by this estimator is optimal.

For simplicity we (again) restrict ourself to the Poisson problem \eqref{eq:poisson} on a two-dimensional domain $\O \subset \R^2$,
and for some conforming triangulation $\T_h$ we consider $\VV(\T_h)$: the (Lagrange) finite element space of degree $p$. \todo{p of p+1}
The Galerkin approximation (discrete solution) is denoted by $U_h \in \VV(\T_h)$. If we do not need to stress dependence on the triangulation, we 
also write $V$ for this finite element space.
\section{Prager and Synge}
The so-called equilibrated residual estimators are based on the fundamental theorem of Prager and Synge \cite{prager}. 
For this we require the space $H(div; \Omega)$ as defined in \ref{def:hdiv}.
\begin{thm}[Prager and Synge]
  Let $u \in H_0^1(\O)$ be the exact solution of the Poisson problem. 
  For a {flux} $\sigma \in H(\div; \O)$ satisfying the  equilibrium condition $\div \sigma + f = 0$ pointwise\todo{Is distributional also correct?},
it holds
\[
  \norm{\nabla u - \nabla v}^2_{L^2(\O)} + \norm{\nabla u - \sigma}^2_{L^2(\O)} = \norm{\nabla v - \sigma}^2_{L^2(\O)} \quad \forall v \in H_0^1(\O).
\]
\end{thm}
\begin{proof}
   Applying the divergence theorem yields,
  \begin{align*}
    &\int_\O  \left(\nabla u - \sigma\right) \cdot \nabla \left( u -  v\right)  \, \dif x \\ 
    =  - &\int_\O \left(u - v\right) \,  \nabla \cdot \left (\nabla u - \sigma\right) \dif x + \int_{\partial \O} (u - v) \left( \nabla u \cdot n - \sigma \cdot n\right) \dif s  = 0
  \end{align*}
  since from the assumptions  $\Delta u = -f = \div \sigma$ in $\O$, and $u - v = 0$ on $\partial \O$.
  From this orthogonality relation and Pythagoras we may conclude that
  \[
    \norm{\nabla(u-v)}^2_{L^2(\O)} + \norm{\nabla u - \sigma}^2_{L^2(\O)} = \norm{ -\nabla (u-v) + \nabla u - \sigma}^2_{L^2(\O)},
  \]
  which equals the asserted.
\end{proof}
For a flux $\sigma$ satisfying the equilibrium condition we obtain an 
\emph{reliable} constant-free estimator by replacing $v$ with the discrete solution $U_h$ in the previous theorem,
\begin{equation}
  \label{eq:synest}
  \enorm{u - U_h}_{\O}^2 =  \norm{\nabla u - \nabla U_h }^2_{L^2(\O)} \leq \norm{\nabla U_h - \sigma}^2_{L^2(\O)}.
\end{equation}

The question now arises how to construct $\sigma$. In general, one would like an estimator to be proportional the error.
For this, we need the estimator to also be \emph{efficient}: it should provide a lower bound for $\enorm{u -U_h}_{\O}$, up to
a constant and an oscillation term.

Suppose that $f$ is constant on each element in $\mathcal{T}$, then suitable fluxes $\sigma$ can be 
found in the lowest order Raviart-Thomas space $\RT_1(\T)$ --- properly defined in \S \todo{Introduce, alongside some
basic characterisations?}.
From \eqref{eq:synest} we then see that 
the best estimator in the Raviart-Thomas space is found by minimizing $\norm{\nabla U_h - \sigma}$ for all 
fluxes $\sigma \in \RT_1(\T_h)$ that are in equilibrium. However,
this global minimization procedure --- equivalent to the mixed finite element solution --- 
is too expensive for computation of an error estimate.

To overcome this problem, Braess and Sch\"oberl \cite{braessequil} propose minimizing local problems instead; a procedure
called \emph{equilibration}.

\section{Equilibration} 
Instead of constructing $\sigma$ the difference $\sigma^\triangle := \sigma - \nabla U_h$ is considered.
For the moment we let  $f$ be piece-wise constant as this avoids the effect of data oscillation.
The flux $\sigma$ will be constructed as an element in the 
Raviart-Thomas space $\RT_p(\T_h)$. The correction $\sigma^\triangle$ will therefore belong to the broken Raviart-Thomas space $\RT_p^{-1}(\T_h)$.


Following \cite{ernequil} we denote $\V_h, \E_h$ for the set of vertices and edges in the triangulation~$\T_h$. 
We add the superscripts  $^{int}$ or $^{bdr}$ to indicate restrictions of $\V_h, \E_h$ to the interior or the boundary.
For each vertex $a \in \V_h$ we write $\psi_a$ for the hat function at vertex $a$, 
i.e. the unique function in the linear finite element space
that takes value $1$ at $a$ and vanishes at the other vertices.
These hat functions provide us with a partition of unity: $\sum_{a \in \V_h} \psi_a \equiv 1$.
The local problems are solved on patches $\omega_a$, the star at a vertex $a \in \V_h$, also the support of
the hat function $\psi_a$:
\[
  \omega_a := \omega\left( \T_h, a\right) = \supp{\psi_a} = \bigcup \set{ K \in \T_h : a \in \partial T}.
\]
We will denote $\gamma_a$ for the interior edges of $\w_a$, i.e.
\[
  \gamma_a = \E_h \cap \text{int}(\w_a).
\]
The total difference flux $\sigma^\triangle$ is then the sum of local fluxes over these patches, 
\[
  \sigma^\triangle := \sum_{a \in \V_h} \sigma_{\omega_a}.
\]

As proposed  in \cite{braessequilrobust}, local fluxes $\sigma_a$ are found by decomposing the residual using the partition of unity. 
More precisely, let $v \in H_0^1(\O)$ be given and write $r$ for the usual residual, \todo{Define usual residual}
\[
  \ip{r,v} := a(u - U_h, v) = \ip{\nabla \left(u - U_h\right), \nabla v}_\O = \ip{f,v}_\O - \ip{\nabla U_h,\nabla v}_\O.
\]
For $a \in V_h$ we define the local residual by
\[
  \ip{r_a,v} := \ip{r,\psi_a v} = a(u - U_h, \psi_a v)_{\w_a}.
\]
The local flux $\sigma_a$ is taken from the broken Raviart-Thomas space $\RT_{p}^{-1}(\omega_a)$ such that
\[
\div \sigma_a = r_a,
\]
holds in distributional sense. Since we consider the \emph{weak} derivative, the latter
statement is equivalent to
\[
  \ip{r_a, v} = \ip{\div \sigma_a, v}_{\w_a} = - \ip{\sigma_a, \nabla v}_{\w_a} \quad \forall v \in H^1(\w_a) \cap H_0^1(\O).
\]
The residual can be expressed in triangle and edge related terms by applying the divergence theorem.
This leads to the following decomposition for the local residual:
\begin{equation}
  \label{eq:locdecomp}
  \ip{r_a,v} = \sum_{K \in \T_h : K\subset \w_a} \langle{r^T_a, v}\rangle_K + \sum_{e \in \E_h: e \subset \gamma_a} \ip{r^e_a, v}_e,
\end{equation}
where 
\[
  r^T_a := \psi_a \left[ f + \Delta U_h \right], \quad r^e_a := \psi_a \llbracket \nabla U_h \rrbracket.
\]
Similarly we can rewrite $\ip{\sigma_a, \nabla v}_{\w_a}$ in triangle and edge related terms using the divergence theorem.
After expanding, we see that $\div \sigma_a = r_a$ holds if and only if,
\begin{alignat}{3}
  \div \sigma_a &= \psi_a \left (f + \Delta U_h\right) && \quad\text{in }  K\subset \w_a \label{eq:sigmacons}\\
  \llbracket \sigma_a \rrbracket &= \psi_a \llbracket \nabla U_h \rrbracket && \quad \text{on } e \subset \gamma_a\nonumber\\
  \sigma_a \cdot n &= 0 &&\quad \text{on } \partial \w_a\nonumber
\end{alignat}
Since there is no unique solution for the above system, a solution $\sigma_a \in \RT_p^{-1}(\T_h)$
with \emph{minimal} $L_2$-norm is chosen as the equilibrated local flux.

Per construction and by the partition of unity we have for $v \in H_0^1(\O)$,
\[
  \langle \div \sigma^\triangle,v\rangle_{\O} = \sum_{a \in \V_h} \ip{\div \sigma_a,v}_\O = \sum_{a \in \V_h} \ip{r_a,v} = \langle r, \sum_{a \in \V_h} \psi_a v \rangle  = \ip{r, v}.
\]
Set $\sigma = \nabla U_h + \sigma^\triangle$, then from Green's identities we see that for $v \in V_h$,
\begin{align*}
  \ip{\div \sigma, v}_\O &=  \ip{\div \nabla U_h, v}_\O + \langle\div \sigma^\triangle, v\rangle_{\O}\\
  &= \ip{\Delta U_h, v}_\O + \ip{r,v}\\
  &= \sum_{K \in \T_h} \ip{\Delta U_h,v}_K + \ip{f,v}_K - \ip{\nabla U_h, \nabla v}_K\\
  &= \sum_{K \in \T_h} \ip{\Delta U_h + f,v}_K + \ip{\Delta U_h, v}_K - \ip{\nabla U_h \cdot n, v}_{\partial K}\\
\end{align*}
\textcolor{blue}{
  This is complete non-sense, it does not equal $f$ in distribution, let alone pointwise. How is this $\sigma$ eligible for Prager and Synge thm?  According to Braess et al. in \cite{braessequilrobust} it is. Annoying.
  Maybe the entire Prager and Synge theorem is not used? Does the flux satisfy the equilibrium condition states by Ern and Vohralik
  $\ip{\div \sigma, 1}_K = \ip{f, 1}_K$? Not sure.
}
\section{Residual bounds}
Using the relation between $\sigma_a$ and the local residual $r_a$ we can directly derive some useful bounds.
For an interior vertex $a \in \V_h^{int}$ the hat function $\psi_a$ belongs to the finite element space $V_h$. 
Using that $U_h$ is the Galerkin approximation therefore gives us
\[
  \ip{r_a, 1} = \ip{r, \psi_a} = \ip{f, \psi_a}_\O - \ip{\nabla U_h, \nabla \psi_a}_\O = 0.
\]
So the functional $r_a$ vanishes on constants, and therefore we may interpret $r_a$
as a functional on the mean zero space. Again in the notation of \cite{ernequil}, we define for $a \in \V_h$,
\[
  H^1_\star(\omega_a) := \left[\begin{aligned}
      &\set{v \in H^1(\omega_a):  \ip{v,1}_{\omega_a} =:v_{\omega_a}  = 0} &\quad a \in \V_h^{int},\\
    &\set{v \in H^1(\omega_a): v = 0 \text{ on }\partial \omega_a \cap \partial \O} &\quad a \in \V_h^{bdr}.
  \end{aligned}
\right.
\]
We employ this space with the norm $\norm{\nabla \cdot }_{\omega_a}$, which is a norm since 
\[
  \norm{\nabla v}_{\omega_a} = 0 \implies v = C_{st}, \quad v \in H_\star^1(\omega_a) \implies C_{st} = 0.
\]
Even stronger,
this norm is equivalent to the $H^1$-norm due to Poincar\'e-Friedrichs inequality (c.f. \S\ref{sec:aux}).
With these definitions we are able to proof local efficiency and global reliability in terms of $r_a$.
\begin{lem}
  \label{lem:loceff}
  For every vertex $a \in \V_h$ we have the following local efficiency on $\omega_a$,
  \[
    \norm{r_a}_{H_\star^1(\omega_a)'} \lesssim \enorm{ u - U_h}_{\omega_a}.
  \]
\end{lem}
\begin{proof}
  By definition and the Cauchy-Schwarz inequality we find
  \begin{align*}
    \norm{r_a}_{H_\star^1(\omega_a)'} &= \sup_{v \in H_\star^1(\omega_a) : \norm{\nabla v}_{\w_a} = 1} \abs{\ip{r_a, v}}\\
    &= \sup_{v \in H_\star^1(\omega_a) : \norm{\nabla v}_{\w_a} = 1} \abs{\ip{\nabla \left(u - U_h\right), \nabla \left(\psi_a v\right)}_\O} \\
    &\leq \norm{\nabla u - \nabla U_h}_{\omega_a} \sup_{v \in H_\star^1(\omega_a): \|\nabla v\|_{\w_a}=1} \norm{ \nabla \left (\psi_a v \right)}_{\omega_a}.
  \end{align*}
  In order to get rid of the $\psi_a$-term we first apply the product rule,
  \begin{align*}
    \norm{\nabla\left(\psi_a v\right)}_{\omega_a} &= \norm{v \nabla \psi_a + \psi_a \nabla v}_{\omega_a}\\
    &\leq \norm{v \nabla \psi_a}_{\omega_a} + \norm{\psi_a \nabla v}_{\omega_a} \\
    &\leq \norm{v}_{\omega_a} \norm{\nabla \psi_a}_{\infty, \omega_a} + \norm{\psi_a}_{\infty, \omega_a} \norm{\nabla v}_{\omega_a},
  \end{align*}
  where the last inequality follows since $\omega_a$ is a bounded set. For the $\psi_a$ terms we note that 
  $\norm{\psi_a}_{\infty, \omega_a} = 1$, whereas the vector norm $\|\nabla \psi_a\|_{\w_a}$ is constant and bounded on each triangle by
  $\rho_{K}$, and thus $\norm{\nabla \psi_a}_{\infty, \omega_a} \leq C h^{-1}_{\w_a}$ for some constant only depending on the 
  shape regularity of the triangulation and the maximum amount of triangles in a patch. 
  The $\norm{v}_{\omega_a}$-term can be estimated using Poincar\'e-Friedrich inequalities (cf \S\ref{sec:aux}):
  \[
    \norm{v}_{\omega_a} = \begin{cases}
      \norm{v - v_{\omega_a}} \leq C_{P,\omega_a} h_{\omega_a}\norm{\nabla v}_{\omega_a} & a \in \V_h^{int}, \\
      \norm{v} \leq C_{F,\omega_a} h_{\omega_a}\norm{\nabla v}_{\omega_a} & a \in V_h^{bdr},
    \end{cases}
  \]
  with the constants depending only on the shape regularity of the triangulation \todo{Reference??}.
  Combining these inequalities and using that $\norm{\nabla v}_{\w_a} = 1$ we arrive at the asserted.
\end{proof}
\begin{lem}
  \label{lem:globrel}
  A global error bound in terms of $r_a$ is given by,
  \[
    \enorm{u - U_h}_{\O}^2 \lesssim \sum_{a \in \V_h} \norm{r_a}^2_{H^1_\star(\omega_a)'}
  \]
\end{lem}
\begin{proof}
  Since $H_0^1(\O)$ is a Hilbert space with respect to $a(\cdot, \cdot) = \ip{\nabla \cdot, \nabla \cdot}_\O$, we find by
  duality
  \[
    \enorm{u - U_h}_{\O}^2 = \norm{\nabla u - \nabla U_h}_{\O}^2 = \sup_{ v \in H_0^1(\O): v \ne 0 } {\abs{a(u - U_h, v)} \over \norm{\nabla v}_{\O}} = \sup_{ v \in H_0^1(\O): v \ne 0 } {\abs{\ip{r, v}} \over \norm{\nabla v}_{\O} }.
  \]
  Rewriting the residual operator using the partition of unity yields for $v \in H_0^1(\O)$,
  \begin{align*}
    \ip{r , v} &= \sum_{a \in \V_h^{int}} \ip{r_a, v} + \sum_{a \in \V_h^{bdr}} \ip{r_a,v} \\
    &= \sum_{a \in \V_h^{int}} \ip{r_a, {v - v_{\omega_a}}} + \sum_{a \in \V_h^{bdr}} \ip{r_a, v} \\
    &\leq \sum_{a \in \V_h^{int}} \norm{r_a}_{H^1_\star(\omega_a)'} \norm{\nabla (v- v_{\omega_a})}_{\omega_a} 
    + \sum_{a \in \V_h^{bdr}} \norm{r_a}_{H^1_\star(\omega_a)'} \norm{\nabla v}_{\omega_a}\\
    &\lesssim \norm{\nabla v}_\O \sum_{a \in \V_h} \norm{r_a}_{H^1_\star(\omega_a)'}.
      \end{align*}
  The second equality follows from orthogonality on the constants; the first inequality follows since $v - v_{\omega_a}, v \in H_\star^1(\w_a)$ 
  for an interior resp. boundary vertex $a$; and $\sum_{a \in \V_h} \norm{\nabla v}_{\omega_a} \lesssim \norm{\nabla v}_{\O}$ since
  every triangle is contained in a uniformly bounded number of triangles. Combining these inequalities yields,
  \begin{align*}
    \norm{\nabla u - \nabla U_h}_{\O}^2 \leq \norm{\nabla v}_{\O}^{-1} \ip{r,v} \lesssim \norm{\nabla v}_{\O}^{-1} \norm{\nabla v}_{\O} \sum_{a \in \V_h} \norm{r_a}_{H^1_\star(\omega_a)'}.
  \end{align*}
  \textcolor{blue}{In particular, the constant can be (cheaply) calculated  given the initial triangulation. Are
  there tighter bounds, preferably without constants?}
\end{proof}

These bounds become relevant if they can be related to the (local) estimators~$\sigma_a$. Recall
that $\sigma_a$ was found as minimal norm function from $\RT_p^{-1}(\omega_a)$ satisfying $\div~\sigma_a~=~r_a$.
The fact that this vector function is of minimum norm enabled Braess et al. \cite{braessequilrobust} to prove the following powerful theorem.
\begin{thm}
  \label{thm:locresequiv}
  For $a \in \V_h$, let $\sigma_a$ be found as described above. 
  We have the bound $\norm{r_a}_{H_\star^1(\omega_a)'} \leq \norm{\sigma_a}_{\omega_a}$. But more importantly,
  a reverse bound also holds for a constant {not} depending on the polynomial-degree $p$ used in the finite element space $\VV(\T)$:
  \[
    \norm{\sigma_a}_{\omega_a} \lesssim \norm{r_a}_{H_\star^1(\omega)'}.
  \]
\end{thm}
\begin{proof}
  From the characteristic definition of $\sigma_a$ we see that for $v \in H_\star^1(\w_a)$ it holds
  \[
    \ip{r_a, v} = \ip{\div \sigma_a, v}_{\w_a} = - \ip{\sigma_a, \nabla v}_{\w_a}.
  \]
  This directly provides us with the first inequality, i.e.
  \[
    \norm{r_a}_{H_\star^1(\omega_a)'} = \sup_{v \in H_\star^1(\w_a) : \norm{\nabla v}_{\w_a} = 1} \abs{\ip{r_a, v}} = \sup_{v \in H_\star^1(\w_a) : \norm{\nabla v}_{\w_a}=1} \abs{\ip{\sigma_a, \nabla v}_{\w_a}} \leq \norm{\sigma_a}_{\w_a}.
  \]

  The proof of the second inequality is much more involved. A constructive proof is given in \cite[Theorem~7]{braessequilrobust}.
\end{proof}
\begin{rem}
  Combing the previous Theorem and Lemma's we see that $\sum_{a \in \V_h} \norm{\sigma_a}_{\w_a}$ provides a reliable and efficient estimator.
  When using this estimator in an adaptive finite element method, an obvious refine strategy would consist
  of refining those patches for which $\norm{\sigma_a}_{\w_a}$ is large. Later we will show that this is indeed
  an optimal choice. To prove this, we will relate this estimator the classical residual estimator.
  \\
  \textcolor{blue}{
    This sounds like --- and is provable --- an optimal strategy up to constant. However, there is still some 
    overlap in the patches causing constants to arrise. In view of lower constants, a more preferable method probably consists of refining the triangles $K$
    where $\|{\sigma^\triangle|_{K}}\| =\|{\sum_{a} \sigma_a|_{K}}\|$ is large. This should cancel the overlap in patches,
    and therefore provide better constants.
  }
\end{rem}
\begin{lem}
  \label{lem:starequiv}
  The estimator $\norm{\sigma_a}_{\w_a}$ is equivalent to the classical estimator for the patch~$\w_a$, 
  \[
    \norm{\sigma_a}_{\w_a} \approx h_{\w_a}\norm{f + \Delta U_h}_{\w_a} + h^{1/2}_{\w_a} \norm{\llbracket \nabla U_h  \rrbracket}_{\gamma_a}.
  \]
\end{lem}
\begin{proof}
  We start with the $\gtrsim$ relation.  Applying the previous Theorem \ref{thm:locresequiv} and using orthogonality on constants yields  
  \begin{align*}
    \norm{\sigma_a}_{\w_a} &\geq \norm{r_a}_{H^1_\star(\w_a)'} \\
    &\geq \sup_{v \in H^1_\star(\w_a) : v \ne 0} { \ip{r_a, v} \over \norm{\nabla v}_{\w_a}} \\
    &= \sup_{v \in H^1(\w_a) : v \not \equiv C_{st}} { \ip{r_a, v - v_{\w_a}} \over \norm{\nabla (v - v_{\w_a})}_{\w_a}} \\
          &= \sup_{v \in H^1(\w_a) : v \not \equiv C_{st}} { \ip{r_a, v} \over \norm{\nabla v}_{\w_a}}.
  \end{align*}
  Recall the decomposition of $r_a$ in triangle and edge terms \eqref{eq:locdecomp},
  \[
    \ip{r_a,v} = \sum_{K\subset \w_a} \langle{r^T_a, v}\rangle_K + \sum_{e \subset \gamma_a} \ip{r^e_a, v}_e,
  \]
  with 
  \[
   r^T_a = \psi_a \left[ f + \Delta U_h \right], \quad r^e_a = \psi_a \llbracket \nabla U_h \rrbracket.
  \]
  In particular $(r^T_a, r^e_a)$ are piecewise polynomials taken from the \todo{Define} broken spaces $\P_r^{-1}(\w_a)\times \P_r^{-1}(\gamma_a)$ for some degree $r$.

  Fix a triangle $K \subset \w_a$, and consider the space $H_0^1(K)$. Since a function $v \in H_0^1(K)$ vanishes on the boundary, 
  we can naturally extend it to a function $v \in H^1(\w_a)$ by setting $v \equiv 0$ on $\w_a \setminus K$. As $v$
  vanishes on every triangle except $K$, and all edges $e \subset \gamma_a$, we have $\ip{r_a, v} = \langle{r_a^T, v}\rangle_K$ and thus
  \[
         \sup_{v \in H^1(\w_a) : v \not \equiv C_{st}} { \ip{r_a, v} \over \norm{\nabla v}_{\w_a}} 
         \geq \sup_{v \in H_0^1(K) : v \ne 0} { \langle{r^T_a, v}\rangle_K \over \norm{\nabla v}_{K}}
         \gtrsim h_K \|{r^T_a}\|_K.
  \]
  The last inequality is a known result, since $r^T_a$ is a polynomial resticted to $K$ (cf. \cite[Ex~9.x.5]{brenner}).
  
  Next, fix an edge $e \subset \gamma_a$ and denote $K_1, K_2$ for the triangles that share this edge. Consider the following space of functions:
  \[
  V_e := \set{v \in H_0^1(K_1 \cup K_2) : \ip{v,P}_{K_1}= \ip{v,P}_{K_2} = 0\quad \forall P \in \P_r}.\]
  Again we can naturally extend these functions to live in $H^1(\w_a)$. Since $v \in V_e$ vanish on all edges except $e$, and
  is orthogonal to polynomials of degree $r$ on $K_1$ and $K_2$ we have
  \[
         \sup_{v \in H^1(\w_a) : v \not \equiv C_{st}} { \ip{r_a, v} \over \norm{\nabla v}_{\w_a}} 
         \geq \sup_{v \in V_e: v \ne 0} { \langle r^e_a, v \rangle_e \over \norm{\nabla v}_{K_1 \cup K_2}}
         \gtrsim h_e^{1/2} \norm{r^e_a}_e.
  \]
  The last inequality is again a known result, since $r^e_a$ is a polynomial resticted to $e$ (cf. \cite[Ex~9.x.7]{brenner}).

  Finally, summing the inequalities over $K \subset \w_a$ and $e \subset \gamma_a$ yields
  \begin{align*}
    \norm{r_a}_{H_\star^1(\w_a)'} &\gtrsim \sum_{K \subset \w_a} h_K \|{r^T_a}\|_K + \sum_{e \subset \gamma_a} h_e^{1/2} \norm{r^e_a}_e\\
    &\gtrsim h_{\w_a} \norm{\psi_a \left[ f + \Delta U_h \right]}_{\w_a} + h_{\w_a}^{1/2} \norm{ \psi_a \llbracket \nabla U_h \rrbracket}_{\gamma_a}.
  \end{align*}
  With the last equality following since the number of triangles in a patch is uniformly bounded.

  \textcolor{blue}{How can we get rid of $\psi_a$? Away from the boundary $\partial \w_a$ we know that $\psi_a > \epsilon$ so we can bound this term. Not sure how to find a bound on boundary `strip'.}


  Now the other inequality $\lesssim$. From the previous theorem we know that $\norm{\sigma_a}_{\w_a} \lesssim \norm{r_a}_{H_\star^1(\w_a)'}$. For $v \in H_\star^1(\w_a)$ decompose the local residual $r_a$ in element and edge terms,
  \begin{align*}
    \ip{r_a, v} &= \sum_{K \subset \w_a} \ip{\psi_a\left(f + \Delta U_h\right), v}_{K} + \sum_{e \subset \gamma_a} \ip{\psi_a \llbracket \nabla U_h \rrbracket, v}_e\\
    &\leq \sum_{K \subset \w_a} \norm{\psi_a}_{\infty,K} \norm{f + \Delta U_h}_{K}\norm{v}_{K} + \sum_{e \subset \gamma_a} \norm{\psi_a}_{\infty, e} \norm{ \llbracket \nabla U_h \rrbracket}_e \norm{v}_e\\
    &\leq \sum_{K \subset \w_a} \norm{f + \Delta U_h}_{K}\norm{v}_{K} + \sum_{e \subset \gamma_a} \norm{ \llbracket \nabla U_h \rrbracket}_e \norm{v}_e
  \end{align*}
  For the element terms we may use Poincar\'e-Friedrichs inequality (cf. \S\ref{sec:aux}) to find
  \[
    \sum_{K \subset \w_a} \norm{f + \Delta U_h}_{K} \norm{v}_K \leq \norm{v}_{\w_a} \sum_{K\subset \w_a} \norm{f + \Delta U_h}_{K} \leq 
    h_{\w_a} C_{FP} \norm{\nabla v}_{\w_a} \norm{f + \Delta U_h}_{\w_a}.
  \]
  \textcolor{blue}{Does not introduce the scaling term correctly? Does this constant depends on $\w_a$? Normally this would be avoided
  by considering a reference element, unfortunately I don't see how we can consider a reference \emph{patch}.
  Or does the constant only depend on the diameter and the shape regularity $k_{\T_h}$,
  i.e. $C_{FP}(\w_a) = \widetilde{C}_{FP}(k_{\T_h})\text{diam}(\w_a)$.}

  For the edge terms, let $e \subset \gamma_a$ be an edge with $K_e$ an adjoint triangle. Applying the trace theorem and transformation lemma
  gives us,
  \[ 
    \norm{v}_e \leq \norm{v}_{\partial K_e} \lesssim h_{K_e}^{-1/2} \norm{v}_{K_e} + h_{K_e}^{1/2} \norm{\nabla v}_{K_e}
  \]
  for a constant only depending on the shape regularity. 
  Summing over all edges then yields,
  \[
    \sum_{e \subset \gamma_a} \norm{v}_e \lesssim \sum_{e\subset \gamma_a}h_{K_e}^{-1/2} \norm{v}_{K_e} h_{K_e}^{1/2} \norm{\nabla v}_{K_e}
    \lesssim h_{\w_a}^{-1/2} \norm{v}_{\w_a} + h_{\w_a}^{1/2} \norm{\nabla v}_{\w_a} \lesssim h_{\w_a}^{1/2} \norm{\nabla v}_{\w_a}.
  \]
  Where we used that maximum number of triangles in a patch is universally bounded, and that $v \in H_\star^1(\w_a)$ to apply the
  Poincar\'e-Friedrichs inequality. For the edge terms we thus have
  \[
    \sum_{e \subset \gamma_a} \norm{ \llbracket \nabla U_h \rrbracket}_e \norm{v}_e \leq \norm{\llbracket \nabla U_h \rrbracket}_{\gamma_a} \sum_{e \subset \gamma_a} \norm{v}_e \lesssim h_{\w_a}^{1/2} \norm{\llbracket \nabla U_h \rrbracket}  \norm{\nabla v}_{\gamma_a}. 
  \]
  Combining the inequalities in the last three paragraphs yields the wanted upper bound.
\end{proof}
\section{Oscillation}
\textcolor{blue}{Should we introduce oscillation afterwards, or integrate it directly?
Dont we need to assume that $U$ is found solving the discrete solution for $f_{p}$ instead of $f_{ex}$?}

Thus far we have assumed the right hand side $f$ to be a piecewise constant function. In practice, one 
has a more general $f \in L^2(\O)$, and thus we should use alter the above method to replace $f$ with a computable function.
The general approach is replace the exact $f$ with a polynomial approximation. 
Denote $Q_K$ for the $L^2(K)$-orthogonal projector onto polynomials of degree $p$ on $K$. We will replace the (exact) local residuals~$r_a$ by  (cf. \cite[Thm~3.17]{ernequil})
\[
  \ip{\tilde r_a, v} := \sum_{K \subset \w_a} \ip{Q_K \left(\psi_a \left[f + \Delta U_h\right]\right), v}_K 
  +\sum_{e \subset \gamma_a} \ip{\psi_a \llbracket \nabla U_h\rrbracket, v}_e.
\]
Notice that there indeed exist $\sigma_a \in \RT_p^{-1}(\w_a)$ such that $\div \sigma_a = \tilde r_a(v)$.

Of course, this discretization comes at a price. For $a \in \V_h$ let  $\sigma_a\in\RT_p^{-1}(\w_a)$ be the minimal $L^2$-norm
flux satisfying $\div \sigma_a = \tilde r_a$. 
Then Lemma's \ref{lem:loceff}, \ref{lem:globrel} and Theorem \ref{thm:locresequiv} have
the following discretized counterpart. Here we notate $Q_a$ for the $L^2(\w_a)$-orthogonal projection on the
broken polynomial space $\P_p^{-1}(\w_a)$.
\begin{thm}
  \label{thm:discbounds}
  For $a \in \V_h$ with $\sigma_a$ found as described above we have
  \begin{align*}
    \norm{\tilde r_a}_{H_\star^1(\w_a)'} &\lesssim \enorm{u - U_h}_{\w_a} + h_{\w_a} \norm{(I - Q_a) (\psi_a f)}_{\w_a}, \\
    \enorm{u - U_h}_{\O}^2 &\lesssim \sum_{a\in \V_h}\norm{\tilde r_a}^2_{H^1_\star(\w_a)'} + h^2_{\w_a} \norm{(I - Q_a)(\psi_a f)}^2_{\w_a},
  \end{align*}
  and
  \[
    \norm{\sigma_a}_{\w_a} \lesssim \norm{\tilde r_a}_{H_\star^1(\w_a)'} \leq \norm{\sigma_a}_{\w_a}.
  \]
  for a constant only depending on the shape regularity of the triangulation.
\end{thm}
\begin{proof}
  \textcolor{blue}{Can we get rid of the $\psi_a$ in the oscilation term?}


  Recall the decomposition of $r_a(v)$ in terms of its triangle and edge terms. 
  Subtracting $r_a - \tilde r_a$, we directly see that the edge terms cancel. Moreover $Q_K(\psi_a \Delta U_h) = \psi_a \Delta U_h$,
  since the latter is a $p-1$ degree polynomial. We are left with the $\psi_a f$ term, i.e.
  \begin{align*}
    \norm{r_a - \tilde r_a}_{H^1_\star(\w_a)'} &= \sup_{v \in H_\star^1(\w_a) : v \ne 0} \frac{\ip{r_a,v} - \ip{\tilde r_a, v}}{\norm{\nabla v}_{\w_a}} \\
    &= \sup_{v \in H_\star^1(\w_a) : v \ne 0} \frac{ \sum_{K \subset \w_a} \ip{ \psi_a f - Q_K (\psi_a f), v}_K}{\norm{\nabla v}_{\w_a}}\\
    &= \sup_{v \in H_\star^1(\w_a) : v \ne 0} \frac{ \ip{(I - Q_a)(\psi_a f), v}_{\w_a}}{\norm{\nabla v}_{\w_a}}\\
    &\leq \sup_{v \in H_\star^1(\w_a) : v \ne 0} \frac{\norm{v}_{\w_a} \norm{(I - Q_a)(\psi_a f)}_{\w_a}}{\norm{\nabla v}_{\w_a}}\\
    &\lesssim h_{\w_a} \norm{(I - Q_a)(\psi_a f)}_{\w_a},
  \end{align*}
  where the last inequality follows (again) from Poincar\'e-Friedrichs inequality.

  The first result can be found using the triangle inequality in combination with Lemma~\ref{lem:globrel},
  \begin{align*}
    \norm{\tilde r_a}_{H^1_\star(\w_a)'} \leq \norm{r_a}_{H^1_\star(\w_a)'} + \norm{\tilde r_a - r_a}_{H^1_\star(\w_a)'} \lesssim \enorm{u - U_h}_{\w_a} + h_{\w_a} \norm{(I - Q_a) (\psi_a f)}_{\w_a}.
  \end{align*}
  The second result follows similarly by combing Lemma \ref{lem:loceff} and 
  \[
    \norm{r_a}^2_{H_\star^1(\w_a)'} \leq 3  \left(\norm{\tilde r_a}^2_{H_\star^1(\w_a)'} +  \norm{\tilde r_a - r_a}^2_{H^1_\star(\w_a)'}\right).
  \]

  The proof of Theorem~\ref{thm:locresequiv} holds for any functional $r \in H_\star^1(\w_a)'$ that can be decomposed in triangle and edge terms.
  In particular it holds for $\tilde r_a$, providing us with the second part of the theorem.
\end{proof}

The oscillation terms will also appear in the local equivalence the standard residual estimator. 
The effects are summarized in the following lemma. The proof follows easily by mimicking the proof of Lemma \ref{lem:starequiv}
using the relations found in the previous Theorem.
\begin{lem}
  \label{lem:locequivosc}
  The estimator $\norm{\sigma_a}_{\w_a}$ is locally equivalent to the standard residual estimator up to oscillation terms.
  To be precise, 
  \begin{align*}
    \norm{\sigma_a} + h_{\w_a}\norm{(I - Q_a)(\psi_a f)}_{\w_a} &\gtrsim h_{\w_a} \norm{f + \Delta U_h}_{\w_a} + h^{1/2}_{\w_a} \norm{\llbracket \nabla U_h \rrbracket}_{\gamma_a},\\
    \norm{\sigma_a} &\lesssim h_{\w_a}\norm{(I - Q_a)(\psi_a f)}_{\w_a} + h_{\w_a} \norm{f + \Delta U_h}_{\w_a} + h^{1/2}_{\w_a} \norm{\llbracket \nabla U_h \rrbracket}_{\gamma_a}.
  \end{align*}
\end{lem}
\section{Adaptive finite element method}
We will now consider an adaptive finite element method using the equilibrated residual estimator.
Recall that AFEM can be described by the following loop,
\[
  \texttt{SOLVE} \to \texttt{ESTIMATE} \to \texttt{MARK} \to \texttt{REFINE}.
\]
In words: calculate the Ritz-Galerkin solution for a triangulation, estimate
the error using an a posteriori error estimator, mark elements that need to be refined, refine
the triangulation, and start over. 

After specifying the details of these four modules, we will be able to show that the produced
sequence of Galerkin approximations converges with an optimal rate. This result
is similar to the one given before, except that we will now use a different estimator. This
result is a simplified form on the proof given by Casc\'on and Nochetto \cite{cascon2012}.
A different optimality proof is given by Kreuzer and Siebert \cite{ainsworthbernstein}; therein
slightly different versions of \texttt{MARK} and \texttt{REFINE} are used.

First we need some additional notation. Given a triangulation $\T$ we denote $\W$ for the set of patches, i.e. 
\todo{Better notation than $\W$?}
\[
  \W := \set{\w_a : a \in \V}.
\]
For $a \in \V$ let $\sigma_a$ be the equilibrated local flux. We notate the
  error estimator, oscilation and total error resp. by:
\begin{align*}
  \eta^2_\T(U, \w_a) &:= \norm{\sigma_a}^2_{\w_a}, \\
  \osc^2_\T(f, \w_a) &:=h^2_{\w_a} \norm{(I - Q_a) (\psi_a f)}^2_{\w_a}, \\
  \vartheta^2_\T(U, \w_a) &:= \eta^2_\T(U, \w_a) + \osc^2_\T(f, \w_a).
\end{align*}
The above quantities extend to sets of patches in the usual way 
--- just decompose it as a sum over patches in the set.
Often we drop $a$ from the notation, but we stress that any patch $\w \in \W$ is implicitly
associated to a vertex $a \in \V$.


\paragraph{SOLVE}Given a triangulation $\T$, this method computes
\[
  U = \texttt{SOLVE}(\T),
\]
where $U \in \VV(\T)$ is the \emph{exact} Ritz-Galerkin solution. The effects of using an inexact solver have
been studied in the literature as well, see for example \cite[\S7]{carstensen2014axioms}.
\paragraph{ESTIMATE} Given a triangulation $\T$ with vertices $\V$ and its discrete solution $U \in \VV(\T)$, this module 
calculates the total error indicators --- the equilibrated estimator plus the oscillation. We have
\[
  \set{\vartheta_\T(U, \w)}_{\w \in \W} = \texttt{ESTIMATE}(U, \T).
\]
\paragraph{MARK}
A \emph{D\"orfler Marking} strategy is used for marking. For a parameter $\theta \in (0,1]$ we calculate
\[
  \mathcal{M} = \texttt{MARK}( \set{\vartheta_\T(U, \w_a)}_{a \in \V}, \T),
\]
where $\mathcal{M} \subset \W$ is the \emph{minimal} cardinality set that satisfies
\[
  \vartheta_\T(U,\mathcal{M}) \geq \theta \vartheta_\T(U, \W).
\]
\begin{rem}
  In contrast to the standard residual estimator, marking is done using the total error.
\end{rem}
\todo{Marking for total error really needed?}
\paragraph{REFINE}
The \texttt{MARK} outputs a set $\mathcal{M}$ of \emph{patches} on which the error is proportionally large. 
Logically, all the triangles in this set should be refined. This module calculates a refinement
\[
  \T_\star = \texttt{REFINE}(\T, \mathcal{M}),
\]
where $T_\star$ is the smallest conforming refinement of $\T$ such that the triangles of
all patches in the marked set are bisected.
\todo{Weird refine method in cascon en nochetto}
\\\\
We can now formulate the AFEM algorithm and its produced sequences. For this we use the
iteration counter $k$ as a subscript to differentiate among the sets.
Let an initial triangulation $\T_0$ be given, set $k = 0$ and iterate the following steps:
\begin{enumerate}
\item $U_k = \texttt{SOLVE}(\T_k)$;
\item $\set{\vartheta_k(U_k, \w)}_{\W_k} = \texttt{ESTIMATE}(U_k, \T_k)$;
  \item $\mathcal{M}_k = \texttt{MARK}(\set{\vartheta_k(U_k, \w)}_{\W_k}, \T_k)$;
  \item $\T_{k+1} = \texttt{REFINE}(\T_k, \mathcal{M}_k)$;
  \item $k  = k + 1$.
\end{enumerate}

In \cite[\S 4]{cascon2012} a list of assumptions are given under which the above AFEM algorithm produces
an optimal sequence of solutions $(U_k, \T_k)_{k \geq 0}$. First we will show that these
assumptions --- compare \cite[Assump~4.1]{cascon2012} --- hold using the equilibrium based estimator as described above.
In the standard case we had a refined set holding the triangles that were refined. In this case
we need to define something similar, but then in terms of refined patches:
\[
  \RR_{\T \to \T_\star} = \set{ \w \in \W : K \not \in \T_\star\quad \forall K \subset \w}.
\]
So $\RR_{\T \to \T_\star}$ consists of all patches in $\T$ that are \emph{entirely} refined.
\begin{lem}
  \label{lem:assumptions}
  Let a triangulation $\T$ and a refinement $\T_\star \geq \T$ be given, with $U \in \VV(\T)$
  and $U_\star \in \VV(\T)$ the appropriate discrete solutions. There
  exists constants $C_1, C_2, C_3$ such that:
  \begin{enumerate}
    \item The estimator is reliable; the approximation error can be bound using the total error estimator:
      \[
        \enorm{ u - U}_{\O}^2 \leq C_1 \vartheta^2_\T(U, \W) = C_1 \left [ \eta^2_\T(U, \W) + \osc^2_\T(f, \W)\right]
      \]
    \item The estimator is efficient; the estimator provides a lower bound for the error (up to oscillation):
      \[
        C_2 \eta^2_\T(U, \W) \leq \enorm{u - U}^2_{\O} + \osc_\T^2(f, \W).
      \]
    \item The estimator provides a localized upper bound, i.e. the error between $U$ and $U_\star$ can be bound using the
      refined set $\RR_{\T \to \T_\star}$:
      \[
        \enorm{U_\star - U}^2_{\O}  \leq C_1\vartheta^2_\T(U, \RR_{\T \to \T_\star}).
      \]
    \item The estimator provides a local lower bound, i.e. the reverse of the above
      \[
        C_3 \eta^2_\T(U, \RR_{\T \to \T_\star}) \leq \enorm{U_\star - U}^2_{\O}  + \osc_\T^2(U, \RR).
      \]
      \textcolor{blue}{In \cite{cascon2012} this is actually defined in terms of $\RR^n$, this consists
        of all refined patches that satisfy the interior node property. In \cite{kreuzersiebert} this
        restriction is not made, instead the estimator is locally compared to the residual estimator.
      The latter is what we do as well, so I don't think we need this additional interior node property.}
  \end{enumerate}
\end{lem}
\begin{proof}
  Reliability follows directly from Theorem \ref{thm:discbounds},
  \[
    \enorm{u - U_h}_{\O}^2 \lesssim \sum_{a\in \V_h}\norm{\tilde r_a}^2_{H^1_\star(\w_a)'} + h^2_{\w_a} \norm{(I - Q_a)(\psi_a f)}^2_{\w_a} \leq \sum_{a \in V_h} \norm{\sigma_a}^2_{\w_a} + \osc_\T^2(f, \w_a) = \vartheta^2_\T(U, \W)
  \]
  By the same theorem we similarly deduce the efficiency,
  \begin{align*}
    \enorm{u - U}_{\O}^2 + \osc_\T^2(f, \W) &\gtrsim \sum_{a \in \V_h} \enorm{u - U}_{\w_a}^2 + h_{\w_a}^2 \norm{(I-Q_a)(\psi_af)}_{\w_a}\\
      &\gtrsim \sum_{a\in \V_h} \norm{\tilde r_a}_{H_\star^1(\w_a)'} \\
    &\gtrsim \sum_{a \in \V_h} \norm{\sigma_a}_{\w_a} = \eta_\T^2(U, \W).
  \end{align*}
  where the first inequality follows since each triangle is contained in a uniformly bounded number of patches.

  The localized upper bound follows from equivalence with the classical residual estimator. Denote $\hat \eta_\T$ for the
  classical residual estimator (cf. Definition \ref{def:clasest}). 
  From Theorem \ref{thm:residual_erro} with $\hat \RR = \T \setminus \T_\star$ we have a localized upper bound for the standard 
  estimator $\hat \eta_\T$. We can then bound this in terms in patches, specifically
  \begin{align*}
    \enorm{ U_\star -  U}^2 &\lesssim \hat\eta_\T(U, \hat\RR)^2 \\
    &=\sum_{K \in \hat \RR}h^2_K\norm{f +\Delta U}_K^2+h_K \norm{\llbracket \nabla U  \rrbracket}^2_{\partial K \setminus \partial \O} \\
    &\leq\sum_{K\in\hat\RR} \sum_{\w \in \W:K \subset \w}h_\w^2 \norm{f + \Delta U}^2_\w + h_\w \norm{\llbracket \nabla U \rrbracket}^2_{\E^{int} \cap \w}\\
    &= \sum_{w \in \widetilde\RR} h_\w^2\norm{f + \Delta U}^2_\w + h_\w \norm{\llbracket \nabla U \rrbracket}^2_{\E^{int} \cap \w}\\
    &\lesssim \eta_\T^2(U, \widetilde \RR) + \osc_\T^2(f, \widetilde \RR) = \vartheta^2_\T(U, \widetilde \RR).
  \end{align*}
  Where the last inequality follows from the equivalence between the two estimators on patches; see Lemma \ref{lem:locequivosc}.
  Note that we have established the required upper bound for a larger refined set
  \[
    \RR = \RR_{\T \to \T_\star} \subset \widetilde \RR = \set{\w \in \W : K \not \in \hat \RR \quad \exists K \subset \w}.
  \]
  In words, $\RR$ consists of those patches for which all the triangles have been refined, whereas $\widetilde \RR$ consists
  of all patches for which at least one triangle has been refined. However, we have
  \[
    \# \RR \approx \#\widetilde \RR,
  \]
  so this does not become a problem later on.
  \textcolor{blue}{Since we only consider one type of estimator, maybe its wise to just consider $\widetilde \RR$ as the
  definition of the refined set.  }

  For the discrete lower bound, we first need to relate the standard oscillation $\hat\osc_\T^2$ (cf. Definition \ref{def:clasest})
  with the equilibrated oscillation $\osc_\T^2$. We calculate,
  \todo{Does the second inequality hold? Proof?}
  \begin{align*}
    h_{\w_a}^2 \norm{(I-Q_a)(\psi_af)}^2_{\w_a} &= h_{\w_a}^2 \sum_{K \subset \w_a} \norm{(I-Q_K)(\psi_a f)}^2_K \\
     &\geq h^2_{\w_a} \sum_{K \subset \w_a} \norm{(I-Q_K)(f)}^2_K \\
    &\gtrsim \sum_{K \subset \w_a} h_K^2 \norm{(I-Q_K)(f)}^2_K = \hat\osc_\T^2(f, \w_a).
  \end{align*}
  Using this, we can deduce the discrete lower bound from the standard estimator,\todo{Reference for this standard bound}
  \begin{align*}
    \enorm{ U_\star -  U}^2 + \osc_\T^2(f, \widetilde \RR) &\gtrsim \norm{ U_\star -  U}^2 + \hat\osc_\T^2(f, \hat \RR) \\
    &\gtrsim \hat \eta^2_\T(U, \hat \RR) \\
    &= \sum_{K \in \hat \RR} h_K^2\norm{f + \Delta U}_K^2 + h_K\norm{\llbracket U  \rrbracket}^2_{\partial K \setminus \partial \O}\\
    &\gtrsim \sum_{\w \in \RR} h_\w^2\norm{f + \Delta U}_\w^2 + h_\w \norm{\llbracket U  \rrbracket}^2_{\E^{int} \cap \w}= \eta_\T^2(U, \RR).
  \end{align*}
  Here the last inequality follows from the fact that every triangle is contained in a uniformly bounded number of patches.
  \textcolor{blue}{Deze ongelijkheid staat op lossen schroeven. We beginnen met $\widetilde \RR$ en eindigen met $\RR$. Eigenlijk wil ik
  de hele ongelijkheid in termen van $\widetilde \RR$ hebben.}
\end{proof}
Furthermore the oscillation satisfies the reduction property.
\begin{lem}
  \label{lem:oscasum}
  The oscillation satisfies the oscillation reduction property. That is, for any $V \in \VV(\T)$ and $\V_\star \in \VV(\T_\star)$ with $\T \leq \T_\star$  there exists a constant $0 < \lambda < 1$ such that
  \[
  \osc_{\T_\star}^2(f, \W_{\T_\star}) \leq \osc^2_{\T}(f, \W_{\T}) - \lambda \osc^2_{\T}(f, \RR).
  \]
\end{lem}
\begin{proof}
  \textcolor{blue}{Todo, but this seems like a reasonable property.}
\end{proof}


We can now prove the so-called contraction property; this shows convergence of the AFEM method described above.
\begin{thm}
There exists a constant $0 < \alpha < 1$ and $\gamma > 0$, depending on the shape regularity of $\T_0, \theta$ such that
\[
  \enorm{ u - U_{k+1}}^2 +\gamma \osc^2_{k+1}(f, \W_{k+1}) \leq \alpha^2\left(\enorm{ u - U_k}^2 + \gamma\osc^2_k(f, \W_k)\right)
\]
\end{thm}
\begin{proof}
  \textcolor{blue}{
    In the original proof the following  property is used  (based on Young's inequality), for all $\delta > 0$,
    \begin{align*}
      \osc^2_{\T_\star}(V_\star, \W_{\T_\star}) \leq &(1+\delta)\left[ \osc_{\T}^2(V, \W_\T) - \lambda \osc^2(V,\RR)\right]\\
              + & (1 + \delta^{-1}) C_4 \enorm{V_\star - V}^2_\O.
      \end{align*}
    However, in our case the oscillation is not expressed in terms of discrete solutions. How do we fix this?
    Maybe the result shoud use $\eta^2_{j}$ instead of $\osc^2_j$, i.e.
    \[
      \enorm{ u -  U_{k+1}}^2 +\gamma \eta^2_{k+1}(U_{k+1}, \W_{k+1}) \leq \alpha^2\left(\enorm{ u -  U_k}^2 + \gamma\eta^2_k(U_k, \W_k)\right)
  \]}

  Adapt the notation used in \cite{cascon2012}, for $j =0,1,2,\dots$ we write
  \begin{alignat*}{2}
    e_j &:= \enorm{u - U_j}_{\O}, &\quad E_j &:= \enorm{U_{j+1} - U_j}_{\O}^2,\\
    \osc^2_j &:= \osc_{\T_{j}}^2(f, \W_{\T_{j}}), &\quad \osc^2_j(\mathcal{M}_j) &:= \osc_{\T_j}^2(f, \mathcal{M}_j), \\
    \eta^2_j &:= \eta_{\T_j}^2(U_j, \W_{\T_j}), &\quad \eta^2_j(\mathcal{M}_j) &:= \eta^2_{\T_j}(f, \mathcal{M}_j).
  \end{alignat*}

  Notice that a piecewise polynomial on $\T_{k}$ is also a piecewise polynomial on $\T_{k+1}$, since $\T_{k+1} \geq \T_k$.
  Therefore we have $U_{k+1} - U_{k} \in \VV(\T_k)$; combined with Galerkin orthogonality this yields,
  \[
   % e_{k+1}^2 = \enorm{u - U_{k+1}}^2_{\O} = \enorm{u - U_k}^2_{\O} - \enorm{U_{k+1} - U_{k}}^2_{\O} = e_k^2 - E_k^2.
    e_{k+1}^2 = e_k^2 - E_k^2 \implies e_{k+1}^2 + \gamma \osc^2_{k+1} = e_k^2 + \gamma \osc^2_{k+1} - E_k^2.
  \]
  For a constant $\gamma$.
  %Let $\gamma$ be a constant that is to be determined, then adding an oscillation term to both sides yields,
  %\[
  %  e_{k+1}^2 + \gamma \osc^2_{k+1} = e_k^2 + \gamma \osc^2_{k+1} - E_k^2.
  %  %\enorm{u - U_{k+1}}^2_{\O} + \gamma \osc^2_{k+1} = \enorm{u - U_k}^2_{\O} + \gamma \osc^2_{k+1}- \enorm{U_{k+1} - U_{k}}^2_{\O}.
  %\]
  Next invoking the oscillation reduction from Lemma \ref{lem:oscasum} property results in,
  \begin{align}
    e_{k+1}^2 + \gamma \osc^2_{k+1} &\leq e_k^2 + \gamma \osc^2_k - \lambda \gamma \osc^2_k(\RR_k) - E_{k}^2\label{eq:ineq1}\\
    &= \left[ e_k^2 + \gamma \osc^2_k \right] - \left[ E_k^2 + \lambda \gamma \osc^2_k(\RR_k)\right] \nonumber.
    %\enorm{u - U_{k+1}}^2_{\O} + \gamma \osc^2_{k+1} &\leq \enorm{u - U_k}^2_{\O} + \gamma \osc^2_k - \lambda \gamma \osc^2_k(\RR_k) - \enorm{U_{k+1} - U_{k}}^2_{\O}\label{eq:ineq1} \\
    %&= \left[ \enorm{u - U_K}^2_\O + \gamma \osc^2_k\right]- \left[\enorm{U_{k+1} - U_k}_{\O} + \lambda \gamma \osc^2_k(\RR_k)\right] \nonumber.
  \end{align}
  The discrete lower bound from Lemma \ref{lem:assumptions} read as
  \[
    \enorm{U_{k+1} -  U_k}_\O \geq C_3 \eta_k^2(\RR_k) - \osc_k^2(\RR_k) \geq C_3 \eta_k^2(\mathcal{M}_k) - \osc_k^2(\RR_k),
  \]
  where the last inequality follows since $\mathcal{M}_k \subset \RR_k$.
  Combining this lower bound with \eqref{eq:ineq1} gives us,
  \[
    e_{k+1} + \gamma \osc^2_{k+1} \leq \left[ e_k^2 + \gamma \osc^2_k\right] - \left[C_3 \eta_k^2(\mathcal{M}_k) + (\lambda \gamma - 1)\osc_k^2(\RR_k)\right].
   % \enorm{u - U_{k+1}}^2_\O + \gamma \osc^2_{k+1} \leq 
    %\left[ \enorm{u - U_K}^2_\O + \gamma \osc^2_k\right]- \left[C_3 \eta_k^2(\mathcal{M}_k) + (\lambda \gamma - 1)\osc_k^2(\RR_k)\right].
  \]
  Now let $\gamma$ be such that $\lambda\gamma - 1 \geq C_3$, then we may replace $\osc_k^2(\RR_k)$ by the smaller $\osc_k^2(\mathcal{M}_k)$.
  This gives us
  \begin{align*}
    e_{k+1} + \gamma \osc^2_{k+1} &\leq \left[ e_k^2 + \gamma \osc^2_k\right] - \left[C_3 \eta_k^2(\mathcal{M}_k) + (\lambda \gamma - 1)\osc_k^2(\mathcal{M}_k)\right] \\
    &\leq \left[ e_k^2 + \gamma \osc^2_k\right] - \left[C_3 \vartheta_k^2(\mathcal{M}_k) + (\lambda \gamma - 1 - C_3)\osc_k^2(\mathcal{M}_k)\right] \\
  \end{align*}
  
  Now recall that by D\"orfler marking property, we have $\vartheta_k^2(\mathcal{M}_k) \geq \theta^2 \vartheta_k^2$, so that the above simplifies to,
  \[
    e_{k+1} + \gamma \osc^2_{k+1} \leq  e_k^2 + \gamma \osc^2_k - (\lambda \gamma - 1 - C_3) \osc_k^2(\mathcal{M}_k) - C_3 \theta^2 \vartheta_k^2.
  \]
  Invoke the reliability bound from \ref{lem:assumptions} to reduce this to,
  \begin{align*}
 e_{k+1} + \gamma \osc^2_{k+1} &\leq (1 - C_1^{-1} C_3 \theta^2)e_k^2 + \gamma \osc^2_k - (\lambda \gamma - 1 - C_3) \osc_k^2(\mathcal{M}_k).
  \end{align*}
  \textcolor{blue}{
    Can we proceed? Does not seem like it.
  }



\end{proof}


\section{Optimality AFEM}
We will show that convergence of AFEM driven by the equilibrated estimator is actually optimal.
From the reliability and efficiency of the estimator, we find that
\[
  \enorm{u - U}^2_\O + \osc^2_{\T}(f, \W) \approx \eta_\T^2(U, \W) + \osc_\T^2(f, \W) = \vartheta_\T^2(U, \W).
\]
The total error estimator is equivalent to the approximation error (up to an oscillation term).
Since we cannot get rid of the oscillation term (cf. \cite[Rem~5.1]{cascon2008}), it makes sense to incorporate this into the
definition of an approximation class.

Denote $\TT$ for the set of all conforming refinements of $\T_0$ found by applying bisection. Moreover, let $\TT_N \subset \TT$ be the restriction of
triangulations with at most $N$ triangles more than $\T_0$, i.e. $\T \in \TT_N$ if $\T \leq \#T_0 + N$.
We define the optimal approximation class in terms of $s > 0$. For $v \in H_0^1(\O)$ a measure for the optimal convergence
rate is given by
\[
  \abs{v}_{\A_s} := \sup_{N > 0}N^s \inf_{\T \in \TT_N} \left(\enorm{v - V}_{\O}^2 + \osc_\T^2(V, \W)\right)^{1/2}.
\]
As approximation class we take all the functions for which this measure is finite, i.e.
\[
  \A_s := \set{ v \in H_0^1(\O) : \abs{v}_{\A_s} < \infty}.
\]
So if the solution $u$ of the Poisson problem satisfies $u \in \A_s$, then the total 
error of the discrete solution $U \in \VV(\T)$ on the \emph{best} partition $\T \in \TT_N$ satisfies
\[
  \enorm{u - U}^2_\O + \osc^2_{\T}(f,\W) \leq N^{-s} \abs{u}_{\A_s}.
\]
Our goal is to proof that the sequence $\left(U_k\right)_{k \geq 0}$ satisfies a similar convergence rate.

\textcolor{blue}{
  Notice that this optimality class explicitly depends on our definition of oscillation. This makes
  it tougher to compare optimally results among different estimators. Possibly show equivalence of
our oscillation term with a standard one?}

The proof of this optimally is very similar to the standard case in \S\ref{sec:afem}. 
We opt to follow the steps given by Casc\'on and Nochetto in \cite{cascon2012}.
Define the maximum marking parameter by
\[
  \theta_*^2 := \frac{C_2}{(1+ C_1)(1 + C_2)}.
\]
\begin{lem}
  Let $\T \in \TT$ with discrete solution $U \in \VV(\T)$, with a marking parameter $\theta \in (0, \theta_*)$ and refinement $\T \leq \T_\star \in \TT$ with
  a discrete solution $U_\star \in \VV(\T_\star)$ that satisfies
  \[
    \enorm{ u - U_\star}_\O^2+ \osc^2_{\T_\star}(f, \W_\star) \leq \mu (\enorm{u - U}_\O^2 + \osc_\T^2(f, \W)),
  \]
  for $\mu := 1 - {\theta^2 \over \theta_*^2}$.

  Then the refined set $\RR_{\T \to \T_\star}$ satisfies the D\"orfler property
  \[
    \vartheta_\T(U, \RR) \geq \theta \vartheta_\T(U, \W).
  \]
\end{lem}
\begin{proof}
  From the lower bound in Lemma \ref{lem:assumptions} we see
  \[
    C_2 \vartheta_\T^2(U, \W) \leq \enorm{u - U}_\O^2 + (1+C_2)\osc_\T^2(f, \W) \leq (1 + C_2)\left(\enorm{u - U}_\O + \osc_\T^2(f, \W)\right),
  \]
  and thus
  \[
    {C_2 \over 1 + C_2} \vartheta_\T^2(U, \W) \leq \enorm{u - U}_\O^2 + \osc_\T^2(f,\W).
  \]
  Combining this with the assumption gives us
  \begin{align*}
    (1 - \mu){C_2 \over 1 + C_2} \vartheta_\T^2(U, \W) &\leq (1 - \mu) \left[ \enorm{u - U}_\O^2 + \osc_\T^2(f, \W)\right]\\
    &\leq \enorm{u - U}_\O^2 + \osc_\T^2(f,\W) - \enorm{u - U_\star}_\O^2 - \osc_{\T_\star}(f, \W_\star)\\
    &\leq \enorm{U_\star - U}_\O^2 + \osc_\T^2(f, \W) - \osc^2_{\T_\star}(f ,\W_\star).
  \end{align*}
  Now for the oscillation we recall that \todo{Proof?}
  \[
    \osc_\T^2(f,\W) - \osc^2_{\T_\star}(f, \W_\star) \leq \osc^2_\T(f, \RR) \leq \vartheta_\T^2(U, \RR).
  \]
  Put these last two inequalities together with the localized upper bound from Lemma \ref{lem:assumptions} to find,
  \[
    (1 - \mu){C_2 \over 1 + C_2}\vartheta_\T^2(U, \W) \leq \enorm{U_\star - U}_\O^2 + \vartheta_\T^2(U, \RR) \leq (1 + C_1) \vartheta_\T^2(U, \RR).
  \]
  In conclusion, by definition of $\theta_*^2$ and $\mu$ we see that
  \[
    \vartheta_\T^2(U, \RR) \geq  (1 - \mu){C_2 \over (1 + C_1)(1 + C_2)}\vartheta_\T^2(U, \W) \geq \theta^2 \vartheta_\T^2(U, \W). 
  \]
\end{proof}

\begin{cor}
  Let $\set{\T_k, \VV_k, U_k}_{k \geq 0}$ be the sequence of results produced by AFEM for $\theta \in (0, \theta_*)$. If $u \in \A_s$ then the minimal D\"orfler set $\M_k$ satisfies
  \[
    \M_k \lesssim \left(1 - \frac{\theta^2}{\theta^2_*}\right)^{-1 / 2s} \abs{u}_{\A_s}^{1/s} \left(\enorm{u - U_k}_\O^2 + \osc^2_{\T_k}(f, \W_k)\right)^{-1/2s}
  \]
\end{cor}
\begin{proof}
  Let $\mu$ be as in the previous Lemma and set $\epsilon^2 := \mu \left( \enorm{u - U_k}_\O^2 + \osc_{\T_k}^2(f, \W_k)\right)$.
  Since $u \in \A_s$, by definition of the approximation class there exists a triangulation $\T_\epsilon \in \TT$ with
  discrete solution $U_\epsilon$ such that \todo{Elaborate}
  \[
    \# \T_\epsilon - \# \T_0 \leq \abs{u}^{1/s} \epsilon^{-1/s}, \quad \text{and } \enorm{u - U_\epsilon}_\O^2 + \osc^2_{\T_\epsilon}(f, \W_{\epsilon}) \leq \epsilon^2.
  \]
  Now take $\T_\star = \T_k \oplus \T_\epsilon$ --- the smallest common refinement of $\T_k$ and $\T_\epsilon$ --- then
  we have  \todo{proof this, see Lemma 6.1 \cite{cascon2012}}
  \begin{align*}
    \enorm{u - U_\star}_\O^2 + \osc^2_{\T_\star}(f, \W_\star) \leq \enorm{u - U_\epsilon}_\O^2 + \osc^2_{\T_\epsilon}(f,\W_\epsilon) \leq \epsilon^2.
  \end{align*}
  By definition of $\epsilon^2$ we see that $u - U_\star$ satisfies the conditions of the previous lemma, and thus we 
  may conclude that $\RR_{\T_k \to \T_\star}$ satisfies the D\"orfler property. Since $\mathcal{M}_k$ is the \emph{minimal} cardinality set that satisfies
  the D\"orfler property we may conclude that $\# \mathcal{M}_k \leq \# \RR_{\T_K \to \T_\star}$.
  
  Now $\RR_{\T_k \to \T_\star}$ consists of all those patches that were completely refined when going from $\T_k$ to $\T_\star$. Since
  every triangle is contained in a bounded number of patches, with a bound only depending on $\T_0$, we conclude
  \[
    \# \mathcal{M}_k \leq \RR_{\T_k \to \T_\star} \lesssim \# \T_\star - \# T_k \leq \# \T_\epsilon - \# \T_0 \leq \abs{u}_{\A_s}^{1/s}\epsilon^{-1/s}.
  \]
  The third inequality follows by a property of the smallest common refinement.\todo{Lemma toevoegen}
\end{proof}
We are almost ready to give the main theorem. For this we need to bound the number of refined triangles. This number can inflate in
one iteration, but the cumulative sum behaves correctly. We have
\begin{equation}
  \label{eq:nvbound}
  \# \T_k - \# \T_0 \lesssim \sum_{i = 0}^{k-1} \# \mathcal{M}_i.
\end{equation}
For a proof in the standard case, see \cite{ste08}. In the classical case $\mathcal{\hat M}_i$ consists of the marked \emph{triangles}.
We consider marked \emph{patches} $\mathcal{M}_i$, but these are equivalent in terms of cardinality --- $\#\mathcal{\hat M}_i \approx \# \mathcal{M}_i$ --- so the above result also holds in our case.
We are finally ready to state the main theorem.
\begin{thm}
  Take the marking parameter $\theta \in (0, \theta_*)$ and suppose that $u \in \A_s$ for some $s > 0$.

  It holds that
  \[
    \enorm{u - U_k}_\O + \osc_k(f,\W_k) \lesssim \abs{u}_{\A_s} \left(\#\T_k - \#\T_0\right)^{-s}
  \]
\end{thm}
\begin{proof}
From the previous corollary, bound \eqref{eq:nvbound} and contraction property we deduce,
\begin{align*}
  \# \T_k - \# \T_0 &\lesssim \sum_{i = 0}^{k-1} \# \mathcal{M}_i \\
  &\lesssim \left(1 - \frac{\theta^2}{\theta^2_*}\right)^{-1 / 2s} \abs{u}_{\A_s}^{1/s} \sum_{i=0}^{k-1} \left(\enorm{u - U_i}_\O^2 + \osc^2_{\T_i}(f, \W_i)\right)^{-1/2s}\\
  &\approx \left(1 - \frac{\theta^2}{\theta^2_*}\right)^{-1 / 2s} \abs{u}_{\A_s}^{1/s} \sum_{i=0}^{k-1} \left(\enorm{u - U_i}_\O^2 + \gamma\osc^2_{\T_i}(f, \W_i)\right)^{-1/2s}\\
  &\leq \left(1 - \frac{\theta^2}{\theta^2_*}\right)^{-1 / 2s} \abs{u}_{\A_s}^{1/s} \left(\sum_{i=1}^{k} \alpha^{i \over s}\right) \left(\enorm{u - U_k}_\O^2 + \gamma\osc^2_{\T_k}(f, \W_k)\right)^{-1/2s}\\
  &\lesssim \abs{u}_{\A_s}^{1/s}\left(\enorm{u - U_k}_\O^2 + \gamma\osc^2_{\T_k}(f, \W_k)\right)^{-1/2s},
\end{align*}
since $\alpha < 1$. Standard algebra calculations on this inequality provide us with the wanted result.
\end{proof}
\textcolor{blue}{Write something about why this shows optimality. Why do these results involve $(1 - \frac{\theta^2}{\theta^2_*})^{-1/2s}$, this is just a non-interesting constant right?}

\textcolor{blue}{
  Most of the Lemma's in this last part do not differ from the ones for the standard residual estimator. Can we
  shorten it? The contraction property still needs work; in its current case it is broken.  Similarly,
  the discrete lower and upper bounds need some work. 
  \\
  Can we circumvent the equivalence with the standard residual estimator?
  \\
  Considering all of the above proofs, I think the more appropriate definition for refined patches is to let 
  $\RR_{\T \to \T_\star}$ consist of all patches such that at least one triangle is refined. Feels more natural.
  \\
  The above AFEM works by marking elements according to the total error $\vartheta$. Can we achieve
  optimality as well if we mark only for the estimator $\eta$? I think this is possible, given that we have something like
  \[
    \osc_\T^2(f,\w_a) \leq \eta_\T^2(U, \w_a).
  \]
  \\
  This is more or less the optimally proof. My biggest objection with everything is that this AFEM method assumes
  we mark the elements according tot $\norm{\sigma_a}_{\w_a}$. Whereas the constant-free
  estimator is given by $\norm{\sum \sigma_a}_{K}$ for triangles $K \in \T$.
Is the above easily adaptable to use with the second estimator instead?
}


\section{Ern and Volharik}
\textcolor{blue}{Explain construction Ern an Volharik. Allows for easier implementation}
The equilibration method is based on constructing the difference flux $\sigma^\triangle = \sum_{a \in \V} \sigma_a =  \sigma - \nabla U$,
with $\sigma^\triangle$ from the broken Raviart-Thomas space $\RT_p^{-1}$. 
An equivalent equilibration method is introduced by Ern and Volharik in \cite{ernequil}. They
propose constructing the actual flux $\sigma$, instead of the difference $\sigma - \nabla U$. 
\textcolor{Actually, I think there is a minus sign missing somewhere, shouldn't we have $\sigma^\triangle = \nabla U - \sigma$?}

In notation of \cite{ernequil} we define $\zeta_a := \sigma_a - \psi_a \nabla u_h$. By partition of unity we see that 
$\sum_{a \in \V} \zeta_a = \sigma^\triangle - $

\section{Auxiliary results}
\textcolor{blue}{I need references for this: does the constant only depend on shape regularity in case of a star patch?}
\label{sec:aux}
Let $a \in V_h^{int}$, for $v \in H_\star^1(\w_a)$ we have $v_{\w_a} = 0$, and by Poincare we have a constant $C_P(\w_a)$ only dependent
on the shape regularity, such that 
\[
  \norm{v}_{\w_a} = \norm{v - v_T}_{\w_a} \leq C_P(\w_a) h_{\w_a} \norm{\nabla v}_{\w_a}
\]
For $a \in V_h^{bdr}$, then for $v \in H_\star^1(\w_a)$ we know that $v = 0$ on $\partial \O \cap \partial \w_a$, by Friedrichs inequality
 we
have a constant $C_F(\w_a, \partial \O \cap \partial \w_a)$ only dependent on the shape regularity, such that
\[
  \norm{v}_{\w_a} \leq C_F(\w_a, \partial \O \cap \partial \w_a) h_{\w_a} \norm{\nabla v}_{\w_a} \quad \forall v \in H_\star^1(\w_a).
\]

\begin{proof}
\end{proof}
(cf Finite and Boundary Element Tearing and Interconnecting Solvers)

\iffalse
\section{Dump}

  \textcolor{red}{
  Using an equivalence of norms we will show that for $P = (P^T, P^e) \in \P_r^{-1}(\w_a) \times \P_r^{-1}(\gamma)$ we have
    \[
      \|P\| := \sup_{v \in H^1(\w_a) : V\not \equiv C_{st}} \frac{\langle{v, P^T}\rangle_{\w_a} + \langle v, P^e \rangle}{\norm{\nabla v}_{\w_a}} \gtrsim
      \norm{p^T}_{\w_a} + \norm{p^e}_{\gamma_a}.
    \]
    First we need to show that $\norm{P}$ defines a norm: triangle inequality, scalar multiplication are easy.
    The next property is $\norm{P} = 0 \implies P^T = 0 \text{ and }  P^e = 0$. For this, consider a bubble function $B_a \in H^1(\w_a)$
    which vanishes on all edges  ($\E_h \cap \w_a$) and such that $B_a > 0$ on each triangle in $\w_a$. Then we have $B_a P^T \in H^1(\w_a)$, since $P^T$ is only problamatic on the edges (FORMALISE). And thus,
    \[
      \norm{P} = 0 \implies \langle{B_aP^T, P^T}\rangle_{\w_a} + \langle B_aP^T, P^e \rangle_{\gamma_a} = 0 \implies B_a(P^T)^2 = 0\quad\text{a.e.} \implies P^T = 0
    \]
    Similarly, we can pick (how?) a function $w \in H^1(\w_a)$ such that $w > 0$ on $\gamma_a$ and such that $w$ is orthogonal to $\P_r^{-1}(\w_a)$.
    We can (construction?) naturally extend $P^e$ (which is defined on the interior edges only) to a function $P^e$ which is defined for the whole of $\w_a$ such that $P^e \in H^1(\w_a)$. Then $wP^e \in H^1(\w_a)$ is a non-constant function for which we find
    \[
      \norm{P} = 0 \implies \langle{wP^e, P^T}\rangle_{\w_a} + \langle wP^e, P^e \rangle_{\gamma_a}  = 0 \implies w(P^e)^2 = 0 \quad\text{a.e on }\gamma_a \implies P^e = 0
    \]
    by orthogonality on $P^T$.
    The previous constructions also directly show that $\norm{P} \geq 0$, so that $\norm{P}$ is a norm on $\P_r^{-1}(\w_a)\times\P_r^{-1}(\gamma_a)$. By equivalence of norms on finite dimensional spaces we find the wanted. Unfortunately this constant
    still depends on $\w_a$. How can we overcome this problem? Replacing the inner products using substition?
    Given that the equivalence holds, we are directly done with the proof. We do not need the second part.
  }
\fi
\end{document}
