\documentclass[thesis.tex]{subfiles}
\begin{document}
The (theoretical) results from the previous chapter show the potency of the equilibrated flux estimator. 
Most results are upper- or lower bounds and some of these still depend on unknown constants.
To address these uncertainties we would like to measure the performance in practice.
Unfortunately, constructing $\vsiga$ in its current form is far from trivial, as one can already see from
the complexity of the lowest order algorithm \cite[Alg~4]{braess2006equilibrated}.
An equivalent --- and easier to implement --- construction is given by Ern and Volharik in \cite{ernequil}. In this
chapter we will prove this equivalency and give some other implementational details.

\section{Ern and Volharik's construction}
The equilibration method presented in the previous chapter is based on constructing the difference flux $\vsig^\triangle 
= \nabla U - \v{\sigma}$, 
with $\vsig^\triangle$ from the broken Raviart-Thomas space~$\RT_p^{-1}(\T)$. 
Ern and Volhralik \cite{ernequil} propose constructing the actual flux $\vsig \in \RT_p(\T)$ instead of the difference.

In notation of \cite{ernequil} we locally define $\v{\zeta_a} := \v{\sigma_a} - \psi_a \nabla U$ and globally write $\vzet := \sum_{a \in \V} \vzeta$.
By the partition of unity we have 
$ \v{\zeta} = \vsig^\triangle - \nabla U$, so that one retrieves a reliable and efficient estimator after replacing $\vsig^\triangle$ with $\vzet + \nabla U$ in the previous
chapter. Rewriting the characteristic property \eqref{eq:sigmadefdisc} of a locally equilibrated flux $\v{\sigma_a}$ reveals
\begin{align*}
  -\ip{\v{\sigma_a}, \nabla v}_{\w_a}  = \ip{\tilde r_a, v} &=  \ip{\Pi_{p-1}^a(\psi_a f), v}_{\w_a} - \ip{\nabla U, \nabla \left(\psi_a v\right)}_{\w_a} \\
  &=  \ip{\Pi_{p-1}^a(\psi_a f) - \nabla \psi_a \cdot \nabla U, v}_{\w_a} - \ip{\psi_a \nabla U, \nabla v}_{\w_a} \quad \forall v \in H_\star^1(\w_a).
\end{align*}
This holds if and only if for all $v \in H_\star^1(\w_a)$ we have 
\begin{equation}
  \label{eq:defzeta}
  \ip{\Pi_{p-1}^a(\psi_a f) - \nabla \psi_a \cdot \nabla U, v}_{\w_a} = - \ip{\v{\sigma_a} - \psi_a \nabla U, \nabla v}_{\w_a} = - \ip{\v{\zeta_a}, \nabla v}_{\w_a}.
\end{equation}
Notice\footnote{
For $a\in \V^{bdr}$ it is clear that $C_c^{\infty}(\w_a) \subset H_\star^1(\w_a)$. For $a \in \V^{int}$, take $v \in C_c^{\infty}(\w_a)$ and consider $v - v_{\w_a} \in H_\star^1(\w_a)$. Plugging $v - v_{\w_a}$ into the equation \eqref{eq:defzeta} and using orthogonality relations shows that actually $ - \ip{\v{\zeta_a}, \nabla v} = \ip{\Pi_{p-1}^a(\psi_a f) - \nabla \psi_a \cdot \nabla U, v}_{\w_a}$ for $v \in C_c^{\infty}(\w_a)$.}
that the weak divergence of $\vzeta$ apparently satisfies  $\div \v{\zeta_a} = \Pi_{p-1}^a(\psi_a f) - \nabla~\psi_a\cdot~\nabla U$.
The latter is a $L^2(\w_a)$ function, and since
%\footnote{We know that $\v{\sigma_a}\in \RT_{p,0}^{-1}(\w_a)$. Moreover $\nabla U \in \RT_p^{-1}(\w_a)$ and since $\psi_a$ vanishes on $\partial \w_a \setminus \partial \O$ we can conclude that $\psi_a\nabla U \in \RT_{p,0}^{-1}(\w_a)$.
 $\v{\sigma_a},\psi_a\nabla U \in \RT_{p,0}^{-1}(\w_a)$ we conclude
\[
 \vsiga - \psi_a \nabla U  = \v{\zeta_a} \in H(\div; \w_a) \cap \RT_{p,0}^{-1}(\w_a) = \RT_{p,0}(\w_a).
\]

Constructing $\v{\sigma_a} - \psi_a\nabla U$ is therefore equivalent to finding $\v{\zeta_a} \in \RT_p(\w_a)$ that satisfies
$\div \v{\zeta_a} = \Pi_{p-1}^a(\psi_af) - \nabla \psi_a \cdot \nabla U$ with minimal $\uanorm{\v{\zeta_a} + \psi_a \nabla U}_{\w_a}$.
This latter problem can be turned into a system of equations. 
\begin{thm}
  For vertex $a \in \V$ the flux $\v{\zeta_a} \in \RT_{p,0}(\w_a)$ can be found by solving
  \begin{subequations}
    \label{eq:systemzeta}
  \begin{alignat}{3}
    \ip{\v{\zeta_a}, \v{\tau}}_{\w_a} - \ip{\div \v{\tau}, {\lambda_a}}_{\w_a} &= - \ip{\psi_a \nabla U, \v{\tau}}_{\w_a} && \quad \forall \v{\tau} \in \RT_{p,0}(\w_a),\label{eq:systemzeta1}\\
    \ip{\div \v{\zeta_a}, q_a}_{\w_a} &= \ip{\psi_a f - \nabla \psi_a \cdot \nabla U, q_a}_{\w_a} && \quad \forall q_a \in \QQ(\w_a), \label{eq:systemzeta2}
  \end{alignat}
\end{subequations}
  for the pair $(\v{\zeta_a}, {\lambda_a}) \in \RT_{p,0}(\w_a) \times \QQ(\w_a)$, with $\QQ(\w_a)$ a polynomial subspace given by
  \[
    \QQ(\w_a) := \left[\begin{aligned}
        &\set{q \in \P_p^{-1}(\w_a) : q_{\w_a} = 0}  &\quad a \in \V^{int},\\
        &\set{q \in\P_p^{-1}(\w_a)}&\quad a \in \V^{bdr}.
      \end{aligned}
      \right.
  \]
\end{thm}
\begin{proof}
  We will start by showing that the characteristic property \eqref{eq:defzeta}  of $\v{\zeta_a}$ is equivalent to \eqref{eq:systemzeta2}.
  Here we will use that $\Pi_{p-1}^a$ is the $L_2$-orthogonal projector on the broken polynomial space $\P_p^{-1}(\w_a)$.  Fix $v \in H_\star^1(\w_a)$, 
  since $\v{\zeta_a} \in \RT_{p,0}(\w_a)$ we know that $\div \v{\zeta_a} \in \P_p^{-1}(\w_a)$, and thus that $\Pi_{p-1}^a( \div \v{\zeta_a} )= \div \v{\zeta_a}$. Combined with orthogonality of the projector the right hand side of \eqref{eq:defzeta} reads
  \[
    -\ip{\vzeta, \nabla v}_{\w_a} = \ip{\div \v{\zeta_a}, v}_{\w_a} = \ip{ \Pi_{p-1}^a(\div \v{\zeta_a}), v}_{\w_a} = \ip{\div \v{\zeta_a}, \Pi_{p-1}^a(v)}_{\w_a}.
  \]

  Something similar for the left hand side of \eqref{eq:defzeta} holds. The expression $\nabla \psi_a \cdot \nabla U$ is part of
  $\P_p^{-1}(\w_a)$, since $\psi_a \in \P_1(\w_a)$ and $U \in \P_{p}(\w_a)$. Similar steps as before give, 
  \begin{align*}
    \ip{\Pi_{p-1}^a(\psi_a f) - \nabla \psi_a \cdot \nabla U, v}_{\w_a} &= \ip{\Pi_{p-1}^a(\psi_a f) - \Pi_{p-1}^a\left(\nabla \psi_a \cdot \nabla U\right),v}_{\w_a}\\
    &= \ip{\Pi_{p-1}^a\left(\psi_a f - \nabla \psi_a \cdot \nabla U\right), v}_{\w_a}\\
    &= \ip{\psi_a f - \nabla \psi_a \cdot \nabla U, \Pi_{p-1}^a(v)}_{\w_a}.
  \end{align*}
  From this we conclude that \eqref{eq:defzeta} is equivalent to
  \begin{equation}
    \label{eq:defzetanoproj}
    \ip{\div \v{\zeta_a}, q_a}_{\w_a} = \ip{\psi_a f - \nabla \psi_a \cdot \nabla U, q_a} \quad \forall q \in \Pi_{p-1}^a\left(H_\star^1(\w_a)\right).
  \end{equation}
  The equivalence is completed by  noting that $\Pi_{p-1}^a\left(H_\star^1(\w_a)\right) = \QQ(\w_a)$ for both interior and boundary vertices $a$.

  \textcolor{blue}{
  For $a \in \V^{int}$ this is obvious, we have $\QQ(\w_a) \subset \Pi_{p-1}^a\left(H_\star^1(\w_a)\right)$ and reversely for $v \in H_\star^1(\w_a)$
  we find  $\ip{\Pi_{p-1}^a(v), 1}_{\w_a} = \ip{v,\Pi_{p-1}^a(1)}_{\w_a} = v_{\w_a} = 0$ so that $\Pi_{p-1}^a(v) \in \QQ(\w_a)$.
  For $a \in \V^{bdr}$ we have $\QQ(\w_a) \subset \Pi_{p-1}^a\left(H_\star^1(\w_a)\right)$. 
  For the other inclusion let $q_a \in \P_p^{-1}(\w_a)$ be given.
  Can we find a sequence $v_\epsilon \in H_\star^1(\w_a)$ such that $\norm{\Pi_{p-1}^a(v_\epsilon) - q_a}_{\w_a} \to 0$ 
  and $v_\epsilon \to v \in H_\star^1(\w_a)$? If so, we have $\Pi_{p-1}^a(v) = q_a$ for a $v \in H_\star^1(\w_a)$ and thus we may conclude
  that $\Pi_{p-1}^a\left(H_\star^1(\w_a)\right) = \QQ(\w_a)$. The difficulty is that $v \in H_\star^1(\w_a)$ vanish on $\partial \w_a \cap \partial \O$. Probably possible using some mollifier.  }

  Next we tackle rewriting the minimizing property. By the polarization identity
  \[
    \uanorm{\v{\zeta_a} + \psi_a \nabla U}_{\w_a}^2 = \uanorm{\v{\zeta_a}}_{\w_a}^2 + \norm{\psi_a \nabla U}^2_{\w_a} + 2 \ip{\v{\zeta_a}, \psi_a \nabla 
    U}_{\w_a}.
  \]
  Now $\psi_a \nabla U$ is fixed,  so minimizing $\uanorm{\v{\zeta_a} + \psi_a \nabla U}_{\w_a}$ for $\vzeta$ is equivalent to minimizing
  \[
    \frac{1}{2} \uanorm{\v{\zeta_a}}^2_{\w_a} + \ip{\v{\zeta_a}, \psi_a \nabla U}_{\w_a}.
  \]
  Together with the previous paragraph we conclude that $\vzeta$ can be found as a solution of \label{eq:defzetanoproj} with
  minimal $\frac{1}{2} \uanorm{\v{\zeta_a}}^2_{\w_a} + \ip{\v{\zeta_a}, \psi_a \nabla U}_{\w_a}$.

  The optimality conditions of the above minimisation problem are given by the system \eqref{eq:systemzeta}.
  To see this we require some evolved optimisation theory. 
  Finding the pair $(\v{\zeta_a}, {{\lambda_a}}) \in \RT_{p,0}(\w_a) \times \QQ(\w_a)$ that solves the system 
  \eqref{eq:systemzeta} is actually a \emph{saddle point problem} (cf. \cite{brezzimixed, braess2007finite}).
  That is, it fits in the general framework of finding $(u, \lambda) \in X \times M$ for Hilbert spaces $X$ and~$M$~with
    \begin{align*}
      a(u,v) + b(v,\lambda) &= \ip{f,v} \quad \forall v \in X,\\
      b(u, \mu) &= \ip{g,\mu} \quad \forall \mu \in M,
    \end{align*}
    for $f \in X', g \in M'$ and \emph{continuous} bilinear forms $a: X \times X \to \R, b : X \times M \to \R$.
    The main result about saddle point problems \cite[Thm~4.2.1]{brezzimixed} provides sufficient conditions under which there 
    is a unique solution $(u, \lambda)$ to the saddle point problem such that~$u$ is also the unique minimizer of 
    $\frac{1}{2} a(u,u) - \ip{f,u}$. Define the operator $B: X \to M' : v \mapsto b(v,\cdot)$. The first condition is that the bilinear form $a(\cdot, \cdot)$ is
    symmetric and coercive on the set of functions $V := \{v \in X : Bv = 0\}$.
    Secondly, the operator $B$ should be surjective.

    Apply this general framework to our system \eqref{eq:systemzeta}; we 
    have $X = \RT_{p,0}(\w_a)$ and $M = \QQ(\w_a)$ with bilinear forms
    \[
      a(\v{\zeta_a},\v{\tau}) = \ip{\v{\zeta_a},\v{\tau}}_{\w_a} \quad\text{and} \quad b(\v{\zeta_a},q_a) = \ip{\div \v{\zeta_a},q_a}_{\w_a} \quad \text{for }\v{\zeta_a}, \v{\tau} \in \RT_{p,0}(\w_a), q_a \in \QQ(\w_a). 
    \]
    The flux space $\RT_{p,0}(\w_a)$ is equipped with the $H^1(\div; \w_a)$-norm: $\norm{\v{\tau}}_{H^1(\div; \w_a)}^2 = \norm{\v{\tau}}_{\w_a}^2 + \norm{\div \v{\tau}}^2_{\w_a}$,
    cf. Definition~\ref{def:hdiv}. One easily sees that the above bilinear forms are continuous.
    Let $\v{\tau} \in V$, then by definition $\ip{\div \v{\tau}, q_a}_{\w_a} = 0$ for all $q_a \in \QQ(\w_a)$, which implies that $\div \v{\tau} = 0$ since $\div \v{\tau} \in \QQ(\w_a)$. Coercivity of  $a(\cdot, \cdot)$ on $V$ therefore follows easily,
    \[
      a(\v{\tau},\v{\tau}) = \ip{\v{\tau},\v{\tau}}_{\w_a} = \norm{\v{\tau}}^2_{\w_a} = \norm{\v{\tau}}^2_{\w_a} + \norm{\div \v{\tau}}^2_{\w_a} = \norm{\v{\tau}}^2_{H(\div; \w_a)}.
    \]
    Given $q_a \in \QQ(\w_a)$ there exists $ \v{\tau_a} \in \RT_{p,0}(\w_a)$  such that 
    $\div  \v{\tau_a} = q_a$ (cf. proof of Theorem~\ref{thm:sigmasolvable}).
    By duality of $Q$ this turns $B$ into a surjective map. 
    All in all, we satisfy the necessary conditions and hence we may conclude from \cite[Thm~4.2.1]{brezzimixed} that there is a unique pair 
    $(\v{\zeta_a}, \lambda) \in \RT_{p,0}(\w_a) \times \QQ(\w_a)$ 
    that solves the saddle point system \eqref{eq:systemzeta}. From \cite[Rem~4.2.1]{brezzimixed} we deduce that $\vzeta$ is the
    unique minimizer of $\frac{1}{2}\uanorm{\vzeta}_{\w_a}^2 + \ip{\vzeta, \psi_a \nabla U}_{\w_a}$ subject to equation \eqref{eq:systemzeta2}. 
\end{proof}
Since both function spaces in the system \eqref{eq:systemzeta} are finite dimensional, the
system can be reduced to a finite set of equations over the bases of $\RT_{p,0}(\w_a)$ and $\QQ(\w_a)$. 
This provides a straightforward algorithm for constructing $\vzeta$.
Using the correspondence with $\vsiga$ we directly find an upper bound (cf. Theorem~\ref{thm:discsynge}). What's more, for
the data oscillation we find $\norm{(I - \Pi_{p})(f)}_K = \uanorm{f - \div\vzet}_K$.
\begin{thm}
  \label{thm:zetaupper}
  Let $\vzeta$ be found by solving \eqref{eq:systemzeta}, then for $\v{\zeta} = \sum_{a \in \V} \vzeta$ we have
  \begin{align*}
    \enorm{u - U}_{\O}^2 &\leq \sum_{K \in \T} \left[ \uanorm{\vzet + \nabla U}_{K} + \frac{h_K}{\pi} \uanorm{ (I-\Pi_{p})(f)}_{K}\right]^2,\\
      &= \sum_{K \in \T} \left[ \uanorm{\vzet + \nabla U}_{K} + \frac{h_K}{\pi} \uanorm{ f - \div \vzet}_{K}\right]^2.
  \end{align*}
\end{thm}
\begin{proof}
  The upper bound follows from Theorem~\ref{thm:discsynge}, since  $\vsig^\triangle = \vzeta + \nabla U$.

  We will show that $f|_K - \div \vzet|_K$  perpendicular to $\P_p(K)$ on every element $K$.
  For an interior vertex $a \in \V^{int}$ we find from the Divergence theorem that 
  \[
    \ip{\div \vzeta, 1}_{\w_a} =  \ip{\vzeta \cdot n, 1}_{\partial \w_a} - \ip{\vzeta, \nabla 1}_{\w_a} = 0,
  \]
  since $\vzeta \in \RT_{p,0}(\w_a)$ ensures that $\vzeta \cdot n = 0$ on $\partial \w_a$.
  This implies that \eqref{eq:systemzeta2} holds the entire test space $\P_p^{-1}(\w_a)$. Now pick a polynomial $q \in \P_p(K)$,
  which we implicitly extend to vanish outside $K$.  Since $\P_p^{-1}(\w_a)$ consists of broken polynomials, we see that $q \in \P_p^{-1}(\w_a)$
  for all patches $\w_a$. Calculating the inner product with $f - \div \vzet$ yields
  \begin{align*}
    \ip{f - \div \vzet, q}_K &= \sum_{a \in \V_K} \ip{ \psi_a f - \div \vzeta, q}_{K} = \sum_{a \in \V_K} \ip {\psi_a f - \div \vzeta, q}_{\w_a}\\
     &=\sum_{a \in \V_K} \ip{\nabla \psi_a \cdot \nabla U, q}_{\w_a}\\
    &= \ip{\nabla 1, q}_{K} = 0,
  \end{align*}
  where third inequality follows from \eqref{eq:systemzeta2}. We conclude that  $f|_K - \div \vzet|_K \perp \P_p(K)$, and because $\div \vzet|_K \in \P_p(K)$
  we infer that $\div \vzet= \Pi_p f$.
\end{proof}


%\section{Bases for $\RT_{p,0}(\w_a)$ and $\QQ(\w_a)$}
\section{Raviart-Thomas space}
The findings in the previous section provide us with the necessary steps for implementation of the equilibrated flux
estimator. In particular we must find bases for $\RT_{p,0}(\w_a)$ and $\QQ(\w_a)$. We will give a short
intermezzo deriving some theory for this first space. 

Consider an arbitrary domain $ \O \subset \R^2$ with triangulation $ \T$, vertices $ \V$, et cetera.
We will start by examining the Raviart-Thomas space $\RT_p( \T)$, without the boundary conditions,
generated by the Raviart-Thomas element.
Recall from \S\ref{sec:deffem} that finite elements are formally represented by a triplet $(K, \mathcal{P}, \mathcal{N})$. For
the construction of the Raviart-Thomas space ~$\RT_p( \T)$
we obviously have $\mathcal{P} := \RT_p(K)$.
We are left with the task of finding suitable nodal variables $\mathcal{N}$ that ensure $H(\div;  \O)$ conformity of the entire space.
For this latter constraint we will use the following characterization of $H(\div;  \O)$.
\begin{thm}
  \label{thm:rtjump}
  The Raviart-Thomas space $\RT_p( \T)$  can be characterized by
  \[
    \RT_p( \T) := H(\div;  \O) \cap \RT_p^{-1}( \T)  = \set{\vsig \in \RT_p^{-1}( \T) : \llbracket \vsig \rrbracket = 0 \, \text{ in } L^2(e) \quad \forall e \in \E^{int}}.
  \]
\end{thm}
\begin{proof}
  We show the inclusion "$\supset$". That is, let $\vsig \in \RT_p^{-1}( \T)$ be given such that $\llbracket \vsig \rrbracket$
  vanishes on all interior edges. We have to show that $\vsig$ has a weak divergence in $L^2( \O)$.
  Let $\f \in D( \O)$ be given, from the divergence theorem for $\vsig|_K \in H^1(\div; K)$ we infer that
  \begin{align*}
    - \int_{ \O}  \nabla \f \cdot \vsig &= - \sum_{K \in  \T} \int_K \nabla \f \cdot \vsig \\
    &=  \sum_{K \in  \T} \int_K \f \cdot \div \vsig - \int_{\partial K} \f \vsig \cdot n\\
    &=  \int_{ \O} \f \cdot \div \vsig - \sum_{e \in  \E^{int}} \int_e \f \llbracket \vsig \rrbracket = \int_{ \O} \f \cdot \div \vsig.
  \end{align*}
  Here we used that $\f$ vanishes on the boundary, whilst $\llbracket \vsig \rrbracket$ vanishes on interior edges.
  The result follows since $\div \vsig \in \P_p^{-1}(\O) \subset  L^2( \O)$, hence we may conclude that $\vsig \in H(\div;  \O)$.

  The other inclusion follows similarly (cf. \cite[Thm~3.2]{gaticasimple}).
\end{proof}

For a triangle $K \in  \T$ we have $\RT_p(K) := \left[ \P_p(K)\right]^2 + \P_p(K)\v{x}$.
Let $\wt \P_p(K)$ denote polynomials of \emph{exactly} degree $p$, then one easily proves the decomposition
$\RT_p(K) = \left [\P_p(K)\right]^2 \oplus \wt \P_p(K) \v{x}$.
From this decomposition it follows that
\[
  \dim{\RT_p(K)} = 2 \dim{\P_p(K)}+ \dim{\wt \P_p(K)} = 2 {2 + p \choose p} + {2 + p - 1 \choose p} = (p+1)(p+3).
\]
The Raviart-Thomas space $\RT_p(K)$ has  the following (unisolvent) characterisation.
\begin{thm}
  Let $K \in  \T$ and $\vsig \in \RT_p(K)$ be given such that
  \begin{alignat*}{2}
    \ip{\vsig \cdot n_e, q}_e &= 0 \quad \forall q \in \P_p(e), &&\quad \forall \text{edge } e \subset \partial K,\\
    \ip{\vsig, \vtau}_K &= 0 \quad \forall \vtau \in \left[\P_{p-1}(K)\right]^2, &&\quad p \geq 1.
  \end{alignat*}
  Then $\vsig \equiv 0$ in $K$.
\end{thm}
\begin{proof}
  Examining the dimensions of the function spaces reveal that the number of equations equals the dimension of $\RT_p(K)$.
  Since $\vsig \cdot n_e \in \P_p(e)$ the first equation implies that $\vsig \cdot n_e$ vanishes on all edges of $K$.

  Let $p \geq 1$, then for any $q \in \P_p(K)$ we have $\nabla q \in \left[\P_{p-1}(K)\right]^2$, and by the divergence theorem this gives that
  $\ip{\div \vsig, q}_K = - \ip{\vsig, \nabla q}_K + \ip{\vsig \cdot n, q}_{\partial K} = 0$. Since $\div \vsig \in P_p(K)$, we
  may conclude that $\div \vsig = 0$.  Decompose $\vsig = \v{q} + q_0\v{x}$ for $\v{q} = [q_1, q_2]^\top$ with $q_1, q_2  \in\P_p(K)$ and $q_0 \in  \wt \P_p(K)$. From the fact that $\div \vsig =0$ we infer that
  \[
    0 = \div \vsig = \div \v{q} + \div \left(q_0 \v{x}\right) = \div \v{q} + (2 + p) q_0 \implies (2 + p) q_0 = - \div \v{q} \in \P_{p-1}(K).
  \]
  Since $q_0 \in \wt \P_p(K)$ we must have $q_0 \equiv 0$ and thus $\vsig = \v{q} \in \left[\P_{p}(K)\right]^2$.

  Notice that $\vsig \cdot n_e = \v{q} \cdot n_e$ is a polynomial of degree $p$  on the whole of $K$.
  For any polynomial $g \in \P_{p-1}(K)$ we infer from the second assumption that
  \[
    \ip{\v{q} \cdot n_e, g}_K = \int_K g \left(\v{q} \cdot n_e\right) = \int_K \v{q} \cdot \v{\tilde g} = \ip{\vsig, \v{\tilde g}}_K = 0 \quad \text{for } \v{\tilde g} = [g n_e^1, gn_e^2]^\top \in \P_{p-1}(K).
  \]
  On the other hand, for some edge $e \in \partial K$ we know that  $\v{q} \cdot n_e$ vanishes.
  Therefore we can decompose $\v{q}\cdot n_e = Lg$ in a polynomial $g \in \P_{p-1}$ and $L$ the hyperplane of $e$ \cite[Lem~3.1.10]{brenner}. Using this polynomial $g$ in the above identity then shows
  \[
    0 = \ip{\v{q}\cdot n_e, g}_K = \ip{Lg, g}_K = \ip{L,g^2}_K.
  \]
  Since $L$ does not vanish on $K$ we conclude that $g = 0$, and thus that $\v{q}\cdot n_e$ vanishes
  on the entire element $K$. This holds for all edge  normals $n_e$,  hence we infer that $\vsig~=~\v{q}~=~0$.
\end{proof}
The number of equations in the previous theorem equals $\dim{\RT_p(K)}$, and therefore provides us with a
set of nodal variables. That is, for each edge $e$ let $\set{q_{e,1},\dots, q_{e,m}}$ be a basis of $\P_p(e)$, and let
 $\set{\vtau_1, \dots, \vtau_r}$ be a  basis for $\left[\P_{p-1}(K)\right]^2$. 
 Define linear functionals on $\RT_p(K)$ associated to edges and elements by
\begin{equation}
  \label{eq:nodalrt}
  \begin{alignedat}{2}
    N_{e,i}(\vsig) &= \ip{\vsig \cdot n_e, q_{e,i}}_e && \quad 1 \leq i \leq m,  e \subset \partial K\\
    N_{K,j}(\vsig) &= \ip{\vsig, \vtau_j}_K &&\quad 1 \leq j \leq r.
  \end{alignedat}
\end{equation}
The set $\mathcal{N}$ containing all of the above functionals forms a basis for $\RT_p(K)'$, and thus
turns the triplet $(K, \RT_p(K), \mathcal{N})$ into a valid finite element.

\section{Raviart-Thomas basis}
We have formally introduced an Raviart-Thomas element $(K, \RT_p(K), \mathcal{N})$. How can we glue these elements
together to form a basis for the Raviart-Thomas space $\RT_p(\T)$? In particular we must ensure $H(\div; \O)$-conformity.

For simplicity we start with two adjoint Raviart-Thomas elements $K_1, K_2 \in  \T$ having a common edge $e = K_1 \cap K_2$.
From the characterization in Theorem~\ref{thm:rtjump} we know that
\[
\RT_p(K_1 \cup K_2) = \set{\vsig \in \RT_p^{-1}(K_1 \cup K_2) : \llbracket \vsig \rrbracket = 0 \text{ in } L_2(e)}.
\]
Pick some $\vsig \in \RT_p^{-1}(K_1 \cup K_2)$ and let $n_e$ be the unit outward normal on $e$ for element $K_1$,
then the conformity condition can be rewritten as
\[
  0 = \llbracket \vsig \rrbracket  = \vsig|_{K_2} \cdot  n_e - \vsig|_{K_1} \cdot n_e.
\]
Since $\vsig|_{K_1} \cdot n_{e}, \vsig|_{K_2} \cdot n_{e}$ are both elements in $\P_p(e)$ we see that 
the  above condition is equivalent to requiring
\[
  \ip{\vsig|_{K_2} \cdot n_e, q} = \ip{\vsig|_{K_1} \cdot n_e, q} \quad \forall q \in \P_p(e).
\]
This is happens if and only if the edge-related nodal variables of $K_1$ and $K_2$ coincide for~$\vsig$.

The above observation is easily extended to the entire Raviart-Thomas space $\RT_p( \T)$.
That is, the space consists of those functions from $\RT_p^{-1}( \T)$ for which all the edge-related
nodal variables of two adjoint elements coincide. For each element $K \in  \T$, denote $\phi^K_{e,i}$ and $\phi^K_{K,j}$
for the nodal basis with respect to \eqref{eq:nodalrt}. A basis for the entire space is then found by
glueing together edge related functions. To be precise, a basis for $\RT_p( \T)$ is given by
\begin{equation}
  \label{eq:basisrt}
  \{\phi^K_{K, j} : K \in  \T, 1 \leq j \leq r\} \cup \{\phi^{K^1}_{e,i} + \phi^{K^2}_{e,i}: K_1, K_2 \in  \T, e = K_1 \cap K_2, 1 \leq i \leq m\},
\end{equation}
with all basis functions naturally extended from $K$ to $\O$.

Next consider $\RT_{p,0}( \T) = \set{ \vsig \in \RT_{p,0}( \T) : \vsig \cdot n = 0 \text{ on } \partial  \O}$.
Such a boundary condition is easily incorporated into the basis \eqref{eq:basisrt}; one just has to remove all the basis functions associated
to boundary edges.

\section{Explicit bases lowest order}
With this (general) information on the  Raviart-Thomas space in mind we will again look at the
system of equations \eqref{eq:systemzeta} defining $\vzeta$. For simplicity we start with an
implementation of this system for $p=0$.

\subsection{Raviart-Thomas}
For the lowest order Raviart-Thomas space we only have edge related nodal variables.
Let $K$ be a triangle with vertices $\set{v_1, v_2, v_3}$ and opposite edges given by $\set{e_1, e_2, e_3}$.
A basis for $\P_0(e_i)$ is simply the constant function~$\1$.
Consider the nodal basis function $\v{\phi_1} \in \RT_0(K)$ associated with $e_1$.
By construction this the solution of
\[
  \int_{e_1} \v{\phi_1}\cdot n_{e_1} = 1 \quad \text{ and } \int_{e_2} \v{\phi_1} \cdot n_{e_2} = 0 = \int_{e_3} \v{\phi_1} \cdot n_{e_3}.
\]

This system can be solved geometrically; as vertex $v_i$ lies opposite of $e_i$, we have $(\v{x} - v_1) \perp n_{e_2}$ for $\v{x} \in e_2$ and similarly $(\v{x} - v_1) \perp  n_{e_3}$ for $\v{x} \in e_3$.
From this we deduce that $\v{\phi_1} = \alpha (\v{x} - v_1)$ for some constant $\alpha$. Pick a  point $\v{x}$ on the hyperplane given by $e_1$ such that
$(\v{x} - v_1) \perp e_1$, then from vector algebra we see
\[
  1 = \alpha(\v{x} - v_1)\cdot n_{e_1} = \alpha \norm{\v{x} - v_1}\norm{n_{e_1}} \cos \theta = \alpha \norm{\v{x} - v_1}.
\]
This latter norm is also known as the \emph{height} $h_1$ of $e_1$. From the well known formula $\frac{1}{2} h_1 \vol(e_1) = \vol(K)$
we observe that $\alpha = \frac{\vol(e_1)}{2\vol(K)}$; a  basis for $\RT_0(K)$ is therefore 
\[
  \v{\phi_i} = \frac{1}{2}\frac{\vol(e_1)}{\vol(K)}( \v{x} - v_i).
\]

Next, we consider the global space $\RT_0(\T)$. For this space all the degrees of freedom are associated
to edges. For every shared edge $e$ we must fix a \emph{global} orientation of the normal vector $n_e$. After fixing such an orientation we can (uniquely) identify the adjacent elements by $K_+$ and $K_-$; with $K_+$ the element for which $n_e$
is an outward pointing normal vector. Similarly write $v_+, v_-$ for the vertex opposite of $e$ in $K_+$ resp $K_-$. The
basis function associated to $e$ is given by
\[
  \v{\phi_e}(\v{x}) = \begin{cases}
    \frac{1}{2}\frac{\vol(e)}{\vol(K_+)}( \v{x} - v_+) & \v{x} \in K_+\\
    - \frac{1}{2}\frac{\vol(e)}{\vol(K_-)}( \v{x} - v_-) & \v{x} \in K_-.
\end{cases}
\]
The extra boundary constraints needed in \eqref{eq:systemzeta} are easily incorporated; a basis for $\RT_{0,0}(\w_a)$ is simply given
by removing the basis functions associated to edges $\partial \w_a \setminus \partial \O$.
\subsection{Polynomial basis}
\todo{Betere titel?}
The system \eqref{eq:systemzeta} defines the polynomial space $\QQ(\w_a) \subset \P_0^{-1}(\w_a)$. For boundary
vertices we can simply take the constant functions $\1$ on each of the elements as basis. For an interior vertex $a \in \V^{int}$
we have the additional mean zero constraint, i.e.
\[
  \QQ(\w_a) = \set{q \in \P_0^{-1}(\w_a) : \ip{q,1}_{\w_a} = 0}.
\]
Write $K_1,\dots, K_n$ for the elements in the patch $\w_a$. A function $p \in \P_0^{-1}(\w_a)$ is constant
with value $p_i$ on $K_i$. The mean zero condition translates into
\[
  0 = \ip{p,\1}_{\w_a} = \sum_{i =1}^n \ip{p_i, \1}_{K_i} = \sum_{i=1}^n p_i \vol(K_i).
\]
A basis for $\QQ(\w_a)$ is therefore given by $q_1, \dots, q_{n-1}$ with
\[
  q_i(x) = \begin{cases}
    \vol(K_n) & \quad x \in K_i\\
    -\vol(K_i) & \quad x \in K_n.
  \end{cases}
\]
Unfortunately this is not an orthogonal basis, but for out test purposes it suffices.
\section{Implementation details}
Lets consider an implementation that finds $\vzeta$ by solving the system \eqref{eq:systemzeta}. 
For simplicity of the implementation we let $p=0$, with $f$ piecewise constant and we take $U$ to be the \emph{linear} Lagrange finite element solution on for some triangulation $\T$. 
Since $U$ is of polynomial degree $1$ we use reliability and efficiency results from \S\ref{sec:lowerorder}. This introduces an extra oscillation term which we simply ignore.

\textcolor{blue}{Do we indeed retrieve the bounds from \S\ref{sec:lowerorder}? Not entirely sure whether solving the system \eqref{eq:systemzeta} is also equivalent to lower 
order approximations of $\vsiga$. Equivalence shown in beginning of this chapter is in trouble due to the edge projector--- integration by parts does not (directly) work.}

Implementation of the estimator $\vzeta$ for $p=0$ follows now quite easily. 
Starting with an interior vertex $a \in \V$ we must solve $(\vzeta, \lambda_a) \in \RT_{0,0}(\w_a)\times \QQ(\w_a)$ from the system
\begin{equation}
  \label{eq:systemp0}
  \begin{alignedat}{3}
    \ip{\v{\zeta_a}, \v{\tau}}_{\w_a} - \ip{\div \v{\tau}, {\lambda_a}}_{\w_a} &= - \ip{\psi_a \nabla U, \v{\tau}}_{\w_a} && \quad \forall \v{\tau} \in \RT_{0,0}(\w_a),\\
    \ip{\div \v{\zeta_a}, q_a}_{\w_a} &= \ip{\psi_a f - \nabla \psi_a \cdot \nabla U, q_a}_{\w_a} && \quad \forall q_a \in \QQ(\w_a).
  \end{alignedat}
\end{equation}
  Let $n$ be the number of triangles in $\w_a$.
  For the space $\RT_{0,0}(\w_a)$ we have one degree of freedom associated to every interior edge, and thus $n$ basis functions $\vtau_i$. The space $\QQ(\w_a)$ has $n-1$ basis functions $q_k$.  To solve the above system we must calculate four type of basis interactions:
  \begin{equation}
    \label{eq:basinact}
    \ip{\vtau_i, \vtau_j}_{\w_a}, \, \ip{\div \vtau_i, q_k}_{\w_a}, \, \ip{\psi_a \nabla U, \vtau_i}_{\w_a}, \text{ and } \ip{\psi_a f - \nabla \psi_a \cdot \nabla U, q_k}_{\w_a}.
  \end{equation}
  The standard finite element approach is to decompose these inner products over the elements $K$, and then use an affine transformation
  to reduce the calculations to interactions on one reference triangle $\hat K$. Unfortunately these transformations do no preserve
  normal components; the result of an affine transformation therefore do not preserve $H(\div)$-conformity. This can be solved 
  by using the Piola transformation~\cite[\S2.1.3]{brezzimixed}. 
  
  We opt not to follow this approach and instead solve the inner products by quadrature. Since  $f$ is piecewise constant, all of the
  terms in \eqref{eq:basinact} are actually polynomials of degree less or equal 2. 
  The edge-midpoint quadrature rule is exact for polynomials for quadratic polynomials, and hence provides us with an easy
  method to calculate the integrals. Denote the edge-midpoints of a triangle $K$ by $m_i$, then this quadrature is given by
  \[
    \int_K p = \frac{\vol(K)}{3} \sum_{k=1}^3 p(m_k)\quad \forall p \in \P_2(K).
  \]
  Applying this to the first term in \eqref{eq:basinact} gives 
  \[
  \ip{\vtau_i, \vtau_j}_{\w_a} = \sum_{K \subset \w_a} \frac{\vol(K)}{3}\sum_{k=1}^3 \vtau_i(m_k) \cdot \vtau_j(m_k).
  \]
  The other terms decompose similarly.  
  
  Quadrature allow us to calculate all of the interactions in \eqref{eq:basinact}.
  Recall that one had to chose an global orientation for each of the edge normal vectors. For this orientation we make clever
  use of the framework provided by iFEM, see \cite{chenifem} for an in-depth overview of this framework. In the next step we collect
  the computed values from \eqref{eq:basinact} into the matrix analogue of the system \eqref{eq:systemp0}. Solving
  this system then provides us with the basis coefficients of $\vzeta$. Using a local-to-global mapping of edge coefficients
  then allows for iterative construction of $\v{\zeta} = \sum_{a \in \V} \vzeta$. 

  The construction of $\vzeta$ for a boundary vertex $a \in \V^{bdr}$ is similar; the interactions
  are just calculated for a slightly different set of functions. Since $f$ is piecewise constant
  we do not have data data oscillation; and the upper bounds in Theorem~\ref{thm:zetaupper} are therefore completely given in terms of $\v{\zeta}$ and 
  $\vzeta$. The last step thus consists in calculating the norms $\norm{\v{\zeta} + \nabla U}_K$ or $\norm{\vzeta + \psi_a \nabla U}_{\w_a}$.
  Since these terms are linear polynomials, we again use edge-midpoint quadrature to exactly calculate them. 

\end{document}
