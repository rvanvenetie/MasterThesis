\documentclass[thesis.tex]{subfiles}
\begin{document}
Exact solutions of many partial differential equations are hard or impossible to find. 
This is a problem in many practical (research) fields --- think of an engineer wanting to solve a heat
equation for some object. One way to overcome this problem is to use some \emph{approximation} method.
That is, instead of seeking the exact solution one constructs a function that is close in norm to the exact solution.
For many types of partial differential equations such approximations are provided by the \emph{finite element method} (FEM).

Consider for example a two-dimensional Poisson
boundary value problem on the unit square $\O = [0,1] \times [0,1]$: solve $u$ from
  \begin{equation*}
  \begin{alignedat}{2}
    -\Delta u &=&1 \quad &\text {in } \Omega, \\
    u &=& 0 \quad &\text{on } \partial\O.
  \end{alignedat}
\end{equation*}
The first step of FEM is to subdivide the domain $\O$ into smaller pieces --- called \emph{elements}. 
For our example this means that
we subdivide the unit square into triangles. In the next step we define a function space on each of these elements, and
combine the elements to form a \emph{finite} dimensional function space over the domain $\O$. An approximation $U$ of the exact solution $u$
is then found in this finite element space by solving a system of equations (each element induces a set of equations).

In order to improve the quality of $U$ we could subdivide every element and calculate an approximation for the
refined subdivision. This classical approach works, i.e. the sequence of FEM approximations converges towards the exact solution.
For some problems a faster convergence rate can be achieved by only subdividing
those elements where the current approximation $U$ has the worst quality. This is the so-called \emph{adaptive} finite element method (AFEM).

The immediate question is how to determine the approximation error $\norm{u - U}$ on an element. We apply approximation techniques
in the first place since the exact solution $u$ is unknown. Fascinatingly there are \emph{estimators} that can be used
to estimate the error $\norm{u - U}$ on an element --- without knowing the exact solution $u$! One well-known
estimator is the residual error estimator. This was the first estimator for which \emph{optimal} convergence of adaptive finite element 
method was proved.

Recently quite some research interest has been in finding estimator with more favorable properties.
In this work we concentrate on the \emph{equilibrated} flux estimator. We given an in-depth overview its
properties and us these properties to show that AFEM using this estimator also provides optimal convergence.
Lastly we experimentally verify these claims using an actual implementation.
\end{document}
