\documentclass[thesis.tex]{subfiles}
\begin{document}
Exact solutions of many partial differential equations are hard or impossible to compute. 
This is a problem in many practical (research) fields --- think of an engineer wanting to solve a heat
equation for some object. One way to overcome this problem is to use an \emph{approximation} method.
That is, instead of seeking the exact solution one constructs a function that is close in norm to the exact solution.
For many types of partial differential equations, such approximations are provided by the \emph{finite element method} (FEM).

Consider for example a two-dimensional Poisson
boundary value problem on the unit square $\O = [0,1] \times [0,1]$: solve $u$ from
\[
  -\Delta u = 1 \,\,\, \text{in} \,\,\, \O, \quad\quad u = 0 \,\,\, \text{on} \,\,\,\partial \O.
\]
The first step of FEM is to subdivide the domain $\O$ into smaller pieces, called \emph{elements}. 
For our example, this means that
we subdivide the unit square into triangles. In the next step, we define a function space on each of these elements, and
combine the elements to form a \emph{finite} dimensional function space over the domain $\O$. An approximation $U$ of the exact solution $u$
is then found in this finite element space by solving a system of equations, where each element imposes a set of equations.
The resulting $U$ is the best approximation of  $u$  in this finite element space.
%BLA

In order to improve the quality of $U$ we could subdivide every element and calculate an approximation for the
refined subdivision. This classical approach works, i.e.~the sequence of FEM approximations converges towards the exact solution.
For some problems, a faster convergence rate can be achieved by only subdividing
those elements where the current approximation $U$ has the worst quality. This is the so-called \emph{adaptive} finite element method (AFEM).

The immediate question is how to determine the approximation error $\norm{u - U}$ on an element. After all, we apply approximation techniques
because the exact solution~$u$ is unknown. Fascinatingly, there are \emph{estimators} that can be used
to estimate the error $\norm{u - U}$ on an element --- without knowing the exact solution $u$! One well-known
estimator is the residual error estimator. This was the first estimator for which \emph{optimal} convergence of adaptive finite element 
method was proved. 

The bound provided by the residual error estimator contains an unknown constant, making it hard to determine when
the approximation error $\norm{u - U}$ is small enough.
Recently, quite some research interest has been conducted in finding estimators with more favourable properties. 
In this work, we concentrate on the \emph{equilibrated flux} estimator. This estimator provides a guaranteed upper bound, i.e.~without
an unknown constant.
We give an in-depth overview of its properties and use these properties to show that AFEM driven by this estimator also provides optimal convergence.
Lastly, we experimentally examine the performance of the estimator using an actual implementation.
\end{document}
