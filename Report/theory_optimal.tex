\documentclass[thesis.tex]{subfiles}
\begin{document}
Having properly introduced the equilibrated flux estimator, we can now turn our attention
to an adaptive finite element method that utilizes this estimator.
Recall from \ref{sec:afem} that AFEM can be described by the following loop,
\[
  \texttt{SOLVE} \to \texttt{ESTIMATE} \to \texttt{MARK} \to \texttt{REFINE}.
\]
In words: calculate the Ritz-Galerkin solution for a triangulation, estimate
the error using an a posteriori error estimator, mark elements that need to be refined, refine
the triangulation, and start over. 

After specifying the details of these four modules, we will be able to show that the produced
sequence of Galerkin approximations converges with an optimal rate. This result
is similar to the classical (or standard) one in \S\ref{sec:optimalafem}, except that we will now use the equilibrated flux estimator. This
result is a simplified form of the proof given by Casc\'on and Nochetto~\cite{cascon2012}.
A different optimality proof is given by Kreuzer and Siebert \cite{kreuzersiebert}; therein
slightly different versions of \texttt{MARK} and \texttt{REFINE} are used.

\section{Design of the adaptive finite element method}
First we need some additional notation. Given a triangulation $\T$ we denote $\W$ for the set of star patches, i.e.~
\[
  \W := \set{\w_a : a \in \V}
  %\mathcal{W} = \mathbb{W} = \mathfrak{W}  = \mathscr{W}= \mathlarger{\mathlarger{\mathlarger{\mathbb{\omega}}}}  := \set{\w_a : a \in \V}.
\]
For $a \in \V$, let $\v{\sigma_a}$ be the equilibrated local flux corresponding to the discrete solution $U \in \VV(\T)$. We denote the
  error estimator, oscillation, and total error resp.~by:
\begin{align*}
  \eta^2_\T(U, \w_a) &:= \uanorm{\v{\sigma_a}}^2_{\w_a}, \\
%  \osc^2_\T(f, \w_a) &:=h^2_{\w_a} \norm{(I - \Pi_p^a) (\psi_a f)}^2_{\w_a}, \\
  \osc^2_\T(f, \w_a) &:= \uanorm{h_a(I - \Pi_{p-1}^a)(f)}_{\w_a}^2 = \sum_{K \subset \w_a} h_K^2 \uanorm{(I - \Pi_{p-1})(f)}^2_K,\\
  \vartheta^2_\T(U, \w_a) &:= \eta^2_\T(U, \w_a) + \osc^2_\T(f, \w_a).
\end{align*}
The above quantities extend to sets of patches in the usual way 
--- just take the sum over the patches in the set.
We stress that any patch $\w \in \W$ is implicitly associated to a vertex $a \in \V$.

\begin{rem}
  In contrast to the \emph{element-wise} classical residual estimator, we use the \emph{patch-wise} equilibrated flux estimator $\uanorm{\vsiga}_{\w_a}$ here. This
  is a natural choice due to the localized patch-wise properties of the equilibrated flux estimator. 
  Moreover, we require the patch-wise equivalence with the classical residual estimator for the optimality proof of AFEM.

  On the other hand, from \S\ref{sec:releffdisc} we know that $\uanorm{\vsig^\triangle}_K$ is a reliable and efficient
  element-wise estimator --- without an unknown constant in the reliability bound. This element-wise variant
  therefore is therefore more convenient in practice. We will come back to this issue later, after we have proven
  optimality of the patch-wise version.
\end{rem}

\begin{description}
  \item[SOLVE]
Given a triangulation $\T$, this method computes
\[
  U = \texttt{SOLVE}(\T),
\]
with $U \in \VV(\T)$ the  Ritz-Galerkin solution in \emph{exact} arithmetic. The effects of using an inexact solver have
been studied in the literature as well, see for example \cite{carstensen2014axioms}.
\item[ESTIMATE]
  Given a triangulation $\T$ with vertices $\V$ and its discrete solution $U \in \VV(\T)$, this module 
calculates the total error indicators --- the equilibrated flux estimator plus the oscillation. We have
\[
  \set{\vartheta_\T(U, \w_a)}_{a \in \V} = \texttt{ESTIMATE}(U, \T).
\]
\item[MARK]
A \emph{D\"orfler Marking} strategy \cite{dorfler1996convergent} is used for marking. For a parameter $\theta \in (0,1]$, we calculate
\[
  \mathcal{M} = \texttt{MARK}( \set{\vartheta_\T(U, \w_a)}_{a \in \V}, \T),
\]
where $\mathcal{M} \subset \W$ is the \emph{minimal} cardinality set that satisfies
\[
  \vartheta_\T(U,\mathcal{M}) \geq \theta \vartheta_\T(U, \W).
\]
In contrast to the standard residual case in \S\ref{sec:afem}, marking is done using the total error estimator.
\item[REFINE]
This module refines the marked elements in $\T$. It calculates
\[
  \T_\star = \texttt{REFINE}(\T, \mathcal{M}),
\]
with $\T_\star$ the smallest conforming refinement of $\T$ such that the triangles of
all patches in the marked set $\M$ are bisected at least 3 times. 
To provide conformity, one generally also needs to refine extra elements in that were not marked.

This three times refinement rule ensures the so-called \emph{interior node} property,
which is also used by Mekchay \& Nochetto \cite{mekchay2005convergence} and Morin et al. in \cite{morin2002convergence}.
The interior node property ensures that all the marked triangles, as well as their sides, contain a node of $\T_\star$ in their interior.
This condition is needed for  discrete efficiency. Casc\'on \& Nochetto \cite{cascon2012} avoid
refining every element three times by  spreading the refinements across consecutive steps;
they ensure that the interior node property is satisfied after a fixed number of refinement steps.
\end{description}

We can now formulate the AFEM algorithm and its produced sequences. For this we use the
iteration counter $k$ as a subscript to differentiate among the sets.
Let an initial triangulation $\T_0$ be given, set $k = 0$ and iterate the following steps:
\begin{enumerate}
\item $U_k = \texttt{SOLVE}(\T_k)$;
\item $\set{\vartheta_k(U_k, \w_a)}_{a \in \V_k} = \texttt{ESTIMATE}(U_k, \T_k)$;
  \item $\mathcal{M}_k = \texttt{MARK}(\set{\vartheta_k(U_k, \w_a)}_{a \in \V_k}, \T_k)$;
  \item $\T_{k+1} = \texttt{REFINE}(\T_k, \mathcal{M}_k)$;
  \item $k  := k + 1$.
\end{enumerate}

\section{Optimality conditions}
Casc\'on \& Nochetto provide a general list of assumptions in \cite[\S 4]{cascon2012}, under which the above AFEM algorithm produces
an optimal sequence of solutions $(U_k, \T_k)_{k \geq 0}$.
They do not prove that the equilibrium flux estimator actually  satisfies these assumptions.
In Lemma \ref{lem:assumptions} we will prove that these
assumptions indeed hold for the equilibrium flux estimator.
In AFEM driven by the standard residual estimator, we had a refined set containing the triangles that were refined.
In the current case, however, we need to define similar sets, but then in terms of refined patches.

First, let $\RR^0_{\T \to \T_\star}$ denote the set of all patches in $\T$ that contain at least \emph{one} triangle that is refined when going to $\T_\star$, i.e.
\[
\RR^0_{\T \to \T_\star} := \W \setminus \W_\star =  \set{ \w_a \in \W :  K \not \in \T_\star\quad \text{ for some }  K \subset \w_a}.
\]
For $j \geq 1$, the set $\RR^j_{\T \to \T_\star}$ consists of all patches $\w_a$ such that every triangle in $\w_a$ is bisected at least $j$  times when going from $\T$ to $\T_\star$.
Formally, 
\[
  \RR^j_{\T \to \T_\star} := \set{ \w_a \in \W : g(K') - g(K) \geq j,  \, \forall K' \in \T_\star \text{ with } K' \subset K, \, \forall K \in \T \text{ with } K \subset \w_a},
\]
where $g(K)$ is the generation of $K$ --- the number of bisections needed to create $K$ from the initial triangulation $\T_0$. For $j=1$ this set contains
all patches that are \emph{entirely} refined, for $j=3$ it contains all patches such that every triangle satisfies the interior node property.
Notice that $\RR^0_{\T \to \T_\star} \subset \RR^1_{\T \to \T_\star} \subset \dots$ . Moreover, we let $\O_\RR^j \subset \O$ denote the domain
spanned by the patches $\w_a \in \RR^j_{\T \to \T_\star}$.
With this extra notation we are able to state and prove the assumptions for optimality given in \cite{cascon2012}.
\begin{lem}
  \label{lem:assumptions}
  Let a triangulation $\T$ and a refinement $\T_\star \geq \T$ be given, with $U \in \VV(\T)$
  and $U_\star \in \VV(\T)$ the appropriate discrete solutions. There
  exists constants $C_1, C_2, C_3$ such that:
  \begin{enumerate}
    \item The estimator is reliable; the approximation error can be bound using the total error estimator:
      \begin{equation}
        \label{eq:globupper}
        \enorm{ u - U}_{\O}^2 \leq C_1 \vartheta^2_\T(U, \W) = C_1 \left [ \eta^2_\T(U, \W) + \osc^2_\T(f, \W)\right]
      \end{equation}
    \item The estimator is efficient; the estimator provides a lower bound for the error (up to oscillation):
      \begin{equation}
        \label{eq:globlower}
        C_2 \eta^2_\T(U, \W) \leq \enorm{u - U}^2_{\O} + \osc_\T^2(f, \W).
      \end{equation}
    \item The estimator provides discrete reliability; the error between $U$ and $U_\star$ can be bound using the
      refined set $\RR^0_{\T \to \T_\star}$:
      \begin{equation}
        \label{eq:locupper}
        \enorm{U_\star - U}^2_{\O}  \leq C_1\vartheta^2_\T(U, \RR^0_{\T \to \T_\star}).
      \end{equation}
    \item The estimator provides discrete efficiency, i.e.~the reverse of the above
      \begin{equation}
        \label{eq:loclower}
        C_3 \eta^2_\T(U, \RR^3_{\T \to \T_\star}) \leq \enorm{U_\star - U}^2_{\O_{\RR^3}}  + \osc_\T^2(U, \RR^1_{\T \to \T_\star}).
      \end{equation}
  \end{enumerate}
\end{lem}
\begin{proof}  ~
  \begin{subproof}[Reliability and Efficiency]
  Reliability follows directly from Theorem \ref{thm:discsynge}:
  \[
    \enorm{u - U}_{\O}^2 \lesssim \sum_{a\in \V}\uanorm{\vsiga}^2_{\w_a} + \uanorm{h_a(I - \Pi_{p-1}^a)(f)}_{\w_a}^2 =\sum_{a \in \V} \eta_\T^2(U,\w_a) + \osc_\T^2(f, \w_a) = \vartheta^2_\T(U, \W).
  \]
  Efficiency follows from Theorem~\ref{thm:discbounds}.
\end{subproof}

\begin{subproof}[Discrete reliability]
  Discrete reliability can be proven using equivalence with the classical residual estimator. 
  We opt not to follow this road, and instead give the following direct proof, inspired by \cite[\S3.2]{cascon2012}.
 Write $E_\star = U_\star - U$ for the discrete error and consider a $V \in \VV(\T)$ to be specified later. Notice that $\VV(\T) \subset \VV(\T_\star)$, and thus from Galerkin orthogonality we deduce
 \begin{equation}
   \label{eq:beginn}
        \enorm{E_\star}^2_{\O} = a(E_\star, E_\star - V) = a(u - U, E_\star - V) = \ip{r, E_\star - V} = \sum_{a \in \V} \ip{r_a, E_\star - V}
      \end{equation}
      The second equality holds since $U_\star$ is the Galerkin approximation on $\VV(\T_\star) \ni E_\star - V$.
      From orthogonality of $r_a$ on constants for $a \in \V^{int}$, we infer that
      \begin{equation}
        \label{eq:residualthing}
        \sum_{a \in \V} \ip{r_a, E_\star - V} \leq \sum_{a \in \V}\norm{r_a}_{H_\star^1(\w_a)'} \norm{\nabla \left (E_\star - V\right)}_{\w_a}.
      \end{equation}
      We will construct $V$ using the Scott-Zhang interpolant \cite{scott1990finite} of $E_\star$.

      Consider $\RR := \RR^0_{\T \to \T_\star}$, the set of patches in $\T$ such that at least \emph{one} triangle is refined when going from $\T$ to $\T_\star$. 
      The union of patches in $\RR$ make up a domain $\O_\RR \subset \O$.
      Let~$\O_k$ be one of the connected components of the interior of $\O_\RR$, and consider $\VV_k(\T)$ the restriction of $\VV(\T)$ to $\O_k$.
      We use the Scott-Zhang interpolation operator onto the space~$\VV_k(\T)$, i.e.~ $I_k : H^1(\O_k) \to \VV_k(\T)$ as defined in \cite[2.13]{scott1990finite}.
      An important property of $I_k$ is that it preserves conforming boundary values: for $v \in H^1(\O_k)$ with $v = \tilde v$ on~$\partial \O_k$ for some $\tilde v \in \VV_k(\T)$, we have $I_k v = v$ on $\partial \O_k$.
      Now define $V$ by
      \[
        V := I_kE_\star \quad \text{in } \O_k, \quad V:= E_\star \quad \text{elsewhere}.
      \]
      Since $E_\star$ has conforming boundary values on $\partial \O_k$, we infer that $V$ is continuous. Furthermore,
      as $\O_\RR$ is the union all the patches that were refined, we conclude that $V \in \VV(\T)$. 

      Plug this specific $V$ into equation \eqref{eq:residualthing} and use that $V = E_\star$ outside $\O_\RR$ to discover
      \begin{equation}
        \label{eq:magic}
      \begin{aligned}
        \sum_{a \in \V} \norm{r_a}_{H_\star^1(\w_a)'} \norm{\nabla \left (E_\star - V\right)}_{\w_a} &= \sum_{\w_a \in {\RR}} \norm{r_a}_{H_\star^1(\w_a)'} \norm{\nabla \left (E_\star - V\right)}_{\w_a} \\
        &\leq \sqrt{\sum_{\w_a \in \RR} \norm{r_a}_{H_\star^1(\w_a)'}^2} \sqrt{\sum_{\w_a \in \RR} \norm{\nabla \left(E_\star - V\right)}^2_{\w_a}}.
      \end{aligned}
    \end{equation}
      Focus on this last factor. We have ${E_\star - V = (I - I_k)(E_\star)}$ on every connected component $\O_k$.
      From the Scott-Zhang interpolation theory (cf.~\cite[Thm~4.1]{scott1990finite}) we know that
      $\norm{(I - I_k)(E_\star)}_{H^1(\O_k)} \lesssim \norm{E_\star}_{H^1(\O_k)}$, for a constant only depending on shape regularity.
      Since every triangle in $\O_\RR$ is contained in a maximum of three patches we deduce
      \begin{align*}
        \sum_{\w_a \in \RR} \norm{\nabla \left(E_\star  - V\right)}^2_{\w_a} &\leq \sum_{\w_a \in \RR} \norm{E_\star - V}^2_{H^1(\w_a)} \\
        &\eqsim \sum_{\O_k \subset \O_\RR} \norm{E_\star - V}^2_{H^1(\O_k)} \\
        &\lesssim \sum_{\O_k \subset \O_\RR} \norm{E_\star}^2_{H^1(\O_k)} \\
        &\leq \norm{E_\star}^2_{H^1(\O)} \lesssim \enorm{E_\star}^2_{\O}. 
      \end{align*}
    The last  inequality follows since $U, U_\star \in H_0^1(\O)$ and thus that $E_\star \in H_0^1(\O)$, which allows us to invoke Poincare-Friedrich's inequality.
    Insert this derivation into \eqref{eq:magic}, and chain its result with equations \eqref{eq:beginn} and \eqref{eq:residualthing} to obtain
    \[
      \enorm{E_\star}^2_\O \leq \sum_{a \in \V} \norm{r_a}_{H_\star^1(\w_a)'} \norm{\nabla \left (E_\star - V\right)}_{\w_a} \lesssim \sqrt{\sum_{\w_a \in \RR} \norm{r_a}_{H_\star^1(\w_a)'}^2} \enorm{E_\star}_{\O}.
    \]
    For the first factor in this equality we can use that $\norm{r_a}_{H^1_\star(\w_a)'} \leq \uanorm{\vsiga}_{\w_a} + \osc_\T(f,\w_a)$, as shown
    in the proof of Lemma~\ref{lem:clasequivlow}. 
    Standard calculations then prove discrete reliability:
    \[
      \enorm{U_\star - U}_{\O}^2 \lesssim  \sum_{w_a \in \RR} \norm{r_a}^2_{H_\star^1(\w_a)'} \lesssim \sum_{\w_a \in \RR} \uanorm{\vsiga}^2_{\w_a} + \osc^2_\T(f,\w_a) = \vartheta_\T^2(U, \RR). \qedhere
    \]
  \end{subproof}

  \begin{proof}[Discrete  efficiency]
    Discrete efficiency can be proven using equivalence with the classical residual estimator.
  Write $\RR^j := \RR^j_{\T \to \T_\star}$, and invoke Lemma \ref{lem:locequivosc} for equivalence with the standard residual estimator:
  \begin{equation*}
    \eta_\T^2(U,  \RR^3) = \sum_{\w_a \in  \RR^3} \uanorm{\vsiga}_{\w_a}^2\lesssim \sum_{\w_a \in  \RR^3} \left[ \osc_\T^2(f, \w_a) + h^2_{\w_a} \norm{f + \Delta U}^2_{\w_a} + h_{\w_a} \norm{\llbracket \nabla U \rrbracket}^2_{\gamma_a} \right].
  \end{equation*}
  Focus on the last two terms of this expression.  Every patch $\w_a \in \RR^3$ consists entirely of triangles that satisfy the interior node property.
  This implies that both triangular neighbours of an edge $e \subset\gamma_a$ satisfy the interior node property.
  For such an edge $e$, there exists a discrete lower bound using the classical \emph{edge} estimator (\cite[Thm~4.3]{stevenson2007optimality}  or \cite[Lem~4.2]{morin2000data}):
  \[
    \sum_{K \in \T_e} h_K^2 \norm{f + \Delta U}_K^2 + h_e\norm{\llbracket U\rrbracket}^2_e \lesssim \sum_{K \in \T_e} \left[\enorm{U_\star - U}_K^2 +  h_K^2\norm{(I - \Pi_{p-1})(f)}^2_K\right],
  \]
  where $\T_e$ consists of the two triangles $K \in \T$ adjacent to the (interior) edge $e \subset \gamma_a$. 

  Using that $h_K \eqsim h_{\w_a} \eqsim h_e$ for every triangle $K \subset \w_a$ and edge $e \subset \gamma_a$, allow us to relate the classical edge estimator to its patch-wise version:
  \[
    h_{\w_a}^2\norm{f + \Delta U}^2_{\w_a} + h_{\w_a} \norm{\llbracket \nabla U \rrbracket}^2_{\gamma_a} \eqsim
    \sum_{e \subset \gamma_a} \left[  \sum_{K \in \T_e} h^2_K \norm{f + \Delta U}_K^2 + h_e \norm{ \llbracket \nabla U \rrbracket}^2_{e}\right].
  \]
  After chaining  the above results, we obtain discrete efficiency:
  \begin{align*}
    \eta_\T^2(U,  \RR^3) &\lesssim \sum_{\w_a \in  \RR^3} \left[ \osc_\T^2(f, \w_a) + h^2_{\w_a} \norm{f + \Delta U}^2_{\w_a} + h_{\w_a} \norm{\llbracket \nabla U \rrbracket}^2_{\gamma_a} \right] \\
    &\eqsim \osc_\T^2(f, \RR^3) + \sum_{\w_a \in \RR^3} \sum_{e \subset \gamma_a}  \left[\sum_{K \in \T_e} h^2_K \norm{f + \Delta U}_K^2 + h_e \norm{ \llbracket \nabla U \rrbracket}^2_{e}\right]\\
    &\lesssim \osc_\T^2(f, \RR^3) + \sum_{\w_a \in \RR^3} \sum_{e \subset \gamma_a}  \sum_{K \in \T_e} \left[ \enorm{U_\star - U}_K^2 + h_K^2\norm{(I- \Pi_{p-1})(f)}^2_K\right]\\
    &\eqsim \osc_\T^2(f, \RR^3) + \sum_{\w_a \in \RR^3} \left[ \enorm{U_\star - U}^2_{\w_a} +  \norm{h_a(I- \Pi_{p-1})(f)}^2_{\w_a}\right]\\
    &\eqsim \osc_\T^2(f,\RR^3) + \enorm{U_\star - U}^2_{\O_{\RR}^3}. \qedhere
  \end{align*}
  \end{proof}
\let\qed\relax

\end{proof}
\section{Contraction property}
It is possible that the equilibrated flux estimator does not strictly decrease upon a refinement.
The oscillation term, on the other hand, does satisfy such a reduction property.
\begin{lem}
  \label{lem:oscasum}
  The oscillation satisfies the oscillation reduction property. That is, there exists a constant $0 < \lambda < 1$, such that for any refinement $\T_\star \geq \T$ one has
  \[
    \osc_{\T_\star}^2(f, \W_{\T_\star}) \leq \osc^2_{\T}(f, \W_{\T}) - \lambda \osc^2_{\T}(f, \RR^1_{\T \to \T_\star}).
  \]
\end{lem}
\begin{proof}
  Denote $\W = \W_{\T}$ and $\W_\star = \W_{\T_\star}$. Rewriting the left hand side yields,
  \begin{align*}
    \osc_{\T_\star}^2(f, \W_\star) &= \osc^2_{\T_\star}(f, \W_\star \cap \W) + \osc^2_{\T_\star}(f, \W_\star \setminus \W)\\
    &= \osc^2_{\T}(f, \W_\star \cap \W) + \osc^2_{\T_\star}(f, \W_\star \setminus \W)\\
    &= \osc^2_{\T}(f, \W) - \osc^2_{\T}(f, \W \setminus \W_\star) + \osc_{\T_\star}^2(f, \W_\star \setminus \W).
  \end{align*}
  We see that the result holds if there exists $0 < \lambda < 1$ such that
  \begin{equation}
    \label{eq:blabla}
    \osc_{\T_\star}^2(f, \W_\star \setminus \W) \leq \osc_{\T}^2(f, \W\setminus \W_\star) - \lambda \osc_\T^2(f, \RR^1_{\T \to \T_\star}).
  \end{equation}
  The set $\W_\star \setminus \W$ contains all newly introduced patches in $\T_\star$, whereas $\W \setminus \W_\star$ contains
  all patches that have at least one triangle refined.

  The use of patches instead of elements makes things a little confusing.  
  Therefore we will first analyze the above inequality using triangles.
  Consider some element $K$ and a bisected descendent $K_\star$ of $K$. Under the current definitions,
  we cannot ensure that $h_{K_\star} < h_K$, and we therefore cannot guarantee the strict oscillation reduction. 
  Alternatively, we could define $h'_K := (\vol K)^{1/2}$, for which we have $h'_{K_\star} \leq {1\over\sqrt{2}} h'_K$. 
  Shape regularity gives the equivalence $h'_K \eqsim h_K$, so that all the bounds in Lemma~\ref{lem:assumptions} are still
  valid if we replace $h_K$ by $h'_K$. To prove this lemma, we implicitly replace $h_K$ by
  $h'_K$ in the oscillation definition.

  First, consider
  Let $K$ be an element that is contained in some patch from  $\RR^1_{\T \to \T_\star}$.
  This element is refined, and from properties of the orthogonal projector $\Pi_{p-1}$ we infer that
  \begin{equation}
    \label{eq:strictosc}
  \begin{aligned}
    \sum_{\set{K_\star \in \T_\star : K_\star \subset K}} h^2_{K_\star} \norm{(I - \Pi_{p-1})(f)}^2_{K_\star} &\leq \frac{1}{2}h^2_K \sum_{\set{K_\star \in \T_\star : K_\star \subset K}} \norm{(I - \Pi_{p-1})(f)}^2_{K_\star} \\
    &\leq \frac{1}{2} h^2_K \norm{(I - \Pi_{p-1})(f)}^2_K.
  \end{aligned}
  \end{equation}
  That is, oscillation on $K$ is strictly reduced by refining $K$.
  Enlarging this to a single patch $\w_a \in \RR^1_{\T \to \T_\star}$ gives
  \begin{equation}
    \label{eq:singlepatch}
    \osc_\T^2(f, \w_a) = \sum_{K \subset \w_a} h_K^2\norm{(I-\Pi_{p-1})(f)}_K^2 \geq 2 \sum_{K \subset \w_a} \sum_{K_\star \subset K}h^2_{K_\star} \norm{(I-\Pi_{p-1})(f)}^2_{K_\star}.
  \end{equation}
  Unfortunately, the right hand side cannot be written as the sum of patch-wise oscillations in $\T_\star$.
  The difficulty is that refining $\w_a$ will introduce new patches in $\T_\star$, for which 
  some only partially overlap with $\w_a$. 
  
  It is wise to decompose all patch oscillations from \eqref{eq:blabla} over its triangles:
  \begin{align*}
    \osc_{\T}^2(f, \W \setminus \W_\star)  &= \sum_{\w_a \in \W \setminus \W_\star} \sum_{K \subset \w_a} h_{K}^2 \norm{(I-\Pi_{p-1})(f)}^2_{K}, \\
    \osc_{\T_\star}^2(f, \W_\star \setminus \W)  &= \sum_{\w_a \in \W_\star \setminus \W} \sum_{K_\star \subset \w_a} h_{K_\star}^2 \norm{(I-\Pi_{p-1})(f)}^2_{K_\star}.
  \end{align*}
  Examining these sums reveals that the total area covered by triangles $K$ in the upper sum is identical to the area covered by the (smaller) triangles $K_\star$ in the
  lower sum. For for every patch $\w_a \in \RR^1_{\T \to \T_\star}$, we may use \eqref{eq:singlepatch} to discover a strict decrease oscillation on the refined elements of $\w_a$.
  Carefully examining both of the above sums, and using this strict oscillation decrease, shows that the result is true with $\lambda = \frac{1}{2}$.
\end{proof}


We can now prove the so-called contraction property; this shows a weighted convergence of the AFEM method described above.
\begin{thm}
  There exists a constant $0 < \alpha < 1$ and $\gamma > 0$, depending on the shape regularity of $\T_0,$ and $\set{\lambda, \theta}$ such that
\[
  \enorm{ u - U_{k+1}}^2 +\gamma \osc^2_{k+1}(f, \W_{k+1}) \leq \alpha^2\left(\enorm{ u - U_k}^2 + \gamma\osc^2_k(f, \W_k)\right)
\]
\end{thm}
\begin{proof}
  Adapt the notation used in \cite{cascon2012}, i.e.~for $j =0,1,2,\dots$ we write
  \begin{alignat*}{2}
    e^2_j &:= \enorm{u - U_j}^2_{\O}, &\quad E^2_j &:= \enorm{U_{j+1} - U_j}_{\O}^2,\\
    \osc^2_j &:= \osc_{\T_{j}}^2(f, \W_{\T_{j}}), &\quad \osc^2_j(\mathcal{M}_j) &:= \osc^2_{\T_j}(f, \mathcal{M}_j), \\
    \eta^2_j &:= \eta_{\T_j}^2(U_j, \W_{\T_j}), &\quad \eta^2_j(\mathcal{M}_j) &:= \eta^2_{\T_j}(U, \mathcal{M}_j),\\
    \RR^i_j & := \RR^i_{\T_j \to \T_{j+1}}.
  \end{alignat*}

  A piecewise polynomial on $\T_{k}$ is also a piecewise polynomial on $\T_{k+1}$, because ${\T_{k+1} \geq \T_k}$.
  Therefore, we have $U_{k+1} - U_{k} \in \VV(\T_{k+1})$, and thus by Galerkin orthogonality 
  \[
    \enorm{u - U_k}_{\O}^2 = \enorm{u - U_{k+1}}^2_{\O} + \enorm{U_{k+1} - U_{k+1}}^2_{\O} \implies e_{k+1}^2 = e_k^2 - E_k^2.
  \]
  Introduce constants $\gamma > 0$ and $0 < \beta < 1$ that will be selected later:
  \[
    e_{k+1}^2 + \gamma \osc_{k+1}^2 = e_k^2 - E_k^2 + \gamma \osc_{k+1}^2 \leq e_k^2 - \beta E_k^2 + \gamma \osc_{k+1}^2.
  \]
  Application of the oscillation reduction property from Lemma \ref{lem:oscasum} results in
  \begin{equation}
    \label{eq:ineq1}
  \begin{aligned}
    e_{k+1}^2 + \gamma \osc^2_{k+1} \leq e_k^2 - \beta E_k^2 + \gamma \osc_k^2 - \gamma\lambda \osc_k^2(\RR^1_k).
      %e_{k+1}^2 + \gamma \osc^2_{k+1} \leq e_k^2 
    %e_{k+1}^2 + \gamma \osc^2_{k+1} &= e_k^2 - E_k^2 + \gamma \osc^2_{k+1} \\
    %&\leq e_k^2 - E_k^2 + \gamma \left[\osc_k^2 - \lambda  \osc^2_k(\RR_k) \right].
  \end{aligned}
\end{equation}
Discrete efficiency \eqref{eq:loclower} from Lemma \ref{lem:assumptions} reads as
  \[
    E_k^2 = \enorm{U_{k+1} -  U_k}_\O \geq C_3 \eta_k^2(\RR^3_k) - \osc_k^2(\RR^1_k) \geq C_3 \eta_k^2(\mathcal{M}_k) - \osc_k^2(\RR^1_k),
  \]
  where the last inequality follows since the all marked patches satisfy the interior node property when going from $\T_k$ to $\T_{k+1}$,
  and thus $\M_k \subset \RR^3_k$.
  By inserting this into inequality \eqref{eq:ineq1} we obtain
  \begin{align*}
    e_{k+1}^2 + \gamma \osc^2_{k+1} &\leq e_k^2 - \beta C_3 \eta_k^2(\M_k) + \beta \osc_k^2(\RR^1_k) + \gamma \osc_k^2 - \gamma\lambda  \osc^2_k(\RR^1_k) \\
  &= e_k^2 + \gamma \osc_k^2 - \beta C_3\eta_k^2(\M_k) - \left[\lambda \gamma - \beta\right]\osc^2_k(\RR^1_k).
  \end{align*}
  Under the assumption that $\lambda \gamma \geq \beta$ we may replace $\osc_k^2(\RR^1_k)$ by the smaller $\osc_k^2(\mathcal{M}_k)$:
  \[
    e_{k+1}^2 + \gamma \osc^2_{k+1} \leq e_k^2 + \gamma \osc_k^2 - C_3 \beta \eta_k^2(\M_k) - \left[\lambda \gamma - \beta\right]\osc^2_k(\M_k).
  \]
  Now pick $\beta$ such that  the coefficients of $\osc_k^2(\M_k)$ and $\eta_k^2(\M_k)$ match, i.e.
  \[
    \beta C_3 = \lambda \gamma - \beta \implies \beta := \frac{1 }{1 + C_3}\lambda \gamma,
  \]
  which ensures that $\lambda \gamma \geq \beta$. Substitute $\beta$ and use the definition $\vartheta_k^2 = \eta_k^2 + \osc_k^2$ to attain
  \begin{align*}
  e_{k+1}^2 + \gamma \osc^2_{k+1} &\leq e_k^2+\gamma \osc_k^2 - \frac{C_3}{1+C_3} \lambda \gamma \vartheta_k^2(\M_k) \\
    &\leq e_k^2 + \gamma \osc_k^2 - \frac{C_3}{1 + C_3} \lambda \gamma \theta^2 \vartheta_k^2,
    %- \left[\lambda \gamma - 1 - C_3\right]\osc^2_k(\M_k) \\
    % &\leq e_k^2 + \gamma \osc_k^2 - C_3 \vartheta_k^2(\M_k)\\
    % &\leq e_k^2 + \gamma \osc_k^2 - C_3 \theta^2 \vartheta_k^2,
  \end{align*}
  with the last step following from the D\"orfler marking property, i.e.~$\vartheta_k^2(\mathcal{M}_k) \geq \theta^2 \vartheta_k^2$. 
  Since the total error dominates oscillation, $\vartheta_k^2 \geq \osc_k^2$, we infer that
  \[
    e_{k+1}^2 + \gamma \osc_{k+1}^2 \leq e_k^2 + \gamma \osc_k^2 - \frac{C_3}{2(1 + C_3)} \lambda \gamma \theta^2\left(\vartheta_k^2 + \osc_k^2\right).
  \]
  Rewriting the (global) reliability bound \eqref{eq:globupper} shows $\vartheta_k^2 \geq C_1^{-1} e_k^2$, and thus
  \begin{align*}
    e_{k+1}^2 + \gamma \osc_{k+1}^2 &\leq \left(1 - \frac{C_3 \lambda \theta^2}{2C_1(1+C_3)}\gamma\right) e_k^2 + \gamma \left( 1 - \frac{C_3}{2(1 + C_3)}\lambda\theta^2\right)\osc_k^2\\
    &:= \alpha^2_1(\gamma) e_k^2 + \gamma \alpha_2^2 \osc_k^2.
  \end{align*}
  Since $\lambda$ and $\theta^2$ are both contained in the interval $(0,1)$ we have $0 < \alpha^2_2 < 1$. We are left to
  pick $\gamma$ such that $0 < \alpha_1^2(\gamma) < 1$, which translates into
  \[
    0 < \gamma < \frac{2C_1(1 + C_3)}{C_3\lambda \theta^2}.
  \]
  Any $\gamma$ satisfying the above completes the proof, since $\alpha^2 = \max\set{\alpha_1^2(\gamma), \alpha_2^2} < 1$ with
  \[
    e_{k+1}^2 + \gamma \osc_{k+1}^2 \leq \alpha_1^2(\gamma)e_k^2 + \gamma \alpha_2^2\osc_k^2 \leq \alpha^2 \left( e_k^2 + \gamma \osc_k^2\right).
  \]

\end{proof}


\section{Optimality of AFEM}
\todo{bla}
We will show that convergence of AFEM driven by the equilibrated flux estimator convergence with the best possible rate --- and
thus  optimal in this sense.
From the reliability and efficiency of the estimator we know that
\[
  \enorm{u - U}^2_\O + \osc^2_{\T}(f, \W) \eqsim \eta_\T^2(U, \W) + \osc_\T^2(f, \W) = \vartheta_\T^2(U, \W).
\]
The total error estimator is equivalent to the approximation error (up to an oscillation term).
Since we cannot get rid of the oscillation term (cf.~\cite[Rem~5.1]{cascon2008}), it makes sense to incorporate this into the
definition of an approximation class.

Denote $\TT$ for the set of all conforming refinements of $\T_0$ found by applying bisection. Moreover, let $\TT_N \subset \TT$ be the restriction of
triangulations with at most $N$ triangles more than $\T_0$, i.e.~$\T \in \TT_N$ if $\T \leq \#T_0 + N$.
We define the optimal approximation class in terms of $s > 0$. For $v \in H_0^1(\O)$ with $\Delta v \in L^2(\O)$, let
$V_\T \in \VV(\T)$ denote the Ritz-Galerkin approximation of the Poisson problem associated with $v$.
A measure for the optimal convergence rate of $v$ is given by
\[
  \abs{v}_{\A_s} := \sup_{N > 0}(N+1)^s \inf_{\T \in \TT_N} \left(\enorm{v - V_\T}_{\O}^2 + \osc_\T^2(f, \W)\right)^{1/2}.
\]
As approximation class we take all the functions for which this measure is finite, i.e.
\[
  \A_s := \set{ v \in H_0^1(\O) : \Delta v \in L^2(\O), \quad  \abs{v}_{\A_s} < \infty}.
\]
So if the Poisson solution $u$  satisfies $u \in \A_s$, then the total 
error $e(N)$ of the discrete solution $U_N \in \VV(\T)$ on the \emph{best} partition $\T_N \in \TT_N$ satisfies
\begin{equation}
  \label{eq:errorafem}
  e(N) := \left(\enorm{u - U_N}^2_\O + \osc^2_{\T_N}(f,\W)\right)^{1/2} \leq (N+1)^{-s} \abs{u}_{\A_s}.
\end{equation}
Our goal is to prove that the sequence $\left(U_k\right)_{k \geq 0}$ satisfies a similar convergence rate.

  Notice that this optimality class explicitly depends on our definition of oscillation, and thus differs from the definition for the
  classical estimator given in \S\ref{sec:optimalafem}. Naturally these definitions are equivalent, as one can see from
  \[
    \osc^2_\T(f,\W) = \sum_{\w_a \in \W} \sum_{K \subset \w_a} h_K^2 \norm{(I - \Pi)(f)}_K^2 = 3 \sum_{K \in \T} h_K^2 \norm{(I - \Pi)(f)}_K^2 = \hat \osc^2_\T(f,\T).
  \]

The proof of optimality for AFEM driven by the equilibrated flux estimator is very similar to the standard case in \S\ref{sec:optimalafem}. 
We opt to follow the steps given by Casc\'on and Nochetto in \cite{cascon2012}.
Define the maximum D\"orfler marking parameter by
\begin{equation}
  \label{eq:theta}
  \theta_*^2 := \frac{C_2}{(1+ C_1)(1 + C_2)}.
\end{equation}
The following lemma establishes a relation between the  total error reduction and the D\"orfler marking. 
\begin{lem}
\label{lem:dorfler}
  Let $\T \in \TT$ a triangulation with discrete solution $U \in \VV(\T)$, a marking parameter $\theta \in (0, \theta_*)$ and 
  a refinement $\T_\star \geq \T \in \TT$ with discrete solution $U_\star \in \VV(\T_\star)$ that satisfies
  \[
    \enorm{ u - U_\star}_\O^2+ \osc^2_{\T_\star}(f, \W_\star) \leq \mu (\enorm{u - U}_\O^2 + \osc_\T^2(f, \W)),
  \]
  for $\mu := 1 - {\theta^2 \over \theta_*^2}$.

  Then the refined set $\RR^0 = \RR^0_{\T \to \T_\star}$ satisfies the D\"orfler property
  \[
    \vartheta_\T(U, \RR^0) \geq \theta \vartheta_\T(U, \W).
  \]
\end{lem}
\begin{proof}
  From the efficiency bound \eqref{eq:globlower} in Lemma \ref{lem:assumptions} we deduce
  \[
    C_2 \vartheta_\T^2(U, \W) \leq \enorm{u - U}_\O^2 + (1+C_2)\osc_\T^2(f, \W) \leq (1 + C_2)\left(\enorm{u - U}_\O + \osc_\T^2(f, \W)\right),
  \]
  and thus
  \[
    {C_2 \over 1 + C_2} \vartheta_\T^2(U, \W) \leq \enorm{u - U}_\O^2 + \osc_\T^2(f,\W).
  \]
  Together with the assumption this becomes
  \begin{equation}
    \label{eq:maineq}
  \begin{aligned}
    (1 - \mu){C_2 \over 1 + C_2} \vartheta_\T^2(U, \W) &\leq (1 - \mu) \left[ \enorm{u - U}_\O^2 + \osc_\T^2(f, \W)\right]\\
    &\leq \enorm{u - U}_\O^2 + \osc_\T^2(f,\W) - \enorm{u - U_\star}_\O^2 - \osc_{\T_\star}^2(f, \W_\star).
  \end{aligned}
\end{equation}
  Using the Galerkin orthogonality and $\RR^0 = \W \setminus \W_\star$ shows
  \begin{align*}
    \enorm{U_\star - U}^2_\O &= \enorm{u - U}^2_\O - \enorm{u - U_\star}^2_\O,\\
    \osc_\T^2(f, \W) &= \osc_{\T_\star}^2(f, \W \cap \W_\star) + \osc_{\T_\star}^2(f, \RR^0) \leq \osc_{\T_\star}^2(f, \W_\star) + \osc^2_{\T}(f, \RR^0).
  \end{align*}
  After inserting these relations into  \eqref{eq:maineq} we infer that
  \begin{align*}
    (1 - \mu){C_2 \over 1 + C_2} \vartheta_\T^2(U, \W) &\leq \enorm{U_\star - U}_\O^2 + \osc^2_{\T}(f, \RR^0) \\
    &\leq \enorm{U_\star - U}_\O^2 + \vartheta_\T(U, \RR^0) \\
    &\leq (1 + C_1) \vartheta_\T^2(U, \RR^0),
  \end{align*}
  where the last step follows from discrete reliability \eqref{eq:locupper}.
  In conclusion, by definition of $\theta_*^2$ and $\mu$ we see that
  \[
  \vartheta_\T^2(U, \RR^0) \geq  (1 - \mu){C_2 \over (1 + C_1)(1 + C_2)}\vartheta_\T^2(U, \W) \geq \theta^2 \vartheta_\T^2(U, \W). 
  \]
\end{proof}
The following corollary relates the triangulations produced by AFEM with the optimal triangulations. This hinges on the fact that AFEM selects 
a \emph{minimal} cardinality set $\M_k$. This key idea was first used by Stevenson in \cite{stevenson2007optimality}.
\begin{cor}
  Let $\set{\T_k, \VV_k, U_k}_{k \geq 0}$ be the sequence of results produced by AFEM for a marking parameter $\theta \in (0, \theta_*)$.
  If $u \in \A_s$ then the minimal D\"orfler set $\M_k$ satisfies
  \[
    \M_k \lesssim \left(1 - \frac{\theta^2}{\theta^2_*}\right)^{-1 / 2s} \abs{u}_{\A_s}^{1/s} \left(\enorm{u - U_k}_\O^2 + \osc^2_{\T_k}(f, \W_k)\right)^{-1/2s}.
  \]
\end{cor}
\begin{proof}
  Let $\mu$ be as in the previous Lemma and set $\epsilon^2 := \mu \left( \enorm{u - U_k}_\O^2 + \osc_{\T_k}^2(f, \W_k)\right)$.
  Since $u \in \A_s$, by definition of the approximation class there exists a triangulation $\T_\epsilon \in \TT$ with
  discrete solution $U_\epsilon$ such that\footnote{To see this, let $e(N)$ denote the smallest error on a triangulation from $\TT_N$. This function is decreasing, since $\TT_N \subset \TT_{N+1}$, and we have $e(N) \to 0$ as $N \to \infty$. Therefore, there exists a $N$ such that $e(N) \leq \epsilon \leq e(N-1)$. From the definition we also find
  $N^s e(N-1) \leq \abs{u}_{\A_s}$, and thus $N \leq e(N-1)^{-1/s} \abs{u}_{\A_s}^{1/s} \leq \epsilon^{-1/s}\abs{u}_{\A_s}^{1/s}$.}
  \[
    \# \T_\epsilon - \# \T_0 \leq \abs{u}_{\A_s}^{1/s} \epsilon^{-1/s}, \quad \text{and } \enorm{u - U_\epsilon}_\O^2 + \osc^2_{\T_\epsilon}(f, \W_{\epsilon}) \leq \epsilon^2.
  \]
  For $\T_\star = \T_k \oplus \T_\epsilon$ --- the smallest common conforming refinement of $\T_k$ and $\T_\epsilon$ ---  
  we find from the Galerkin orthogonality and the oscillation reduction property that
  \begin{align*}
    \enorm{u - U_\star}_\O^2 + \osc^2_{\T_\star}(f, \W_\star) \leq \enorm{u - U_\epsilon}_\O^2 + \osc^2_{\T_\epsilon}(f,\W_\epsilon) \leq \epsilon^2.
  \end{align*}
  By the choice of $\epsilon^2$ it follows that $u - U_\star$ satisfies the conditions of the previous lemma, and thus that $\RR^0_{\T_k \to \T_\star}$ satisfies the D\"orfler property. Since $\mathcal{M}_k$ is the \emph{minimal} cardinality set that satisfies
  the D\"orfler property we may conclude that $\# \mathcal{M}_k \leq \# \RR^0_{\T_K \to \T_\star}$.
  
  Now $\RR^0_{\T_k \to \T_\star}$ consists of all those patches containing a refined triangle going from $\T_k$ to $\T_\star$. Since
  every patch has a uniformly bounded number of triangles, with a bound only depending on $\T_0$, we conclude
  \[
    \# \mathcal{M}_k \leq \# \RR^0_{\T_k \to \T_\star} \lesssim \# \T_\star - \# \T_k \leq \# \T_\epsilon - \# \T_0 \leq \abs{u}_{\A_s}^{1/s}\epsilon^{-1/s}.
  \]
  The third inequality is a property of the smallest common refinement \cite[Lem~3.7]{cascon2008}.
\end{proof}
We are almost ready to prove the main optimality result. For this we need to bound the number of refined triangles. 
This number can inflate beyond $\#\M$ in one iteration due to the conformity constraint. Fortunately, the cumulative sum behaves correctly as summarized in the following theorem.
\begin{thm}
  For a constant only depending on shape regularity we have
\begin{equation}
  \label{eq:nvbound}
  \# \T_k - \# \T_0 \lesssim \sum_{i = 0}^{k-1} \# \mathcal{M}_i.
\end{equation}
\end{thm}
\begin{proof}
  A proof for the classical (element-wise) estimator is given in \cite{ste08}. In this case the marked sets $\mathcal{\hat M}_i$ consist \emph{triangles} that
  are bisected once. We however consider marked sets $\M_i$ consisting of \emph{patches} that are to be bisected three times. The proof in \cite{ste08}
  is every technical and therefore omitted here. We only remark that the main arguments are also valid for our case, hence providing \eqref{eq:nvbound}.
\end{proof}
We are finally ready to prove the main optimality theorem.
\begin{thm}
  Suppose that the marking parameter $\theta$ satisfies $\theta \in (0, \theta_*)$ (see \eqref{eq:theta}),
  and suppose that $u \in \A_s$ for some $s >0$.

  Then the produced sequence $\set{U_k, \VV_k \T_k}_{k \geq 0}$ of AFEM-solutions satisfies 
  \[
    \left(\enorm{u - U_k}^2_\O + \osc^2_k(f,\W_k)\right)^{1/2} \lesssim \abs{u}_{\A_s} \left(\#\T_k - \#\T_0\right)^{-s},
  \]
  for a constant depending on $\T_0$, but independent of $u$.
\end{thm}
\begin{proof}
Combine the contraction property, the previous corollary and inequality \eqref{eq:nvbound}:
\begin{align*}
  \# \T_k - \# \T_0 &\lesssim \sum_{i = 0}^{k-1} \# \mathcal{M}_i \\
  &\lesssim \left(1 - \frac{\theta^2}{\theta^2_*}\right)^{-1 / 2s} \abs{u}_{\A_s}^{1/s} \sum_{i=0}^{k-1} \left(\enorm{u - U_i}_\O^2 + \osc^2_{\T_i}(f, \W_i)\right)^{-1/2s}\\
  &\eqsim \left(1 - \frac{\theta^2}{\theta^2_*}\right)^{-1 / 2s} \abs{u}_{\A_s}^{1/s} \sum_{i=0}^{k-1} \left(\enorm{u - U_i}_\O^2 + \gamma\osc^2_{\T_i}(f, \W_i)\right)^{-1/2s}\\
  &\leq \left(1 - \frac{\theta^2}{\theta^2_*}\right)^{-1 / 2s} \abs{u}_{\A_s}^{1/s} \left(\sum_{i=1}^{k} \alpha^{i \over s}\right) \left(\enorm{u - U_k}_\O^2 + \gamma\osc^2_{\T_k}(f, \W_k)\right)^{-1/2s}\\
  &\lesssim \abs{u}_{\A_s}^{1/s}\left(\enorm{u - U_k}_\O^2 + \osc^2_{\T_k}(f, \W_k)\right)^{-1/2s}.
\end{align*}
Here we used that $\alpha < 1$, so that it forms a geometric series.
 Standard algebra calculations on this inequality provide us with the desired result.
\end{proof}
  Compare this result to the definition of the approximation class $\A_s$.
  The best possible error $e(N)$ on $\TT_N$ satisfies $e(N) \leq N^{-s}\abs{u}_{\A_s}$, see  \eqref{eq:errorafem}.
  By the previous theorem exists a constant $C_{opt}$ such that the error made by AFEM is bound by
  \[
    \left(\enorm{u - U_k}^2_\O + \osc^2_k(f, \W_k)\right)^{1/2} \leq C_{opt} \left(\#\T_k - \#\T_0\right)^{-s} \abs{u}_{\A_s}.
  \]
  Hence the AFEM provided solutions exhibit  the same optimal convergence rate.
  \section{Discussion}
  \label{sec:remarks}
  A few remarks can be added to the above optimality proof.
  \subsection{Interior node property}
  In the current proof we need the interior node property which requires \texttt{REFINE} to
  bisect the marked patches at least three times. 
  In practice one often wants to avoid this many refinements; one would like to just bisect every marked patch once.
  However, by letting \texttt{REFINE} apply single bisections we cannot guarantee that the local lower bound holds, which
  is needed to prove the contraction property of AFEM solutions.
  
  Casc\'on and Nochetto \cite{cascon2012} reduce the number of bisections by proposing a variant of the  \texttt{REFINE} method that
  ensures the interior node property after a fixed amount of steps. Although this requires less bisections in one step, it does require
  a non-standard \texttt{REFINE} implementation.
  
  Kreuzer \& Siebert \cite{kreuzersiebert} circumvent
  the interior node property altogether by using equivalence with the standard residual estimator and by requiring that
  the error estimator dominates the oscillation, i.e.~$\eta_\T(U, \w_a) \geq \osc_\T(f,\w_a)$.
  In the equilibrated flux estimator this does not hold for a general $f$. However, if one considers a Poisson problem
  without oscillation (or with a very smooth $f$), then this does hold. For such problems one can therefore 
  obtain an optimal convergence rate using a \texttt{REFINE} method that applies only a single bisection to the marked patches.

  \subsection{Element-wise estimator}
  We have proven  optimality of AFEM driven by the patch-wise estimators $\uanorm{\vsiga}_{\w_a}$.
  As noted before, we ideally would use the element-wise estimators $\uanorm{\vsig^\triangle}_K$. 
  We have proven that this element-wise variant $\uanorm{\vsig^\triangle}_K$ is reliable and efficient. We are left to prove  discrete reliability and discrete efficiency
  from Lemma~\ref{lem:assumptions}. For the patch-wise estimator we used that $\uanorm{\vsiga}_{\w_a}$
  is equivalent to the standard residual estimator on $\w_a$. It is not directly
  clear if such a relation also holds for $\uanorm{\vsig^\triangle}_K$.  Attempts to find an equivalence between $\uanorm{\vsiga}_{\w_a}$ and $\uanorm{\vsig^\triangle}_K$ also
  fail:
  \[
    \uanorm{\vsig^\triangle}_K \leq \sum_{a \in \V_K} \uanorm{\vsiga}_{\w_a},\quad \sum_{a \in \V_k} \uanorm{\vsiga}^2_{\w_a} = \uanorm{\vsig^\triangle}^2_K + \sum_{a \in \V_K} \uanorm{\vsiga}^2_{\w_a \setminus K} \not \lesssim \uanorm{\vsig^\triangle}^2_K.
  \]

  The discrete reliability can be directly proven using the Scott-Zhang local interpolator.
  For simplicity assume we do not have data oscillation. With $E_\star = U_\star - U$ and $V \in \VV(\T)$ we deduce from Galerkin orthogonality~that
  \[
    \enorm{E_\star}^2_{\O} = a(E_\star, E_\star - V) = a(u - U, E_\star - V) = \ip{r, E_\star - V} = \ip{\vsig^\triangle, \nabla V -\nabla E_\star}_\O.
  \]
  The second equality holds since $U_\star$ is the Galerkin approximation on ${\VV(\T_\star) \ni E_\star - V}$.
  The last equality follows from the equilibrated characteristic without oscillation, i.e.~$\ip{\vsig^\triangle, \nabla v} = -\ip{r,v}$.
  Let $V$ be the Scott-Zhang \cite{scott1990finite} interpolator of $E_\star$ so that $V - E_\star = 0$ on $\T \cap \T_\star$. 
  From the Scott-Zhang approximation theory \cite{scott1990finite} and Poincar\'e-Friedrichs inequality for $E_\star \in H_0^1(\O)$ we discover that
  \begin{align*}
    \ip{\vsig^\triangle , \nabla V - \nabla E_\star}_\O &= \sum_{K \in \T \setminus \T_\star} \ip{\vsig^\triangle, \nabla V - \nabla E_\star}_K\leq \sum_{K \in \T \setminus \T_\star} \uanorm{\vsig^\triangle}_K \norm{\nabla V - \nabla E_\star}_K\\
    &\lesssim \sum_{K \in \T \setminus \T_\star} \uanorm{\vsig^\triangle}_K\norm{E_\star}_{H^1(\w_K)}\\
    &\lesssim \norm{E_\star}_{H^1(\O)} \sqrt{\sum_{K \in \T \setminus \T_\star} \uanorm{\vsig^\triangle}^2_K} \\
    &\lesssim \enorm{E_\star}_{\O} \sqrt{\sum_{K \in \T \setminus \T_\star} \uanorm{\vsig^\triangle}^2_K}.
  \end{align*}
  This gives the desired local upper bound
  \[
    \enorm{U_\star - U}^2_\O \lesssim\sum_{K \in \T \setminus \T_\star} \uanorm{\vsig^\triangle}^2_K.
  \]

  For the discrete efficiency no such direct proof comes to mind, and without an equivalence  with the standard residual estimator 
  we cannot prove that it holds. Due to the similarities we \emph{conjecture} that this bound also holds. This would imply optimal convergence of AFEM
  driven by the element-wise equilibrated flux estimator as well.

  \subsection{Lower order estimator}
  In \S\ref{sec:lowerorder} we introduced an equilibrated flux estimator $\vhsiga$ in the lower order Raviart-Thomas $\RT_{p-1,0}^{-1}(\w_a)$. 
  The element-wise estimator $\norm{\v{\hat \sigma}}_K$ provides a reliability bound for which the oscillation is still of higher order. The
  efficiency result for this estimator introduced an extra edge-related oscillation term. As with the above, we conjecture that AFEM driven
  by this estimator is optimal as well. This claim is supported by the numerical results in Chapter 5.
\end{document}
