\documentclass[thesis.tex]{subfiles}
\begin{document}
  Suppose that we have an interpolant $I_\T$ that for $v \in H_0^1(\O)$ satisfies\footnote{We need a different interpolant here than the one given in the previous section, since this required $v$ to have extra smoothness. The so-call \emph{Scott-Zhang} interpolant does the trick (c.f. \cite[Ch~4.9]{brenner}).}
  \[
    \norm{v - I_\tau v}_{K} \leq C h_K \abs{v}_{H^1(\tilde K)}, \quad \norm{v - I_\tau v}_{e}\leq C h_K^{1/2}\abs{v}_{H^1(\tilde K)},
  \]
  for all elements $K \in \T$ and edges $e \in \E^{int}$, with $\tilde K$ the union of elements touching $K$\footnote{For $e \in \E^{int}$ let $h_K$ be for a $K \in \T$ such that $e$ is an edge of $K$.}.
  We will use the shorter notation $\norm{\cdot}_{\w}$ for the $L^2(\w)$-norm; notice that the edges $e$
  are one-dimensional, whilst elements $K$ are two-dimensional. 
  Since $I_\T v \in \VV(\T)$ we infer from the Galerkin orthogonality that
  \begin{align*}
    a(e, v) = a(e, v - I_\T v) \leq& \sum_{K \in \T} \| f + \Delta U\|_{K} \| v - I_\T v\|_{K} + \sum_{e \in \E^{int}} \|\llbracket \nabla U \rrbracket \|_{e} \|v - I_\T v\|_{e}\\
    \leq& \sum_{K \in \T} \norm{f + \Delta U}_K C h_K  \abs{v}_{H^1(\tilde K)} + \sum_{e \in \E^{int}} \norm{\llbracket \nabla U\rrbracket}_e C h_K^{1/2}\abs{v}_{H^1(\tilde K)}\\
    \leq &C \left( \sum_{K \in \T} h_K^2 \norm{f + \Delta U}_{K}^2\right)^{1/2}\left(\sum_{K \in \T}\abs{v}_{H^1(\tilde K)}^2\right)^{1/2} \\
    +&C \left(\sum_{e \in \E^{int}}h_K\norm{\llbracket \nabla U \rrbracket}^2_e\right)^{1/2} \left(\sum_{e \in \E^{int}} \abs{v}^2_{H^1(\tilde K)} \right)^{1/2}\\
    \leq &  \tilde C \abs{v}_{H^1(\O)} \left(\sum_{K \in \T} h_K^2\norm{f + \Delta U}_K^2 + \sum_{e \in \E^{int}} h_K \norm{\llbracket \nabla U\rrbracket}_e^2 \right)^{1/2},
  \end{align*}
  with the third inequality following from Cauchy-Schwarz for sequences. Using Poincar\'e-Friedrichs inequality we deduce
  \todo{Beter uitschrijven}
  \begin{equation}
    \label{eq:enormdual}
    \enorm{u - U}_{\O} = \sup_{0\ne v \in V}\frac{a(u - U, v)}{\abs{v}_{H^1(\O)}} \leq \tilde C \left(\sum_{K \in \T}h_K^2\norm{f +\Delta U}_K^2 + \sum_{e \in \E^{int}}h_K\norm{\llbracket U \rrbracket}_e^2 \right)^{1/2}.
  \end{equation}
\begin{lem}
  \label{lem:globrel}
  A global error bound in terms of $r_a$ is given by,
  \[
    \enorm{u - U}_{\O}^2 \leq 3 \sum_{a \in \V} \norm{r_a}^2_{H^1_\star(\omega_a)'}
  \]
\end{lem}
\begin{proof}
  Since $H_0^1(\O)$ is a Hilbert with respect to $ \ip{\nabla \cdot, \nabla \cdot}_\O$, we find by duality
  \[
    \enorm{u - U}_{\O} = \norm{\nabla u - \nabla U}_{\O} = \sup_{ v \in H_0^1(\O): v \ne 0 } {\abs{a(u - U, v)} \over \norm{\nabla v}_{\O}} = \sup_{ v \in H_0^1(\O): v \ne 0 } {\abs{\ip{r, v}} \over \norm{\nabla v}_{\O} }.
  \]
  Rewriting the residual operator using the partition of unity for $v \in H_0^1(\O)$ yields
  \begin{align*}
    \ip{r , v} &= \sum_{a \in \V^{int}} \ip{r_a, {v - v_{\omega_a}}} + \sum_{a \in \V^{bdr}} \ip{r_a, v} \\
    &\leq \sum_{a \in \V^{int}} \norm{r_a}_{H^1_\star(\omega_a)'} \norm{\nabla (v- v_{\omega_a})}_{\omega_a} 
    + \sum_{a \in \V^{bdr}} \norm{r_a}_{H^1_\star(\omega_a)'} \norm{\nabla v}_{\omega_a}\\
    &= \sum_{a \in V} \norm{r_a}_{H^1_\star(\w_a)'} \norm{\nabla v}_{\w_a}\\
    &\leq \sqrt{\sum_{a \in \V} \norm{\nabla v}_{\w_a}^2} \sqrt{\sum_{a \in \V} \norm{r_a}^2_{H^1_\star(\w_a)'}} \leq \sqrt{3}\norm{\nabla v}_{\O}\sqrt{\sum_{a\in \V} \norm{r_a}^2_{H_\star^1(\w_a)'}}.
  \end{align*}
  The first equality follows from orthogonality on the constants; the first inequality follows since $v - v_{\omega_a}, v \in H_\star^1(\w_a)$ 
  for an interior resp. boundary vertex $a$; and finally $\sum_{a \in \V} \norm{\nabla v}^2_{\omega_a} \leq 3\norm{\nabla v}_{\O}^2$ holds
  as every triangle is contained in a three patches. Combining these inequalities for a non-zero $v\in H_0^1(\O)$ yields
  \begin{align*}
    \norm{\nabla u - \nabla U}_{\O} \leq \norm{\nabla v}_{\O}^{-1} \ip{r,v} \leq \sqrt{3} \norm{\nabla v}_{\O}^{-1} \norm{\nabla v}_{\O} \sqrt{\sum_{a \in \V} \norm{r_a}_{H^1_\star(\omega_a)'}}.
  \end{align*}
\end{proof}
\begin{proof}
  From \eqref{eq:defsigma} we have for $v \in H_\star^1(\w_a)$ that $\ip{r_a, v} = - \ip{\v{\sigma_a}, \nabla v}_{\w_a}$.
  This directly provides us with the first inequality, i.e.
  \[
    \norm{r_a}_{H_\star^1(\omega_a)'} = \sup_{v \in H_\star^1(\w_a) : \norm{\nabla v}_{\w_a} = 1} \abs{\ip{r_a, v}} = \sup_{v \in H_\star^1(\w_a) : \norm{\nabla v}_{\w_a}=1} \abs{\ip{\v{\sigma_a}, \nabla v}_{\w_a}} \leq \uanorm{\v{\sigma_a}}_{\w_a}.
  \]
\end{proof}

The oscillation terms will also appear in the local equivalence with the standard residual estimator. 
The effects are summarized in the following lemma. The proof follows easily by mimicking the proof of Lemma \ref{lem:starequiv}
using the results from the previous Theorem.
Similarly Lemma's \ref{lem:loceff}, \ref{lem:globrel} and Theorem \ref{thm:locresequiv} concerning local bounds on $\w_a$  have
the following discretized counterpart. 

\begin{thm}
  Fix $a \in \V$. The discretized local residual provides reliability by
  \begin{align*}
    \enorm{u - U}_{\O}^2 &\leq 3\sum_{a\in \V} \left[\norm{\tilde r_a}_{H^1_\star(\w_a)'} + C_{PF,\w_a} h_{\w_a} \norm{(I - Q_a)(\psi_a f)}_{\w_a}\right]^2\\
    &\lesssim \sum_{a \in \V} \norm{\tilde r_a}_{H^1_\star(\w_a)'}^2 + h_{\w_a}^2\norm{(I - Q_a)(\psi_a f)}_{\w_a}^2.
  \end{align*}
  and local efficiency with
  \[
    \norm{\tilde r_a}_{H_\star^1(\w_a)'} \lesssim \enorm{u - U}_{\w_a} + h_{\w_a} \norm{(I - Q_a) (\psi_a f)}_{\w_a}.
  \]

\end{thm}
\begin{proof}
  Reliability follows from the triangle inequality in combination with Lemma~\ref{lem:globrel},
  \begin{align*}
    \enorm{u - U}^2_{\O} &\leq 3 \sum_{a \in \V} \norm{r_a}^2_{H_\star^1(\w_a)'}\\
    &\leq 3 \sum_{a \in \V} \left[ \norm{\tilde r_a}^2_{H_\star^1(\w_a)'} + \norm{r_a - \tilde r_a}^2_{H_\star^1(\w_a)'}\right]\\
    &\leq 3 \sum_{a\in \V} \left[\norm{\tilde r_a}_{H^1_\star(\w_a)'} + C_{PF,\w_a} h_{\w_a} \norm{(I - Q_a)(\psi_a f)}_{\w_a}\right]^2\\
    &\lesssim \sum_{a \in \V} \norm{\tilde r_a}_{H^1_\star(\w_a)'}^2 + h_{\w_a}^2\norm{(I - Q_a)(\psi_a f)}_{\w_a}^2,
  \end{align*}
  where we used Young's inequality $(x+y)^2 \leq 2x^2 + 2y^2$ for positive $x,y$.

  The efficiency result follows directly from the triangle inequality and Lemma \ref{lem:loceff}.

  The proof of Theorem~\ref{thm:locresequiv} holds for any functional $r \in H_\star^1(\w_a)'$ that can be decomposed in triangle and edge terms.
  In particular it holds for $\tilde r_a$, providing us with the last part of the theorem.
\end{proof}

  An element $K \in \wt \RR$ satisfies the interior node property. This implies that the three triangles adjacent to $K$ are also refined in going
  from $\T$ to $\T_\star$, i.e. $\w_K$ is entirely refined and thus $\w_K \in \RR^1$. Together with $\RR^3 \subset \RR^1$ this provides the desired bound:
  \[
    \eta_\T^2(U, \RR^3) \lesssim \enorm{U_\star - U}^2_{\O} + \osc^2_\T(f, \RR^1).
  \]



  For every triangle $K$ satisfying the interior node property, there exists a discrete lower bound for the classical estimator \cite[Lem~3.2]{mekchay2005convergence}:
  \[
    \hat \eta^2_\T(U, K) \lesssim \enorm{U_\star - U}^2_{\w_K} + \hat \osc_\T^2(f, \w_K) \quad \text{with } \w_K = \set{K' \in \T: K' \cap K \ne \emptyset}.
  \]
  Here $\hat \osc^2_\T(f, K) := h_K^2 \norm{(I - \Pi_{p-1})(f)}^2_{K}$ is the standard oscillation as in Definition~\ref{def:clasest}. Rewriting
  this classical oscillation term shows
  \[
    \hat \osc^2_\T(f, \w_K) = \sum_{K \subset \w_K} h_K^2 \norm{ (I - \Pi_{p-1})(f)}^2_K \lesssim h_{\w_K} \sum_{K \subset \w_K} \norm{(I - \Pi_{p-1})(f)}^2_{K} = \osc_\T^2(f, \w_K).
  \]

  The discrete reliability from equivalence with the classical residual estimator. Denote $\hat \eta_\T$ for the
  classical residual estimator (cf. Definition \ref{def:clasest}). 
  By Theorem~\ref{thm:residual_erro} with $\hat \RR = \T \setminus \T_\star$ we have a discrete reliability for the standard 
  estimator $\hat \eta_\T$: 
  \[
    \enorm{ U_\star -  U}^2 \lesssim \hat\eta^2_\T(U, \hat\RR) =\sum_{K \in \hat \RR}h^2_K\norm{f +\Delta U}_K^2+h_K \norm{\llbracket \nabla U  \rrbracket}^2_{\partial K \setminus \partial \O}.
  \]
  For $\V_K, \V_e$ the vertices of an element $K$ resp. an edge $e \subset \partial K \setminus \partial \O$ we infer that 
  \[
    h_K^2\norm{f + \Delta U}_K^2 \leq h_{\w_a}^2\norm{f + \Delta U}_{\w_a}^2 \quad \forall a \in \V_K, \quad h_K \norm{\llbracket \nabla U  \rrbracket}^2_{e} \leq h_{\w_a} \norm{\llbracket \nabla U \rrbracket}^2_{\gamma_a} \quad \forall a \in \V_e.
  \]
  The first inequality is obvious as $K \subset \w_a$ for $a \in \V_K$. 
  For the second inequality we note that $e \subset \partial K \setminus \partial \O$ is an interior edge, and thus that 
  that $e \subset \gamma_a$ for $a \in \V_e$.
  From these inequalities we deduce 
  \begin{align*}
    \enorm{ U_\star -  U}^2 &\lesssim\sum_{K \in \hat \RR}h^2_K\norm{f +\Delta U}_K^2+h_K \norm{\llbracket \nabla U  \rrbracket}^2_{\partial K \setminus \partial \O} \\
    &\leq  \sum_{K \in \hat \RR}\sum_{a \in \V_K} h_{\w_a}^2 \norm{f + \Delta U}_{\w_a}^2 + h_{\w_a} \norm{\llbracket \nabla U \rrbracket}_{\gamma_a}^2\\
  &\lesssim \eta^2_\T(U, \RR^0) + \osc_\T^2(f,  \RR^0) = \vartheta_\T^2(U, \RR^0),
\end{align*}
  where the last inequality follows from the equivalence between the two estimators on patches; see Lemma \ref{lem:locequivosc}.
  We also used $\RR^0 := \RR^0_{\T \to \T_\star} = \{\w_a \in \W : a \in \V_K, K \in \hat \RR\}$,
  and that the maximum number of triangles in a patch is universally bounded.



  The last equality follows from the discretized local lower bound for the standard estimator --- with $\hat \osc_\T(f, K) := h_K^2 \norm{f - \Pi_{p-1}(f)}^2_{K}$ the 
  standard oscillation as in Definition~\ref{def:clasest}.
  It is easily seen that the standard oscillation term is dominated, i.e.
  \begin{align*}
    \hat \osc_\T^2(f, \hat \RR) &= \sum_{K \in \hat \RR} h_K^2 \norm{(I - \Pi_{p-1})(f)}^2_K \leq \sum_{\w_a \in \RR}  \uanorm{h_a(I - \Pi^a_{p-1})( f)}^2_{\w_a} = \osc_\T^2(f, \RR).
  \end{align*}
  Since $\wt \RR \subset \RR$ we now have
  $\osc_\T^2(f, \wt \RR) + \hat \osc^2_\T(f, \hat \RR) \lesssim \osc_\T^2(f, \RR)$; together with \eqref{eq:disclowprf} this provides
  the discrete lower bound.


  {
    \color{blue}
    

      

    A start for discrete efficiency can be given. Let $\w_a \subset \RR$ be a patch that is refined, then 
    \begin{align*}
      \norm{r_a}_{\left(H^1_\star(\w_a) \cap \V(\T_\star)\right)'} &= \sup_{\{v \in H^1_\star(\w_a) \cap \V(\T_\star) : \norm{\nabla v}_{\w_a}=1\}}
      a(u - U, \psi_a v) \\
      &= \sup_{\{v \in H^1_\star(\w_a) \cap \V(\T_\star) : \norm{\nabla v}_{\w_a}=1\}} a(U_\star - U, \psi_a v) \\
      &\lesssim \enorm{U_\star - U}_{\w_a}.
    \end{align*}
    Under the assumption that $\norm{r_a}_{H_\star^1(\w_a)'} \lesssim  \norm{r_a}_{\left(H^1_\star(\w_a) \cap \V(\T_\star)\right)'}$ we  directly see that
    \[
      \uanorm{\vsiga}_{\w_a} \lesssim \norm{r_a}_{H_\star^1(\w_a)'} + \osc_\T(f,\w_a) \lesssim \enorm{U_\star - U}_{\w_a} + \osc_\T(f,\w_a).
    \]
    For the assumption to be true, we probably need to use the interior node property.
  }


  \begin{align*}
  \end{align*}
  Consider some element $K$ and a bisected descendent $K_\star$ of $K$. Under the current definitions
  we cannot ensure that $h_{K_\star} < h_K$, and we therefore cannot guarantee the strict oscillation reduction. 
  Alternatively we can consider $h'_K := (\vol K)^{1/2}$, for which we have $h'_{K_\star} \leq {1\over\sqrt{2}} h'_K$. 
  Shape regularity gives the equivalence $h'_K \approx h_K$, so that all the bounds in Lemma~\ref{lem:assumptions} are still
  valid if we replace $h_K$ by $h'_K$. For this oscillation reduction property we therefore implicitly replace $h_K$ by
  $h'_K$ in the  oscillation definition.
  For $K_\star \subset K$ we have $\norm{(I - \Pi_{p-1})(f)}_{K_\star} \leq \norm{(I - \Pi_{p-1})(f)}_K$.
  With the changed definition of $h_K$ we then discover that
  \[
    \sum_{K_\star \in \T_\star \setminus \T} h_{K_\star}^2 \norm{(I - \Pi_{p-1})(f)}^2_{K_\star} 
    \leq \frac{1}{2} \sum_{K \in \T \setminus \T_\star} h_K^2 \norm{(I - \Pi_{p-1})(f)}^2_K.
  \]

{
  \color{blue}
  How to proceed? I think it's possible. The set $\RR$ consists of all patches that have at least one triangle refined.
  Image a situation where all these patches in $\RR$ have a single triangle refined, but where the $\norm{(I - \Pi_{p-1})(f)}_K$ 
  blows for unrefined $K$. This norm appears on both sides of the inequality, and therefore we cannot find a $\lambda$.. 
}



  Next we tackle rewriting the minimizing problem. By the polarization identity
  \[
    \uanorm{\v{\zeta_a} + \psi_a \nabla U}_{\w_a}^2 = \uanorm{\v{\zeta_a}}_{\w_a}^2 + \norm{\psi_a \nabla U}^2_{\w_a} + 2 \ip{\v{\zeta_a}, \psi_a \nabla 
    U}_{\w_a}.
  \]
  Now $\psi_a \nabla U$ is fixed,  so minimizing $\uanorm{\v{\zeta_a} + \psi_a \nabla U}_{\w_a}$ for $\vzeta$ is equivalent to minimizing
  \[
    \frac{1}{2} \uanorm{\v{\zeta_a}}^2_{\w_a} + \ip{\v{\zeta_a}, \psi_a \nabla U}_{\w_a}.
  \]

There exists a basis $\set{\v{\tau_1},\dots, \v{\tau_n}}$ of $\RT_{p,0}(\w_a)$ and
  a basis $\set{q_1, \dots, q_M}$ of $\QQ(\w_a)$. Express the solution $\v{\zeta_a}$ in this basis  and use the previous paragraphs
  to  see that solving $\v{\zeta_a} \in \RT_{p,0}(\w_a)$ is
  equivalent to finding coefficients $\v{c} \in \R^n$ that minimize 
  \[
    J(\v{c}) := \frac{1}{2} \norm{\sum c_i \v{\tau_i}}^2_{\w_a} + \ip{\sum c_i \v{\tau_i}, \psi_a \nabla U}_{\w_a}
  \]
  subject to the constraint $G(\v{c}) = \v{0}$, where
  \[
    G_k(\v{c}) = \ip{\div \sum c_i \v{\tau_i}, q_k}_{\w_a} - \ip{\psi_a f - \nabla \psi_a \cdot \nabla U, q_k}_{\w_a}  \quad k \in \set{1, \dots, M}.
  \]
  This appears like a classical optimisation problem, and we can solve it using the Lagrange multiplier. 
  Let $\v{c}$ be a local minimum of $J$ satisfying the optimization constraint, then
  exists a Lagrange multiplier $\v{\lambda} \in \R^{M}$ such that 
   \begin{equation}
     \label{eq:lagrange}
     \nabla J(\v{c}) + \sum \lambda_k \nabla G_k(\v{c}) = 0\quad \text{and} \quad G(\v{c}) = 0.
   \end{equation}
   Calculating the partial derivatives of $J$ and $G_k$ for $c_j$ with $j=1,\dots, n$ gives
   \begin{align*}
     \frac{\partial J}{\partial c_j} &= \ip{ \sum c_i \v{\tau_i}, \v{\tau_j}}_{\w_a} + \ip{\v{\tau_j}, \psi_a \nabla U}_{\w_a},\\
     \frac{\partial G_k}{\partial c_j} &= \ip{\div \v{\tau_j}, q_k}_{\w_a}.
   \end{align*}
   Plug these derivatives into the Lagrange equation \eqref{eq:lagrange}; one directly sees that
   solving $(\v{c}, \v{\lambda})$ from \eqref{eq:lagrange} is equivalent to solving $(\v{\zeta_a}, \v{\lambda_a})$ from
   the system \eqref{eq:systemzeta} by setting $(\v{\zeta_a},{\v{\lambda_a}}) = (\sum c_i \tau_i, \sum \lambda_k q_k)$.
   In other words, every $\v{\zeta_a}$ that attains a local minimum indeed solves the sytem.





      which also preserves boundary values, see \cite{scott1990finite}. 
    Write $E_\star = U_\star - U$ for the error and let $V \in \VV(\T)$ be the following approximation of $E_\star$:
    \[
      V := I_kE_\star \quad \text{in } \O_k, \quad V:= E_\star \quad \text{elsewhere}.
    \]
    Since $\O_\RR$ contains all the patches that were refined, we see that $V \in \VV(\T)$.

      

  For $E_\star = U_\star - U$ and $V \in \VV(\T)$ we deduce from Galerkin orthogonality~that
  \[
  \]
  Let $V$ be the Scott-Zhang \cite{scott1990finite} interpolator of $E_\star$ so that $V - E_\star = 0$ on $\T \cap \T_\star$. 
  From the Scott-Zhang approximation theory \cite{scott1990finite} and Poincar\'e-Friedrichs inequality for $E_\star \in H_0^1(\O)$ we discover that
  \begin{align*}
    \ip{\vsig^\triangle , \nabla V - \nabla E_\star}_\O &= \sum_{K \in \T \setminus \T_\star} \ip{\vsig^\triangle, \nabla V - \nabla E_\star}_K\leq \sum_{K \in \T \setminus \T_\star} \uanorm{\vsig^\triangle}_K \norm{\nabla V - \nabla E_\star}_K\\
    &\lesssim \sum_{K \in \T \setminus \T_\star} \uanorm{\vsig^\triangle}_K\norm{E_\star}_{H^1(\w_K)}\\
    &\lesssim \norm{E_\star}_{H^1(\O)} \sqrt{\sum_{K \in \T \setminus \T_\star} \uanorm{\vsig^\triangle}^2_K} \\
    &\lesssim \enorm{E_\star}_{\O} \sqrt{\sum_{K \in \T \setminus \T_\star} \uanorm{\vsig^\triangle}^2_K}.
  \end{align*}
  Assume no oscillation occurs, then this gives the wanted local upper bound
  \[
    \enorm{U_\star - U}^2_\O \lesssim\sum_{K \in \T \setminus \T_\star} \uanorm{\vsig^\triangle}^2_K.
  \]
    

    Notice that $\VV(\T_\star) \subset \VV(\T)$, and thus from Galerkin orthogonality we deduce
    \[
      a(E_\star, V) = a(U_\star - u, V) + a(u - U, V) = 0 + 0 = 0.
    \]
    Invoking the partition of unity, and using that the difference $E_\star - V$ vanishes outside of~$\O_\RR$, which is the union
    of patches in $\RR$, shows
    \[
      \enorm{E_\star}^2_\O = a(E_\star, E_\star) = a(E_\star, E_\star - V) = \sum_{a \in \V} a(E_\star, (E_\star - V)\psi_a) = \sum_{\w_a \in \RR} a(E_\star, (E_\star - V)\psi_a).
    \]
    Here we also used that $\w_a = \supp(\psi_a)$. Notice that $(E_\star - V)\psi_a \in \VV(\T_\star)$, and thus using that $U_\star$ is the
    Galerkin approximation shows
    \begin{align*}
      \sum_{\w_a \in \RR} a(U_\star - U, (E_\star - V)\psi_a) &= \sum_{\w_a \in \RR} a(u - U, (E_\star - V)\psi_a) \\
      &= \sum_{\w_a \in \RR} \ip{r_a, E_\star - V} \\
      &\leq \sum_{\w_a \in \RR} \norm{r_a}_{H^1_\star(\w_a)'} \norm{\nabla \left(E_\star- V\right)}_{\w_a}\\
      &\leq \sum_{\w_a \in \RR} \norm{r_a}_{H^1_\star(\w_a)'} \norm{E_\star - V}_{H^1(\w_a)},
    \end{align*}
    where we used the definition of the local residual $r_a$. On a connected component $\O_k$ we have $E_\star - V = (I - I_k)(E_\star)$. 
    From the Scott-Zhang interpolation theory (cf. \cite[Thm~4.1]{scott1990finite}) we know that
    $\norm{(I - I_k)(E_\star)}_{H^1(\O_k)} \lesssim \norm{E_\star}_{H^1(\O_k)}$ for a constant only depending on shape regularity.
    Due to the finite overlap of patches that make up $\O_\star$ we deduce 
    \[
      \sum_{\w_a \in \RR} \norm{r_a}_{H^1_\star(\w_a)'} \norm{E_\star - V}_{H^1(\w_a)} \lesssim \norm{E_\star}_{H^1(\O_\RR)} \sqrt{\sum_{\w_a \in \RR} \norm{r_a}_{H^1_\star(\w_a)'}^2} \lesssim
      \norm{\nabla E_\star}_{\O} \sqrt{\sum_{\w_a \in \RR} \norm{r_a}^2_{H^1_\star(\w_a)'}}.
    \]
    The last inequality follows since $U, U_\star \in H_0^1(\O)$ and thus that $E_\star \in H_0^1(\O)$, which allows us to use Poincare-Friedrichs inequality.
    In the proof of Lemma~\ref{lem:clasequivlow} we have shown that $\norm{r_a}_{H^1_\star(\w_a)'} \leq \uanorm{\vsiga}_{\w_a} + \osc_\T(f,\w_a)$.
    Combining everything then proves the discrete lower bound, i.e.
    \[
      \enorm{U_\star - U}^2 \lesssim \sum_{\w_a \in \RR} \uanorm{\vsiga}^2_{\w_a} + \osc^2_\T(f,\w_a) = \eta_\T^2(U, \RR) + \osc^2_\T(f,\RR).
    \]
  \iffalse
  A piecewise polynomial on $\T_{k}$ is also a piecewise polynomial on $\T_{k+1}$, since $\T_{k+1} \geq \T_k$.
  Therefore we have $U_{k+1} - U_{k} \in \VV(\T_{k+1})$; combined with Galerkin orthogonality yields,
  \[
    \enorm{u - U_k}_{\O}^2 = \enorm{u - U_{k+1}}^2_{\O} + \enorm{U_{k+1} - U_{k}}^2_{\O} \implies e_{k+1}^2 = e_k^2 - E_k^2.
  \]
  Introducing constants $\gamma > 0, \delta > 0$ and $0 < \beta < 1$, one easily sees that
  \[
    e_{k+1}^2 + \gamma \osc_{k+1}^2 = e_k^2 - E_k^2 + \gamma \osc_{k+1}^2 \leq e_k^2 - \beta E_k^2 + (1 + \delta)\gamma \osc_{k+1}^2.
  \]
  Invoking the oscillation reduction from Lemma \ref{lem:oscasum} property results in,
  \begin{equation}
    \label{eq:ineq2}
    e_{k+1}^2 + \gamma \osc^2_{k+1} \leq e_k^2 - \beta E_k^2 + (1 + \delta)\gamma\left[ \osc^2_k - \lambda \osc^2_k(\RR_k)\right].
  \end{equation}
  Discrete efficiency from Lemma \ref{lem:assumptions} read as
  \[
    \enorm{U_{k+1} -  U_k}_\O \geq C_3 \eta_k^2(\RR_k) - \osc_k^2(\RR_k) \geq C_3 \eta_k^2(\mathcal{M}_k) - \osc_k^2(\RR_k),
  \]
  where the last inequality follows since $\mathcal{M}_k \subset \RR_k$.
  Inserting this into inequality \eqref{eq:ineq2} results in,
  \begin{align*}
    e_{k+1} + \gamma \osc^2_{k+1} &\leq e_k^2 - \beta \left[C_3 \eta_k^2(\mathcal{M}_k) - \osc_k^2(\RR_k)\right] +  (1 + \delta)\gamma\left[ \osc^2_k - \lambda \osc^2_k(\RR_k)\right] \\
    &\leq e_k^2 + (1 + \delta)\gamma \osc_k^2 - \beta C_3 \eta_k^2(\M_k) - \left[(1+\delta)\lambda \gamma - \beta\right]\osc^2_k(\RR_k).
  \end{align*}
  Let the  constants satisfy $(1 + \delta) \lambda \gamma\geq \beta$, 
  then we may replace $\osc_k^2(\RR_k)$ by the smaller $\osc_k^2(\mathcal{M}_k)$.
  Pick $\beta$ such that the coefficients of $\osc_k^2(\mathcal{M}_k)$ and $\eta_k^2(\M_k)$ match, i.e.
  \[
    (1 + \delta)\lambda \gamma - \beta = \beta C_3 \implies \beta = \frac{1}{1 + C_3} (1+\delta)\lambda \gamma
  \]
  then the inequality becomes
  \begin{align*}
    e_{k+1} + \gamma \osc^2_{k+1} \leq e_k^2 + (1 + \delta)\gamma \osc_k^2 - \frac{C_3}{1 + C_3} \gamma \lambda (1 + \delta) \left[\eta_k^2(\M_k) + \osc_k^2(\M_k)\right].
  \end{align*}
  This last term is per definition equal to $\vartheta_k^2(\M_k)$.
  By the D\"orfler marking property we have $\vartheta_k^2(\mathcal{M}_k) \geq \theta^2 \vartheta_k^2$, so that the above simplifies to
  \[
    %e_{k+1} + \gamma \osc^2_{k+1} \leq e_k^2 + (1 + \delta)\gamma \osc_k^2 - \frac{C_3}{1 + C_3} \gamma \lambda (1 + \delta)\theta^2\vartheta_k^2.
  \]
  Invoke the reliability bound from \ref{lem:assumptions} to reduce this to,
  \begin{align*}
 e_{k+1} + \gamma \osc^2_{k+1} &\leq (1 - C_1^{-1} C_3 \theta^2)e_k^2 + \gamma \osc^2_k - (\lambda \gamma - 1 - C_3) \osc_k^2(\mathcal{M}_k).
  \end{align*}
  \textcolor{blue}{
    Can we proceed? Does not seem like it.
  }
  \fi



The first recovery based estimator proposed by Zi
The first paper by Zienkiewicz-Zhu 
The initial ZZ-estimator p
Here one aims to construct a $\vsig^\star$ construct an approximation $\vsig^\star$ of $\nabla u$ that is of better quality than $\nabla U$. 
If this done correctly, we would get $\nabla u - \nabla U \approx \vsig^\star - \nabla U$ and thus the latter would provide
an estimator. Zienkewicz and Zhu use the fact that $\nabla U$ has better approximation quality
at certain points in the domain. They propose evaluating $\nabla U$ at these \emph{super-convergent} points and then using these
evaluations to \emph{recover} more accurate values of $\nabla u$ in the vertices. Here

A variety of strategies for gradient recovery is documented in literature, e.g. \cite{zienkiewicz1992superconvergent}. 
We will use a simple averaging strategy: for a linear finite element space $\VV(\T)$, the reconstructed gradient $\vsig^\star$ is constructed as
the function in $[\VV(\T)]^2$ with values in vertices $a \in \V$ given by
\[
  \vsig^\star(a) = \frac{1}{\vol(w_a)} \sum_{K \subset \w_a} \vol(K) \nabla U|_K
\]
The ZZ estimator is then simply $\norm{\vsig^\star - \nabla U}_K$. Notice that $\nabla U$ is piecewise constant, whereas
the recovered gradient $\vsig^\star$ is a (continuous) piecewise linear function. So $\vsig^\star$ is a
locally smoothened version of $\nabla U$ and therefore intuitively a better approximation of $\nabla u$. We will
not go into theoretical details about reliability and efficiency of this estimator.

Also note the patch-wise nature of this ZZ estimator, providing some resemblance with the equilibrated flux estimator.
It is clearly interesting to compare this cheap and simple ZZ estimator with the equilibrated flux estimator. An implementation
of the ZZ estimator 


\end{document}
