\documentclass[thesis.tex]{subfiles}
\begin{document}
For self-completeness this chapter contains summary of the finite element method ---
a method for approximating solutions of partial differential equations.
A complete mathematical derivation of the theory can be found in various books, e.g. \cite{brenner}. 
We shall focus on second-order elliptic boundary value problems. For a bounded Lipschitz domain $\Omega \subset \R^n$ the 
general boundary value problem with Dirichlet boundary conditions is given by
\begin{alignat}{2}
  -\nabla \cdot {\left(A \nabla u\right)} + \mathbf{b} \cdot \nabla u + c &= &f \quad  &\text {in } \Omega,  \label{eq:elliptic}\\
  u  &=& 0 \quad &\text{on } \partial\Omega.\nonumber
\end{alignat}
The theory works equally well for other boundary conditions as Neumann, or Mixed; but we omit
these types for ease of presentation and calculations. Under specific conditions on $A,\mathbf{b}, c$ 
and $f$ we can approximate the solution to this problem using the finite element method.

\section{Weak formulation}
The first step consists of tanslating the boundary value problem \eqref{eq:elliptic} into a \emph{weak formulation}.  
That is, we define a space $V$ of test functions and consider the problem of finding $u$ such that \eqref{eq:elliptic} holds 
in a distributional sense. Suppose that $u$ solves \eqref{eq:elliptic} with $A, \mathbf{b}, c, f$ sufficiently smooth functions, then for $v \in V$ --- which we will define later on--- we calculate
\begin{align*}
  \ip{f,v} = &\ip{-\nabla \cdot {\left(A \nabla u\right)} + \mathbf{b} \cdot \nabla u + c, v} \\
           = &\int_\Omega \left( -\nabla \cdot {\left(A(x) \nabla u(x)\right)}\right)v(x) + \left(\mathbf{b}(x) \cdot \nabla u(x)\right)v(x) + c(x) v(x) \, \dif x\\
           = &\int_{\partial \Omega} -\left(A(x) \nabla u(x) \cdot n\right)v(x) \, \dif x \\
              &+ \int_{\Omega} \left( A(x) \nabla u(x)\right) \cdot \nabla v(x) + \left(\mathbf{b}(x) \cdot \nabla u(x)\right)v(x) + c(x) v(x) \, \dif x.
\end{align*}
Where the third equality follows from the Divergence theorem, see \ref{thm:divergence}. Suppose that all test functions $v$ vanish on the boundary $\partial \Omega$, then the boundary integral disappears and we find
\begin{equation}
  \label{eq:ellipticweak}
  F(v) := \ip{f,v} = \ip{A \nabla u, \nabla v} + \ip{\mathbf{b} \cdot \nabla u + cu, v}=: a(u,v).
\end{equation}
A weak solution of \eqref{eq:elliptic} is a function $u$ that solves the above equation for all $v \in V$.
It turns out that under (mild) conditions there is a unique weak solution $u$. To see this
we must formalise the previous calculations. First we tackle the definition of $V$.

In the weak formulation we use the $L^2(\Omega)$-inner product, so we naturally require that $V \subset L^2(\Omega)$. Moreover, all the partial derivatives of $v$ are taken in a distributive sense. For these to be correctly defined we need that $v$ is at least weakly differentiable. Combining both requirements we see that $V$ should lie in the suitable space $W_2^1(\Omega) = H^1(\Omega)$ (conveniently this is a Hilbert space). First we must check that the calculations, in particular the Divergence theorem, are still valid for $v \in H^1(\Omega)$; of course this is the case, cf. \cite[Ch. 5]{brenner}. Formalizing the vanishing boundary requirement can be done using the trace operator $T$\footnote{Here we need that $\Omega$ has a Lipschitz boundary, cf. \cite[Ch. 1.6]{brenner}.}, i.e. we should have $v \in H^1(\Omega)$ such that $\|Tv\|_{L^2(\Omega)} = 0$. All together this yields
\[
  V := \set{v \in H^1(\Omega): v|_{\partial \Omega} = 0} = H_0^1(\Omega).
\]

The (famous) Lax-Milgram theorem can be used to show that certain weak formulations have a unique solution.
\begin{thm}[Lax-Milgram]
  \label{thm:lax}
  Given a Hilbert space $(V, \ip{\cdot, \cdot})$ and a bilinear form $a(\cdot,\cdot)$ on $V$ which is
  \begin{alignat*}{3}
    \text{Continuous} \quad & \exists C < \infty \quad & \text{s.t.} \quad & \abs{a(v,w)} \leq \|v\|\|w\| \quad \forall v,w \in V \\
    \text{Coercive}   \quad & \exists \alpha > 0 \quad &\text{s.t.} \quad & a(v,v) \geq \alpha \|v\|^2 \quad \forall v \in V
  \end{alignat*}
  Then, for every continuous linear functional $F \in V'$, there exists a unique $u \in V$ such that
  \[
    a(u,v) = F(v) \quad \forall v\in V.
  \]
\end{thm}
A proof of this theorem can be found in many books, e.g. \cite[Ch. 2]{brenner}. 

To use  Lax-Milgram for our weak formulation \eqref{eq:ellipticweak} we need restrictions on the parameters $A, \mathbf{b}, c$ and $f$. We assume that:
\begin{enumerate}[label=(\alph*)]
\item $A(x) = (a_{ij}(x))_{1 \leq i,j \leq n}$ is symmetric, $a_{ij} = a_{ji}$, with $a_{ij} \in L^{\infty}(\Omega)$. Furthermore, we suppose that the matrix is uniformly elliptic, i.e. there is a positive constant $\alpha$ such that 
\[
  \zeta \cdot (A(x) \zeta) \geq \alpha \|\zeta\|^2 \quad \forall \zeta \in \R^n \quad \text{a.e. in } \Omega;
\]
\item $\mathbf{b}(x) = \left(b_1(x), \dots, b_n(x)\right)^\top$ with $b_i \in L^\infty(\Omega)$ such that the weak divergence vanishes, i.e. $\nabla \cdot \mathbf{b} = 0$ in $\Omega$;
\item $c \in L^\infty(\Omega)$ is non-negative;
\item $f  \in L^2(\Omega)$.
\end{enumerate}
Given these assumptions Lax-Milgram can be applied to the weak formulation \eqref{eq:ellipticweak}. The form $a(\cdot, \cdot)$ is bilinear by linearity of the inner product and derivation, and the space $V = H_0^1(\Omega)$ is Hilbert. The functional $F$ is obviously linear, and since $f \in L^2(\Omega)$ we find using Cauchy-Schwarz that is continuous as well,
\[
  \abs{F(v)} = \abs{\ip{f,v}} \leq \|f\|_{L^2(\Omega)} \|v\|_{L^2(\Omega)} \leq \|f\|_{L^2(\Omega)} \|v\|_{H^1(\Omega)}.
\]
We are left to check the continuity and coercivity of the bilinear form.
\begin{description}
  \item[Continuity]Using H\"older we find
    \begin{align*}
      \abs{a(v,w)} \leq &\int_\Omega \abs{\sum_{i,j} a_{ij} \pd{v}{x_j}\pd{w}{x_i}  + \sum_i b_i \pd{v}{x_i}w + cvw}\\
                   \leq &\sum_{i,j} \norm{ a_{ij}}_{L^\infty(\Omega)} \norm{ \pd{v}{x_j}}_{L^2(\Omega)} \norm{ \pd{w}{x_i}}_{L^2(\Omega)} + \sum_i \norm{b_i}_{L^\infty(\Omega)} \norm{\pd{v}{_i}}_{L^2(\Omega)} \norm{w}_{L^2(\Omega)} \\
                   &+ \norm{c}_{L^2(\Omega)} \norm{v}_{L^2(\Omega)} \norm{w}_{L^2(\Omega)} \\
                   \leq &C \|v\|_{H^1(\Omega)} \|w\|_{H^1(\Omega)},
    \end{align*}
    where $C$ is a finite constant since $a_{ij}, b_i, c \in L^\infty(\Omega)$. So $a(\cdot,\cdot)$ is indeed continuous.
  \item[Coercivity] This is where our extra assumptions come into play. Since $\mathbf{b}$ has a vanishing weak divergence, a density argument\footnote{Recall that $C_c^\infty(\Omega)$ is dense in $H_0^1(\Omega)$.} shows that for $v \in H_0^1(\Omega)$ we have
    \[
      \int_\Omega \left(\mathbf{b} \cdot \nabla v\right)v  = \int_\Omega \mathbf{b} \cdot \left({1 \over 2} \nabla v^2\right) = - \int_\Omega \left(\nabla \cdot \mathbf{b} \right) {1 \over 2} v^2 = 0.
    \]
    By nonnegativity of $c$ we have $cv^2 \geq 0$ on $\Omega$. 
    The Poincar\'e-Friedrich  inequality --- stated in \ref{thm:poincare} --- provides us with a constant $C_\Omega$ such that $\|v\|_{L^2(\Omega)} \leq C_\Omega \|\nabla v\|_{L^2(\Omega)}$ for $v \in H_0^1(\Omega)$. Combining the previous with the uniform ellipticity of $A$ yields,
    \begin{align*}
      a(v,v) &= \int_\Omega \left(A \nabla v\right) \cdot \nabla v + \left(\mathbf{b} \cdot \nabla v\right)v + cv^2 \\
             &\geq  \int_\Omega \left(A(x) \nabla v(x)\right) \cdot \nabla v(x)\, \dif x \\
             &\geq  \int_\Omega \alpha \|\nabla v(x)\|^2\, \dif x = \alpha \|\nabla v\|^2_{L^2(\Omega)}\\
             &\geq \frac{\alpha}{2} \left(\frac{1}{C^2_\Omega} \|v\|^2_{L^2(\Omega)} + \|\nabla v\|^2_{L^2(\Omega)}\right) \geq \beta \|v\|^2_{H^1(\Omega)}.
    \end{align*}
    This proves the coercivity of $a(\cdot, c\dot)$.
\end{description}
By the above we may invoke Lax-Milgram to obtain the wanted result, namely that there is a unique $u \in V$ that solves the weak formulation \eqref{eq:ellipticweak}.

We cannot expect to solve the weak formulation in its current form since this is still an infinite dimensional problem.
Instead one generally solves the Galerkin approximation problem:
\begin{equation}
  \label{eq:galerk_finite}
  \text{Given a finite-dimensional } V_h \subset V \text{: find $u_h \in V_h$ s.t. } a(u_h, v) = F(v) \quad \forall v \in V_h.
\end{equation}
This is a finite dimensional problem, and carefully choosing subspaces $V_h$ can lead to increasing approximation accuracy of $u_h$. Another application of Lax-Milgram applied to $V_h$ shows that this system has a unique solution as well.

\begin{rem}
  We have shown that the weak formulation has a unique solution $u \in H_0^1(\Omega)$. We need more smoothness of $u$ to show that this is also a solution to the original boundary value problem.
\end{rem}

\section{Finite Element Space}
  Construction of subspaces $V_h$ can be done in a systematic way using \emph{finite elements}. 
  The general idea is to split the domain $\Omega$ into elements $K$, each with its own function space.
  The space $V_h$ is then constructed by glueing these pieces together. In this section we will summarize \cite[Ch~3]{brenner}.
  \begin{defn} 
    A (unisolvent) finite element is a triplet $(K, \mathcal{P}, \mathcal{N})$ where
    \begin{enumerate}[label=(\alph*)]
      \item $K \subset R^n$ is the \emph{element domain}: a bounded closet set with nonempty interior and piecewise boundary;
    \item $\mathcal{P}$ is the space of \emph{shape functions}: a finite-dimensional space of functions on $K$;
  \item $\mathcal{N} = \set{N_1, \dots, N_k}$ is the set of \emph{nodal variables}: a basis for $\mathcal{P}'$.
    \end{enumerate}
  \end{defn}
  \begin{defn}
    Let an element $(K, \mathcal{P}, \mathcal{N})$ be given. The basis $\set{\phi_1, \dots, \phi_k}$ of $\mathcal{P}$ dual to $\mathcal{N}$ --- so $N_i(\phi_j) = \delta_{ij}$ --- is called the \emph{nodal basis} of $\mathcal{P}$.
  \end{defn}
  \begin{lem}
    Let $\mathcal{P}$ be a $k$-dimensional vector space with $\mathcal{N} = \set{N_1, \dots, N_k} \subset \mathcal{P}'$.
    The set $\mathcal{N}$ is a basis for $\mathcal{P}'$ iff $\mathcal{N}$ determines $\mathcal{P}$, i.e. if $v \in \mathcal{P}$ with $N(v) = 0$ for all $N \in \mathcal{N}$ implies that $v = 0$.
  \end{lem}
  Informally a subspace $V_h$ can now be introduced. Consider a partition $\mathcal{T}$ of the domain $\Omega$ into finite elements,
  i.e. $\overline{\Omega} = \cup_{K \in \mathcal{T}} K$ where $(K, \mathcal{P}_K, \mathcal{N}_K)$ is a finite element for $K \in \mathcal{T}$.
  We can simply define $V_h$ to be the space of functions that coincide with the shape functions on each element,
  i.e. $v \in V_h$ if $v|_{K} \in \mathcal{P}_K$ for all $K \in \mathcal{T}$. 
  Without further conditions on the partition we have no way of giving smoothness  properties for the subspace $V_h$ --- we do not even know if it is correctly defined\footnote{In general, two adjacent elements have a non-finite intersection. 
  Therefore $v \in V_h$ might be incorrectly defined on this intersection in a classical sense.}.
  In case the exact solution $u$ is continuous it makes sense to use a subspace $V_h$ consisting of
  continuous functions as well. 

  In the literature a variety of finite elements are used, each element having its pros and cons. For now we will be interested in the \emph{Lagrange element}. For a triangle $K$ the Lagrange element of order 1 --- also called linear Lagrange --- is given by the triplet $(K, \mathcal{P}_1, \set{N_1, N_2, N_3})$, with $\mathcal{P}_1$ linear polynomials and functionals $N_i(v) = v(z_i)$, evaluation in the vertices $z_i$ of K. This element can be generalised to have shape functions $\mathcal{P}_k$ for of an arbitrary order $k$.

  To ensure extra properties of the subspace $V_h$ we need more regularity in the partition. In this work  we shall therefore consider subdivisions of the domain into simplices (triangles in $\R^2$, tetrahedra in $\R^3$). Implicitly we therefore also require $\Omega$ to have a polyhedral boundary. Moreover, we require the partition to be \emph{conforming}.
  \begin{defn}
    A triangulation $\mathcal{T}_h$ is a conforming partition of the domain $\Omega$ into a finite family of simplices.
    Formally,
    \begin{itemize}
      \item $\overline{\Omega} = \cup_{K \in \mathcal{T}_h} K$;
      \item $\mathring{K_i} \ne \emptyset, \mathring{K_i} \cap \mathring{K_j} = \emptyset \quad \forall K_i, K_j \in \mathcal{T}_h, K_i \ne K_j$;
    \item If $F = K_i \cap K_j$ for $K_i \ne K_j$, then $F$ is a common (lower dimensional) face of $K_i$ and~$K_j$.
  \end{itemize}
  \end{defn}
  
  Let a triangulation $\mathcal{T}_h$ be given. Suppose that we consider the subspace $V_h$ consisting
  of continuous functions $v$ that are linear polynomials when restricted to a triangle $K \in \mathcal{T}_h$. Why
  would one look at finite elements $(K, \mathcal{P}, \mathcal{N})$? 
  Instead one could just try to find a basis for this subspace, without ever needing to define finite elements.
  It appears however, that for analysis of the approximation error these finite elements are extremely useful.

  Linear Lagrange elements are an obvious choice as finite elements, if we want to construct the continuous piecewise linear subspace $V_h$.
  Every function $v \in V_h$ is completely
  determined by its value in the vertices of $\mathcal{T}_h$, because of continuity and piecewise linearity.
  This precisely coincides with the choice of the local nodal variables. We can therefore lift these local nodal variables to become
  global variables of the bigger space $V_h'$.
  
  Formally, the global nodal variables $\N_\O$ are the local nodal variables $\mathcal{N}$ that are also defined for $C(\OO)'$. Certain
  local nodal variables will induce the same global nodal variable. In the linear Lagrange case for example, we see that all nodal variables
  associated to one vertex $z$ will induce the same global nodal variable. We define the finite element space $V_h$ as the span of the 
  nodal basis with respect to $\N_\O$. 
  
  In the linear Lagrange case we see that $V_h$ is indeed the space of continuous linear piecewise
  functions with respect to $\mathcal{T}_h$. In the previous we ignored
  the boundary condition, recall that $v \in V_h$ must vanish on the boundary. This is easily fixed; we can just consider the (non-trivial)
  nodal variables restricted to $C_0(\OO)'$. In the linear Lagrange case this corresponds to removing nodal variables associated with
  boundary vertices.  
  
  Recall that for the Galerkin approximation \eqref{eq:galerk_finite} we actually must have $V_h \subset H^1_0(\O)$. This
  condition is satisfied when using Lagrange elements, as summarized in the next lemma.
  \begin{lem}
    \label{lem:lin_lagrange}
    Let $\mathcal{T}_h$ be a triangulation of the domain $\O$, with $V_h$ the finite element space generated
    by using Lagrange elements. Then $v \in V_h$ is continuous and weak differentiable, i.e. $v \in C(\OO) \cap H^1(\OO)$. 
  \end{lem}
  
  \textcolor{blue}{This section is vague. Delete or shorten it. }
\section{Interpolant}
\label{sec:apriori}
The literature the element space is frequently defined in terms of the \emph{interpolant} (c.f. \cite[Ch~3]{brenner}). This interpolant is also
  an important tool in deriving error bounds. 
  \begin{defn}
    Given a finite element $(K, \mathcal{P}, \mathcal{N})$ with nodal basis $\set{\phi_1,\dots, \phi_k}$. Then for a function $v$ for which all of the nodal variables $N \in \mathcal{N}$ are defined, the local interpolant is given by
    \[
      I_K v = \sum_{i = 1}^k N_i(v)\phi_i.
    \]
  \end{defn}
  \begin{defn}
    Given a triangulation $\mathcal{T}_h$ where each $K \in \mathcal{T}_h$ has an associated finite element triplet $(K, \mathcal{P}, \mathcal{N})$. Then for
    a function $v$ which is in the domain of each local interpolant the global interpolant reads
    \[
      I_{\mathcal{T}_h} v |_K = I|_K v \quad \forall K \in \mathcal{T}_h.
    \]
  \end{defn}
  We would like this global interpolant to preserve some regularity. Most elements provide such a kind of conformity. Consider
  linear Lagrange elements again; one easily sees that the global interpolant can be expressed using the global nodal variables, i.e.
    \[
      I_{\mathcal{T}_h} v = \sum_{i=1}^m N_i(v)\Phi_i \quad \text{ for } \N_\O =\set{N_1, \dots, N_m} \text{ and its global nodal basis } \Phi_1,\dots,\Phi_m.
    \]
    From Lemma~\ref{lem:lin_lagrange} we see that for $f \in C(\OO)$ we have $I_{\mathcal{T}_h} f \in C^0(\OO) \cap H^1(\OO)$. We say
    that the interpolant has \emph{continuity order 0}, and the finite element space $V_h$ is said to be a $C^0$ finite element space.
  \section{Error bounds}
  Given the finite element space $V_h$, one can solve the Galerkin approximation $\eqref{eq:galerk_finite}$. The immediate
  question is of course `what is the quality of the approximation'. In more mathematical terms, can we find (tight) bounds
  on the approximation error?

  The first such bound is provided by C\'ea's lemma (c.f. \cite[Thm~2.8.1]{brenner}).
  \begin{lem}
    Suppose the same conditions as in the Lax-Milgram theorem~\ref{thm:lax} hold. Then for the solution $u_h \in V_h$ of the
    finite Galerkin approximation \eqref{eq:galerk_finite} we have
    \[
      \| u - u_h\| \leq {C \over \alpha} \min_{v \in V_h} \| u - v\|,
    \]
    where $C$ is the continuity constant and $\alpha$ is the coercivity constant of $a(c\dot, \cdot)$ on $V$.
  \end{lem}
  The norm is the same as on $V$; so this is the $H^1$-norm in case of our elliptic boundary value problem \eqref{eq:elliptic}.
  Loosely, the above lemma tells us that the finite element solution $u_h$ is the `best' approximation from the subspace $V_h$. We are
  only left to find an element in $v \in V_h$ for which we can bound $\|u - v\|$. This is where the interpolant comes into play, since $I_{\T_h}u \in V_h$ if we assume that $u \in C(\OO)$.

  The general idea is to bound $\|v - I_{\mathcal{T}_h} v\|$ by (universally) bounding the error restricted to each of the elements $K\in\T_h$.  
  First an interpolation error bound is found for a \emph{reference element} $\hat K$. Then, under certain conditions, this bound can be used to estimate the interpolation error on each of the elements $K \in \T_h$. This idea of first considering a reference element is often 
  used in finite element analysis and implementation. The extra conditions are formalised by the following definitions.
  \begin{defn}
    A finite element $(K, \P, \N)$ is affine-interpolation equivalent to $(\hat K, \hat \P, \hat \N)$ if there
    exists an affine map $ F(\hat x) = B\hat x + c$ ($B$ non singular) such that for $f = \hat f \circ F$,
    \begin{enumerate}[label=(\roman*)]
      \item $F(K) = \hat K$;
      \item $\P = \{f : \hat f \in \hat \P\}$;
    \item $I_{\hat \N} \hat f = \left( I_{\N} f \right)\circ F$ for all sufficiently smooth $\hat f$.
    \end{enumerate}
  \end{defn}
  \begin{defn}
    Write $h_K := \text{diam}(K)$ and $p_K := \sup\set{\text{diam}(S): S \text{ ball }, S \subset K}$.
    A~family of finite elements $(K, \mathcal{P}, \mathcal{N})$ is \emph{uniformly shape regular} if 
    \[
      \sup_K h_k/p_k < \infty.
    \]
  \end{defn}

  Using the \emph{Sobolev inequalities}, the \emph{Bramble-Hilbert lemma} and the \emph{Transformation lemma}, one
  can proof the following (see \cite[Ch~3]{chen} or \cite{stevenson} for a concise overview).
  \begin{thm}
    Let $(\hat K, \hat \P, \hat \N)$ be finite element with $\hat K \subset \R^n$. Suppose that $m - n/ 2 > l$,
    \[
      \P_{m-1}(\hat K) \subset \hat \P \subset H^m(\hat K) \quad \text{and} \quad \hat N \subset  C^l(\hat K)'.
    \]
    
    Then there exists a constant $C(\hat K, \hat P, \hat N)$ such that for all $(K, \P, \N)$ affine-interpolation
      equivalent to $(\hat K, \hat P, \hat N)$, and for $0 \leq s \leq m$,
      \[
        \abs{v - I_K v}_{H^s(K)} \leq C \frac{h_K^{m}}{p_K^s} \abs{v}_{H^m(K)} \quad  \forall v \in H^m(\O).
      \]
      If in addition a family $(K, \P, \N)$ is also uniformly shape regular the above is reduced to,
      \[
        \abs{v - I_K v}_{H^s(K)} \leq C h_K^{m -s} \abs{v}_{H^m(K)} \quad \forall v \in H^m(\O).
      \]
  \end{thm}
  This theorem provides us with the wanted local error bounds. As a consequence we can now
  give an error bound for the global interpolant.
  \begin{thm}
    Let $(\mathcal{T}_h)$ be a family of uniformly shape regular subdivisions of a domain~$\O \subset \R^n$.
    Suppose that all the finite elements are affine-interpolation equivalent to a reference element $(\hat K, \hat P, \hat N)$,
    then under the conditions of the previous theorem we find for $0 \leq s \leq m$ and $h  := \max_{ K \in \T_h} h_K$,
    \[
      \left( \sum_{K \in \T_h} \norm{v - I_K v}^2_{H^s(K)}\right)^{1/2} \leq C h^{m-s} \abs{v}_{H^m(\O)} \quad \forall v \in H^m(\Omega).
    \]
    In case the global interpolant satisfies $I_{\T_h} C^l(\OO) \subset C^{m-1}(\OO)$, the above left hand side becomes a norm, i.e. 
    \[
      \norm{v - I_{\T_h}v}_{H^s(K)} \leq C h^{m-s} \abs{v}_{H^m(\O)} \quad \forall v \in H^m(\Omega).
    \]
  \end{thm}
  Assume we approximate the elliptic problem \eqref{eq:elliptic} with Lagrange elements in the plane. Additionally, if
  the solution $u$ satisfies $u \in H_0^1(\O)\cap H^2(\O)$, then C\'ea's lemma combined with the above theorem tells us that\footnote{
  Here we also used that the global interpolant preserves the Dirichlet boundary condition, i.e.
   $I_{\T_h} u$ is actually an element of $V_h \subset H_0^1(\O)$.}
  \begin{equation}
    \label{eq:conv_h}
    \|u - u_h\|_{H^1(\O)} \leq C \min_{v \in V_h} \| u - v\|_{H^1(\O)} \leq C \| u - I_{\T_h}u \|_{H^1(\O)} \leq C h \abs{u}_{H^2(\O)}.
  \end{equation}
  Which provides us with a concrete convergence proof. 

  \section{A posteriori error estimation}
  \label{sec:afem}
  The last result \eqref{eq:conv_h} shows that the approximation error declines upon solving $u_h$
  for finite element spaces with decreasing $h$. This so called \emph{a priori} estimator provides an error bound
  using problem specific information, e.g. $u \in H^2$. It does not capture
  any local information; it does not tell \emph{where} the approximation error is substantial. 
  
  One might suspect that sharpened error bounds can be given if we actually use $u_h$.
  Instead of globally reducing $h$, it might also be more efficient to consider partitions $\T_h$ with
  the diameter only locally reduced. This could lead to convergence of problems where $u$ is not sufficiently smooth --- for example $u \not \in H^2$. This is where \emph{a posteriori} error estimators are useful. Given an $u_h$,
  an a posteriori estimator can be used to indicate on which elements $K \in \T_h$ the error $\|u - u_h\|_K$ is relatively large.

  We follow the terminology given in \cite{stevenson}. For simplicity  we will consider the Poisson problem, i.e. given a polyhedral domain~$\O \subset \R^n$ and $f \in L^2(\O)$, solve for $u$:
  \begin{alignat}{2}
    -\Delta u &=&f \quad &\text {in } \Omega, \label{eq:poisson} \\
    u &=& 0 \quad &\text{on } \partial\O. \nonumber
  \end{alignat}
  The bilinear form of the weak formulation is given by $a(v,w) = \ip{\nabla v, \nabla w}_\O$ with
  test space $V = H_0^1(\O)$. This form
  induces the energy seminorm,
  \[
    \enorm{v}_\O := a(v,v)^{1/2} = \ip{\nabla v, \nabla v}_\O^{1/2} = \abs{v}_{H^1(\O)} \quad \forall v \in V,
  \]
  which is a norm on $V$ thanks to the Poincar\'e-Friedrichs inequality.

  Let $\T$ be a uniformly shape regular triangulation of $\O$, with $\VV(\T)$ the Lagrange finite element space of degree $p$, i.e. 
  \[
    \VV(\T) = \set{ v \in H_0^1(\O ) \cap C(\O) :  v|_K \in \P_p(K)}.
  \] 
  \todo{Replace $u_h$ by $U$}
  The classical error estimator is based on the \emph{residual} $e_h := u - u_h$, with $u_h\in \VV(\T)$ the discrete solution. Inserting this in the bilinear form yields for $v \in V$,
  \begin{align*}
    a(e_h, v) &= \ip{f,v} - a(u_h, v) \\
              &= \sum_{K \in \T} \int_K fv + \nabla u_h \cdot \nabla v \\
              &= \sum_{K \in \T} \int_K (f + \Delta u_h)v+\int_{\partial K} \left(\nabla u_h \cdot n\right) v\\
    &= \sum_{K \in \T} \int_K (f + \Delta u_h)v+\int_{e \in \E^{int}} \left \llbracket \nabla u_h \right \rrbracket  v.
  \end{align*}
  Here $\llbracket \nabla u_h \rrbracket$ is the jump of $\nabla u_h \cdot n$ over an interface in the direction of the unit normal~$n$,
  \[
    \llbracket \nabla u_h \rrbracket(x) := \lim_{\epsilon \to 0} \nabla u_h(x + \epsilon n) \cdot n - \nabla u_h(x - \epsilon n)\cdot n,
  \]
  and~$\E^{int}$~is the set of interior
  edges of $\T$. The equalities follow from integration by parts, and the fact that $v$ vanishes on the exterior edges of $\T$.
  Since $I_\T v \in \VV(\T)$ we find from the Galerkin orthogonality that
  \[
    a(e_h, v) = a(e_h, v - I_\T v) \leq \sum_{K \in \T} \| f + \Delta u_h\|_{L^2(K)} \| v - I_\T v\|_{L^2(K)} + \sum_{e \in \E^{int}} \|\llbracket \nabla u_h \rrbracket \|_{L^2(e)} \|v - I_\T v\|_{L^2(e)}.
  \]
  The norms of  $v - I_\T v$ can be estimated using the interpolation theory.\footnote{Technically we use a different interpolant here than the one given in the previous section, since this required $v$ to have extra smoothness. The so-call \emph{Scott-Zhang} interpolant does the trick (c.f. \cite[Ch~4.9]{brenner}).} This gives reason for the following definitions.
  \begin{defn}
    \label{def:clasest}
    For $K \in \T$ and $v \in \VV(\T)$ the \emph{error indicator} for $v$ on $K$ reads
    \[
      \eta(v, K)^2 := h_K^2 \norm{f + \Delta v}^2_{L^2(K)} + h_K\norm{\llbracket \nabla v \rrbracket}^2_{L^2(\partial K\setminus \partial \O)}.
    \]
    The \emph{oscillation term} of $f$ on $K$ is given by
    \[
      \osc(f,K)^2 := h_K^2 \norm{f - P_K^r f}^2_{L^2(K)} \quad \text{ for some } r \geq p-2,
    \]
    where $P_K^r$ is the $L^2(K)$-orthogonal projector on polynomials of degree $r$ on $K$.

    For $\M \subset \T$ the above terms are simply defined as the sum over $K \in \M$, i.e.
    \[
      \eta(v, \M)^2 := \sum_{K \in \M} \eta(v, K)^2, \quad \text{and} \quad \osc(f, \M)^2 := \sum_{K \in \M} \osc(f,K)^2.
    \]
  \end{defn}
  Denote $\T_\star \geq \T$ if $\T_\star$ is a refinement of $\T$, and let $R_{\T \to \T_\star}$ be the set of refined elements, 
  i.e. $R_{\T \to \T_\star} = \T\setminus \T_\star$. To distinguish between finite element solutions, denote  $u_\T$ for the Galerkin approximation in $\VV(\T)$. The following theorem holds (see \cite{stevenson} for a proof).
  \begin{thm}
    \label{thm:residual_erro}
    There exists $C_1$ such that for $\T_\star \geq \T$ we have 
    \[
      \enorm{u_{\T_\star} - u_{\T}}^2_{\O} \leq C_1 \eta (u_\T, R_{\T \to \T_\star})^2 \quad \text{and} \quad \enorm{u - u_{\T}}^2_{\O} \leq C_1 \eta (u_\T, \T)^2.
    \]

    Similarly, there exists a $C_2$ such that
    \[
      \eta(u_\T, \T)^2 \leq C_2 \left [\enorm{u - u_\T}^2_{\O} + \osc(f, \T)^2 \right].
    \]
  \end{thm}
  The estimator $\eta(u_\T, \T)^2$ provides an upper- \emph{and} lower bound for the total error:~$\enorm{u-u_\T}^2_{\O} + \osc(f,\T)^2$.
  That is, the error estimator is proportional to the total error. In practice the $\osc(f,\T)^2$ is magnitudes smaller 
  than $\eta(u_\T,\T)^2$; in this case, the estimator $\eta(u_\T, \T)^2$ is proportional to the approximation error.

  TODO: More references.
  \section{Adaptive finite element method}
  \label{sec:afem}
  The difference between the a priori estimator from \S\ref{sec:apriori} is that $\eta(u_\T, \T)$ consists of known quantities: $f, u_\T$. 
  Therefore $\eta(u_\T,\T)$ can be calculated to find out \emph{where} the approximation error is big. This leads to the
  very intuitive algorithm, called the \emph{adaptive finite element method}:
  \[
    TODO
  \]

  Of course, this method is accomplished with a convergence proof.
  \begin{thm}
    There exists constants $\gamma > 0$ and $\alpha \in (0,1)$, such that
    \[
      \enorm{u - u_{k+1}}^2_{\O} + \gamma \eta \left(u_{k+1}, \T_{k+1}\right)^2 \leq \alpha\left(\enorm{u - u_k}^2_{\O} + \gamma \eta(u_k, \T_k)^2\right)
    \]
  \end{thm}
  One can even prove that the convergence is with the best possible rate. To formalise this we introduce an approximation class.
  \begin{defn}
    For $s > 0$, the approximation class $\A^s$ is defined by
    \begin{align*}
      \A^s := \{ u \in H_0^1(\O) :& \Delta u \in L^2(\O),\\
                                  & \abs{u}_{\A^s} := \sup_{N \in \NN} \left(N + 1\right)^s \min_{\set{\T \in \TT : \#\T -\# \T_0 \leq N}} \sqrt{\enorm{u - u_\T}^2_{\O} + \osc(f,\T)^2} < \infty\}.
    \end{align*}
  \end{defn}
  We have the following theorem (cite TODO)
  \begin{thm}
    Let $C_1, C_2$ be as in Theorem~\ref{thm:residual_erro}. Ensure that $\theta^2 < \left(C_1(C_2+1)\right)^{-1}$.
    Take $u \in \A^s$ for some $s > 0$, then there is a $C$ such that
    \[
      \# \T_k - \# T_0 \leq C \abs{u}^{1/s}_{\A^s} \left( \sqrt{ \enorm{u - u_k}^2_{\O} + \osc(\T_k, f)^2}\right)^{-1/s}.
    \]
  \end{thm}
  
  
  
\end{document}
